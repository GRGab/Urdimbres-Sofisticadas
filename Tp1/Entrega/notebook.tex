
% Default to the notebook output style

    


% Inherit from the specified cell style.




    
\documentclass[11pt]{article}

    
    
    \usepackage[T1]{fontenc}
    % Nicer default font (+ math font) than Computer Modern for most use cases
    \usepackage{mathpazo}

    % Basic figure setup, for now with no caption control since it's done
    % automatically by Pandoc (which extracts ![](path) syntax from Markdown).
    \usepackage{graphicx}
    % We will generate all images so they have a width \maxwidth. This means
    % that they will get their normal width if they fit onto the page, but
    % are scaled down if they would overflow the margins.
    \makeatletter
    \def\maxwidth{\ifdim\Gin@nat@width>\linewidth\linewidth
    \else\Gin@nat@width\fi}
    \makeatother
    \let\Oldincludegraphics\includegraphics
    % Set max figure width to be 80% of text width, for now hardcoded.
    \renewcommand{\includegraphics}[1]{\Oldincludegraphics[width=.8\maxwidth]{#1}}
    % Ensure that by default, figures have no caption (until we provide a
    % proper Figure object with a Caption API and a way to capture that
    % in the conversion process - todo).
    \usepackage{caption}
    \DeclareCaptionLabelFormat{nolabel}{}
    \captionsetup{labelformat=nolabel}

    \usepackage{adjustbox} % Used to constrain images to a maximum size 
    \usepackage{xcolor} % Allow colors to be defined
    \usepackage{enumerate} % Needed for markdown enumerations to work
    \usepackage{geometry} % Used to adjust the document margins
    \usepackage{amsmath} % Equations
    \usepackage{amssymb} % Equations
    \usepackage{textcomp} % defines textquotesingle
    % Hack from http://tex.stackexchange.com/a/47451/13684:
    \AtBeginDocument{%
        \def\PYZsq{\textquotesingle}% Upright quotes in Pygmentized code
    }
    \usepackage{upquote} % Upright quotes for verbatim code
    \usepackage{eurosym} % defines \euro
    \usepackage[mathletters]{ucs} % Extended unicode (utf-8) support
    \usepackage[utf8x]{inputenc} % Allow utf-8 characters in the tex document
    \usepackage{fancyvrb} % verbatim replacement that allows latex
    \usepackage{grffile} % extends the file name processing of package graphics 
                         % to support a larger range 
    % The hyperref package gives us a pdf with properly built
    % internal navigation ('pdf bookmarks' for the table of contents,
    % internal cross-reference links, web links for URLs, etc.)
    \usepackage{hyperref}
    \usepackage{longtable} % longtable support required by pandoc >1.10
    \usepackage{booktabs}  % table support for pandoc > 1.12.2
    \usepackage[inline]{enumitem} % IRkernel/repr support (it uses the enumerate* environment)
    \usepackage[normalem]{ulem} % ulem is needed to support strikethroughs (\sout)
                                % normalem makes italics be italics, not underlines
    

    
    
    % Colors for the hyperref package
    \definecolor{urlcolor}{rgb}{0,.145,.698}
    \definecolor{linkcolor}{rgb}{.71,0.21,0.01}
    \definecolor{citecolor}{rgb}{.12,.54,.11}

    % ANSI colors
    \definecolor{ansi-black}{HTML}{3E424D}
    \definecolor{ansi-black-intense}{HTML}{282C36}
    \definecolor{ansi-red}{HTML}{E75C58}
    \definecolor{ansi-red-intense}{HTML}{B22B31}
    \definecolor{ansi-green}{HTML}{00A250}
    \definecolor{ansi-green-intense}{HTML}{007427}
    \definecolor{ansi-yellow}{HTML}{DDB62B}
    \definecolor{ansi-yellow-intense}{HTML}{B27D12}
    \definecolor{ansi-blue}{HTML}{208FFB}
    \definecolor{ansi-blue-intense}{HTML}{0065CA}
    \definecolor{ansi-magenta}{HTML}{D160C4}
    \definecolor{ansi-magenta-intense}{HTML}{A03196}
    \definecolor{ansi-cyan}{HTML}{60C6C8}
    \definecolor{ansi-cyan-intense}{HTML}{258F8F}
    \definecolor{ansi-white}{HTML}{C5C1B4}
    \definecolor{ansi-white-intense}{HTML}{A1A6B2}

    % commands and environments needed by pandoc snippets
    % extracted from the output of `pandoc -s`
    \providecommand{\tightlist}{%
      \setlength{\itemsep}{0pt}\setlength{\parskip}{0pt}}
    \DefineVerbatimEnvironment{Highlighting}{Verbatim}{commandchars=\\\{\}}
    % Add ',fontsize=\small' for more characters per line
    \newenvironment{Shaded}{}{}
    \newcommand{\KeywordTok}[1]{\textcolor[rgb]{0.00,0.44,0.13}{\textbf{{#1}}}}
    \newcommand{\DataTypeTok}[1]{\textcolor[rgb]{0.56,0.13,0.00}{{#1}}}
    \newcommand{\DecValTok}[1]{\textcolor[rgb]{0.25,0.63,0.44}{{#1}}}
    \newcommand{\BaseNTok}[1]{\textcolor[rgb]{0.25,0.63,0.44}{{#1}}}
    \newcommand{\FloatTok}[1]{\textcolor[rgb]{0.25,0.63,0.44}{{#1}}}
    \newcommand{\CharTok}[1]{\textcolor[rgb]{0.25,0.44,0.63}{{#1}}}
    \newcommand{\StringTok}[1]{\textcolor[rgb]{0.25,0.44,0.63}{{#1}}}
    \newcommand{\CommentTok}[1]{\textcolor[rgb]{0.38,0.63,0.69}{\textit{{#1}}}}
    \newcommand{\OtherTok}[1]{\textcolor[rgb]{0.00,0.44,0.13}{{#1}}}
    \newcommand{\AlertTok}[1]{\textcolor[rgb]{1.00,0.00,0.00}{\textbf{{#1}}}}
    \newcommand{\FunctionTok}[1]{\textcolor[rgb]{0.02,0.16,0.49}{{#1}}}
    \newcommand{\RegionMarkerTok}[1]{{#1}}
    \newcommand{\ErrorTok}[1]{\textcolor[rgb]{1.00,0.00,0.00}{\textbf{{#1}}}}
    \newcommand{\NormalTok}[1]{{#1}}
    
    % Additional commands for more recent versions of Pandoc
    \newcommand{\ConstantTok}[1]{\textcolor[rgb]{0.53,0.00,0.00}{{#1}}}
    \newcommand{\SpecialCharTok}[1]{\textcolor[rgb]{0.25,0.44,0.63}{{#1}}}
    \newcommand{\VerbatimStringTok}[1]{\textcolor[rgb]{0.25,0.44,0.63}{{#1}}}
    \newcommand{\SpecialStringTok}[1]{\textcolor[rgb]{0.73,0.40,0.53}{{#1}}}
    \newcommand{\ImportTok}[1]{{#1}}
    \newcommand{\DocumentationTok}[1]{\textcolor[rgb]{0.73,0.13,0.13}{\textit{{#1}}}}
    \newcommand{\AnnotationTok}[1]{\textcolor[rgb]{0.38,0.63,0.69}{\textbf{\textit{{#1}}}}}
    \newcommand{\CommentVarTok}[1]{\textcolor[rgb]{0.38,0.63,0.69}{\textbf{\textit{{#1}}}}}
    \newcommand{\VariableTok}[1]{\textcolor[rgb]{0.10,0.09,0.49}{{#1}}}
    \newcommand{\ControlFlowTok}[1]{\textcolor[rgb]{0.00,0.44,0.13}{\textbf{{#1}}}}
    \newcommand{\OperatorTok}[1]{\textcolor[rgb]{0.40,0.40,0.40}{{#1}}}
    \newcommand{\BuiltInTok}[1]{{#1}}
    \newcommand{\ExtensionTok}[1]{{#1}}
    \newcommand{\PreprocessorTok}[1]{\textcolor[rgb]{0.74,0.48,0.00}{{#1}}}
    \newcommand{\AttributeTok}[1]{\textcolor[rgb]{0.49,0.56,0.16}{{#1}}}
    \newcommand{\InformationTok}[1]{\textcolor[rgb]{0.38,0.63,0.69}{\textbf{\textit{{#1}}}}}
    \newcommand{\WarningTok}[1]{\textcolor[rgb]{0.38,0.63,0.69}{\textbf{\textit{{#1}}}}}
    
    
    % Define a nice break command that doesn't care if a line doesn't already
    % exist.
    \def\br{\hspace*{\fill} \\* }
    % Math Jax compatability definitions
    \def\gt{>}
    \def\lt{<}
    % Document parameters
    \title{TC01\_FerreiraChase\_Leizerovich\_Goren}
    
    
    

    % Pygments definitions
    
\makeatletter
\def\PY@reset{\let\PY@it=\relax \let\PY@bf=\relax%
    \let\PY@ul=\relax \let\PY@tc=\relax%
    \let\PY@bc=\relax \let\PY@ff=\relax}
\def\PY@tok#1{\csname PY@tok@#1\endcsname}
\def\PY@toks#1+{\ifx\relax#1\empty\else%
    \PY@tok{#1}\expandafter\PY@toks\fi}
\def\PY@do#1{\PY@bc{\PY@tc{\PY@ul{%
    \PY@it{\PY@bf{\PY@ff{#1}}}}}}}
\def\PY#1#2{\PY@reset\PY@toks#1+\relax+\PY@do{#2}}

\expandafter\def\csname PY@tok@gd\endcsname{\def\PY@tc##1{\textcolor[rgb]{0.63,0.00,0.00}{##1}}}
\expandafter\def\csname PY@tok@gu\endcsname{\let\PY@bf=\textbf\def\PY@tc##1{\textcolor[rgb]{0.50,0.00,0.50}{##1}}}
\expandafter\def\csname PY@tok@gt\endcsname{\def\PY@tc##1{\textcolor[rgb]{0.00,0.27,0.87}{##1}}}
\expandafter\def\csname PY@tok@gs\endcsname{\let\PY@bf=\textbf}
\expandafter\def\csname PY@tok@gr\endcsname{\def\PY@tc##1{\textcolor[rgb]{1.00,0.00,0.00}{##1}}}
\expandafter\def\csname PY@tok@cm\endcsname{\let\PY@it=\textit\def\PY@tc##1{\textcolor[rgb]{0.25,0.50,0.50}{##1}}}
\expandafter\def\csname PY@tok@vg\endcsname{\def\PY@tc##1{\textcolor[rgb]{0.10,0.09,0.49}{##1}}}
\expandafter\def\csname PY@tok@vi\endcsname{\def\PY@tc##1{\textcolor[rgb]{0.10,0.09,0.49}{##1}}}
\expandafter\def\csname PY@tok@vm\endcsname{\def\PY@tc##1{\textcolor[rgb]{0.10,0.09,0.49}{##1}}}
\expandafter\def\csname PY@tok@mh\endcsname{\def\PY@tc##1{\textcolor[rgb]{0.40,0.40,0.40}{##1}}}
\expandafter\def\csname PY@tok@cs\endcsname{\let\PY@it=\textit\def\PY@tc##1{\textcolor[rgb]{0.25,0.50,0.50}{##1}}}
\expandafter\def\csname PY@tok@ge\endcsname{\let\PY@it=\textit}
\expandafter\def\csname PY@tok@vc\endcsname{\def\PY@tc##1{\textcolor[rgb]{0.10,0.09,0.49}{##1}}}
\expandafter\def\csname PY@tok@il\endcsname{\def\PY@tc##1{\textcolor[rgb]{0.40,0.40,0.40}{##1}}}
\expandafter\def\csname PY@tok@go\endcsname{\def\PY@tc##1{\textcolor[rgb]{0.53,0.53,0.53}{##1}}}
\expandafter\def\csname PY@tok@cp\endcsname{\def\PY@tc##1{\textcolor[rgb]{0.74,0.48,0.00}{##1}}}
\expandafter\def\csname PY@tok@gi\endcsname{\def\PY@tc##1{\textcolor[rgb]{0.00,0.63,0.00}{##1}}}
\expandafter\def\csname PY@tok@gh\endcsname{\let\PY@bf=\textbf\def\PY@tc##1{\textcolor[rgb]{0.00,0.00,0.50}{##1}}}
\expandafter\def\csname PY@tok@ni\endcsname{\let\PY@bf=\textbf\def\PY@tc##1{\textcolor[rgb]{0.60,0.60,0.60}{##1}}}
\expandafter\def\csname PY@tok@nl\endcsname{\def\PY@tc##1{\textcolor[rgb]{0.63,0.63,0.00}{##1}}}
\expandafter\def\csname PY@tok@nn\endcsname{\let\PY@bf=\textbf\def\PY@tc##1{\textcolor[rgb]{0.00,0.00,1.00}{##1}}}
\expandafter\def\csname PY@tok@no\endcsname{\def\PY@tc##1{\textcolor[rgb]{0.53,0.00,0.00}{##1}}}
\expandafter\def\csname PY@tok@na\endcsname{\def\PY@tc##1{\textcolor[rgb]{0.49,0.56,0.16}{##1}}}
\expandafter\def\csname PY@tok@nb\endcsname{\def\PY@tc##1{\textcolor[rgb]{0.00,0.50,0.00}{##1}}}
\expandafter\def\csname PY@tok@nc\endcsname{\let\PY@bf=\textbf\def\PY@tc##1{\textcolor[rgb]{0.00,0.00,1.00}{##1}}}
\expandafter\def\csname PY@tok@nd\endcsname{\def\PY@tc##1{\textcolor[rgb]{0.67,0.13,1.00}{##1}}}
\expandafter\def\csname PY@tok@ne\endcsname{\let\PY@bf=\textbf\def\PY@tc##1{\textcolor[rgb]{0.82,0.25,0.23}{##1}}}
\expandafter\def\csname PY@tok@nf\endcsname{\def\PY@tc##1{\textcolor[rgb]{0.00,0.00,1.00}{##1}}}
\expandafter\def\csname PY@tok@si\endcsname{\let\PY@bf=\textbf\def\PY@tc##1{\textcolor[rgb]{0.73,0.40,0.53}{##1}}}
\expandafter\def\csname PY@tok@s2\endcsname{\def\PY@tc##1{\textcolor[rgb]{0.73,0.13,0.13}{##1}}}
\expandafter\def\csname PY@tok@nt\endcsname{\let\PY@bf=\textbf\def\PY@tc##1{\textcolor[rgb]{0.00,0.50,0.00}{##1}}}
\expandafter\def\csname PY@tok@nv\endcsname{\def\PY@tc##1{\textcolor[rgb]{0.10,0.09,0.49}{##1}}}
\expandafter\def\csname PY@tok@s1\endcsname{\def\PY@tc##1{\textcolor[rgb]{0.73,0.13,0.13}{##1}}}
\expandafter\def\csname PY@tok@dl\endcsname{\def\PY@tc##1{\textcolor[rgb]{0.73,0.13,0.13}{##1}}}
\expandafter\def\csname PY@tok@ch\endcsname{\let\PY@it=\textit\def\PY@tc##1{\textcolor[rgb]{0.25,0.50,0.50}{##1}}}
\expandafter\def\csname PY@tok@m\endcsname{\def\PY@tc##1{\textcolor[rgb]{0.40,0.40,0.40}{##1}}}
\expandafter\def\csname PY@tok@gp\endcsname{\let\PY@bf=\textbf\def\PY@tc##1{\textcolor[rgb]{0.00,0.00,0.50}{##1}}}
\expandafter\def\csname PY@tok@sh\endcsname{\def\PY@tc##1{\textcolor[rgb]{0.73,0.13,0.13}{##1}}}
\expandafter\def\csname PY@tok@ow\endcsname{\let\PY@bf=\textbf\def\PY@tc##1{\textcolor[rgb]{0.67,0.13,1.00}{##1}}}
\expandafter\def\csname PY@tok@sx\endcsname{\def\PY@tc##1{\textcolor[rgb]{0.00,0.50,0.00}{##1}}}
\expandafter\def\csname PY@tok@bp\endcsname{\def\PY@tc##1{\textcolor[rgb]{0.00,0.50,0.00}{##1}}}
\expandafter\def\csname PY@tok@c1\endcsname{\let\PY@it=\textit\def\PY@tc##1{\textcolor[rgb]{0.25,0.50,0.50}{##1}}}
\expandafter\def\csname PY@tok@fm\endcsname{\def\PY@tc##1{\textcolor[rgb]{0.00,0.00,1.00}{##1}}}
\expandafter\def\csname PY@tok@o\endcsname{\def\PY@tc##1{\textcolor[rgb]{0.40,0.40,0.40}{##1}}}
\expandafter\def\csname PY@tok@kc\endcsname{\let\PY@bf=\textbf\def\PY@tc##1{\textcolor[rgb]{0.00,0.50,0.00}{##1}}}
\expandafter\def\csname PY@tok@c\endcsname{\let\PY@it=\textit\def\PY@tc##1{\textcolor[rgb]{0.25,0.50,0.50}{##1}}}
\expandafter\def\csname PY@tok@mf\endcsname{\def\PY@tc##1{\textcolor[rgb]{0.40,0.40,0.40}{##1}}}
\expandafter\def\csname PY@tok@err\endcsname{\def\PY@bc##1{\setlength{\fboxsep}{0pt}\fcolorbox[rgb]{1.00,0.00,0.00}{1,1,1}{\strut ##1}}}
\expandafter\def\csname PY@tok@mb\endcsname{\def\PY@tc##1{\textcolor[rgb]{0.40,0.40,0.40}{##1}}}
\expandafter\def\csname PY@tok@ss\endcsname{\def\PY@tc##1{\textcolor[rgb]{0.10,0.09,0.49}{##1}}}
\expandafter\def\csname PY@tok@sr\endcsname{\def\PY@tc##1{\textcolor[rgb]{0.73,0.40,0.53}{##1}}}
\expandafter\def\csname PY@tok@mo\endcsname{\def\PY@tc##1{\textcolor[rgb]{0.40,0.40,0.40}{##1}}}
\expandafter\def\csname PY@tok@kd\endcsname{\let\PY@bf=\textbf\def\PY@tc##1{\textcolor[rgb]{0.00,0.50,0.00}{##1}}}
\expandafter\def\csname PY@tok@mi\endcsname{\def\PY@tc##1{\textcolor[rgb]{0.40,0.40,0.40}{##1}}}
\expandafter\def\csname PY@tok@kn\endcsname{\let\PY@bf=\textbf\def\PY@tc##1{\textcolor[rgb]{0.00,0.50,0.00}{##1}}}
\expandafter\def\csname PY@tok@cpf\endcsname{\let\PY@it=\textit\def\PY@tc##1{\textcolor[rgb]{0.25,0.50,0.50}{##1}}}
\expandafter\def\csname PY@tok@kr\endcsname{\let\PY@bf=\textbf\def\PY@tc##1{\textcolor[rgb]{0.00,0.50,0.00}{##1}}}
\expandafter\def\csname PY@tok@s\endcsname{\def\PY@tc##1{\textcolor[rgb]{0.73,0.13,0.13}{##1}}}
\expandafter\def\csname PY@tok@kp\endcsname{\def\PY@tc##1{\textcolor[rgb]{0.00,0.50,0.00}{##1}}}
\expandafter\def\csname PY@tok@w\endcsname{\def\PY@tc##1{\textcolor[rgb]{0.73,0.73,0.73}{##1}}}
\expandafter\def\csname PY@tok@kt\endcsname{\def\PY@tc##1{\textcolor[rgb]{0.69,0.00,0.25}{##1}}}
\expandafter\def\csname PY@tok@sc\endcsname{\def\PY@tc##1{\textcolor[rgb]{0.73,0.13,0.13}{##1}}}
\expandafter\def\csname PY@tok@sb\endcsname{\def\PY@tc##1{\textcolor[rgb]{0.73,0.13,0.13}{##1}}}
\expandafter\def\csname PY@tok@sa\endcsname{\def\PY@tc##1{\textcolor[rgb]{0.73,0.13,0.13}{##1}}}
\expandafter\def\csname PY@tok@k\endcsname{\let\PY@bf=\textbf\def\PY@tc##1{\textcolor[rgb]{0.00,0.50,0.00}{##1}}}
\expandafter\def\csname PY@tok@se\endcsname{\let\PY@bf=\textbf\def\PY@tc##1{\textcolor[rgb]{0.73,0.40,0.13}{##1}}}
\expandafter\def\csname PY@tok@sd\endcsname{\let\PY@it=\textit\def\PY@tc##1{\textcolor[rgb]{0.73,0.13,0.13}{##1}}}

\def\PYZbs{\char`\\}
\def\PYZus{\char`\_}
\def\PYZob{\char`\{}
\def\PYZcb{\char`\}}
\def\PYZca{\char`\^}
\def\PYZam{\char`\&}
\def\PYZlt{\char`\<}
\def\PYZgt{\char`\>}
\def\PYZsh{\char`\#}
\def\PYZpc{\char`\%}
\def\PYZdl{\char`\$}
\def\PYZhy{\char`\-}
\def\PYZsq{\char`\'}
\def\PYZdq{\char`\"}
\def\PYZti{\char`\~}
% for compatibility with earlier versions
\def\PYZat{@}
\def\PYZlb{[}
\def\PYZrb{]}
\makeatother


    % Exact colors from NB
    \definecolor{incolor}{rgb}{0.0, 0.0, 0.5}
    \definecolor{outcolor}{rgb}{0.545, 0.0, 0.0}



    
    % Prevent overflowing lines due to hard-to-break entities
    \sloppy 
    % Setup hyperref package
    \hypersetup{
      breaklinks=true,  % so long urls are correctly broken across lines
      colorlinks=true,
      urlcolor=urlcolor,
      linkcolor=linkcolor,
      citecolor=citecolor,
      }
    % Slightly bigger margins than the latex defaults
    
    \geometry{verbose,tmargin=1in,bmargin=1in,lmargin=1in,rmargin=1in}
    
    

    \begin{document}
    
    
    \maketitle
    
    

    
    \section{Redes Complejas 2018: Trabajo Computacional
1}\label{redes-complejas-2018-trabajo-computacional-1}

\emph{Por Tomás Ferreira Chase, Matías Leizerovich y Gabriel Goren}

    Este trabajo fue realizado en Python con \texttt{networkx} versión 2.x.
Se empleó la librería \texttt{rpy2} para llamar a la función
\texttt{fit\_power\_law} de la librería \texttt{igraph} de R en el
Ejercicio 3.

    \subsection{Ejercicio 1: Proteínas}\label{ejercicio-1-proteuxednas}

    En este ejercicio, nos familiarizamos con el manejo más básico de redes
empleando la librería \texttt{networkx} así como con algunas cantidades
que podemos medir sobre las mismas.

Primero importamos las librerías que precisamos.

    \begin{Verbatim}[commandchars=\\\{\}]
{\color{incolor}In [{\color{incolor}6}]:} \PY{k+kn}{from} \PY{n+nn}{lectura} \PY{k+kn}{import} \PY{n}{ldata}
        \PY{k+kn}{import} \PY{n+nn}{networkx} \PY{k+kn}{as} \PY{n+nn}{nx}
        \PY{k+kn}{import} \PY{n+nn}{numpy} \PY{k+kn}{as} \PY{n+nn}{np}
        \PY{k+kn}{from} \PY{n+nn}{matplotlib} \PY{k+kn}{import} \PY{n}{pyplot} \PY{k}{as} \PY{n}{plt}
        \PY{k+kn}{import} \PY{n+nn}{pandas} \PY{k+kn}{as} \PY{n+nn}{pd}
        \PY{k+kn}{from} \PY{n+nn}{\PYZus{}\PYZus{}future\PYZus{}\PYZus{}} \PY{k+kn}{import} \PY{n}{division} \PY{c+c1}{\PYZsh{} Compatibilidad con python2}
\end{Verbatim}


    Luego definimos algunas funciones que vamos a utilizar.

    \begin{Verbatim}[commandchars=\\\{\}]
{\color{incolor}In [{\color{incolor}21}]:} \PY{k}{def} \PY{n+nf}{es\PYZus{}dirigido}\PY{p}{(}\PY{n}{data}\PY{p}{)}\PY{p}{:}
             \PY{l+s+sd}{\PYZdq{}\PYZdq{}\PYZdq{}Función que calcula un criterio para determinar, en ausencia}
         \PY{l+s+sd}{    de otra información, si un cierto grafo es o no es dirigido.}
         \PY{l+s+sd}{    }
         \PY{l+s+sd}{    Recibe la información de un grafo dada en forma de lista de}
         \PY{l+s+sd}{    enlaces (tuplas).}
         \PY{l+s+sd}{    }
         \PY{l+s+sd}{    El criterio implementado es básicamente el siguiente: si un mismo}
         \PY{l+s+sd}{    par de nodos aparece dos veces en la lista de enlaces, una en}
         \PY{l+s+sd}{    cada orden posible, entonces la red debe ser dirigida, puesto que}
         \PY{l+s+sd}{    de lo contrario se nos estaría proveyendo información redundante.}
         \PY{l+s+sd}{    }
         \PY{l+s+sd}{    Devuelve el número de veces que un par de nodos aparece repetido}
         \PY{l+s+sd}{    de esta manera. Si el resultado es 0, entonces es no dirigido; si}
         \PY{l+s+sd}{    es distinto de cero, es dirigido (según este criterio).}
         \PY{l+s+sd}{    }
         \PY{l+s+sd}{    }
         \PY{l+s+sd}{    \PYZdq{}\PYZdq{}\PYZdq{}}
             \PY{n}{n} \PY{o}{=} \PY{l+m+mi}{0}
             \PY{k}{for} \PY{p}{(}\PY{n}{x}\PY{p}{,} \PY{n}{y}\PY{p}{)} \PY{o+ow}{in} \PY{n}{data}\PY{p}{:}
                 \PY{k}{for} \PY{p}{(}\PY{n}{a}\PY{p}{,} \PY{n}{b}\PY{p}{)} \PY{o+ow}{in} \PY{n}{data}\PY{p}{:}
                     \PY{k}{if} \PY{n}{a} \PY{o}{==} \PY{n}{y} \PY{o+ow}{and} \PY{n}{b} \PY{o}{==} \PY{n}{x}\PY{p}{:}
                         \PY{n}{n} \PY{o}{+}\PY{o}{=} \PY{l+m+mi}{1}
             \PY{k}{return} \PY{n}{n}\PY{o}{/}\PY{l+m+mi}{2}
         
         \PY{k}{def} \PY{n+nf}{k\PYZus{}medio}\PY{p}{(}\PY{n}{G}\PY{p}{)}\PY{p}{:}
             \PY{l+s+sd}{\PYZdq{}\PYZdq{}\PYZdq{}Función que calcula el grado medio de un grafo, ya sea}
         \PY{l+s+sd}{    dirigido o no dirigido. \PYZdq{}\PYZdq{}\PYZdq{}}
             \PY{n}{N} \PY{o}{=} \PY{n}{G}\PY{o}{.}\PY{n}{order}\PY{p}{(}\PY{p}{)}
             \PY{k}{if} \PY{n+nb}{isinstance}\PY{p}{(}\PY{n}{G}\PY{p}{,} \PY{n}{nx}\PY{o}{.}\PY{n}{DiGraph}\PY{p}{)}\PY{p}{:}
                 \PY{n}{kin\PYZus{}med} \PY{o}{=} \PY{n+nb}{sum}\PY{p}{(}\PY{n}{k} \PY{k}{for} \PY{p}{(}\PY{n}{nodo}\PY{p}{,} \PY{n}{k}\PY{p}{)} \PY{o+ow}{in} \PY{n}{G}\PY{o}{.}\PY{n}{in\PYZus{}degree}\PY{p}{)} \PY{o}{/} \PY{n}{N}
                 \PY{n}{kout\PYZus{}med} \PY{o}{=} \PY{n+nb}{sum}\PY{p}{(}\PY{n}{k} \PY{k}{for} \PY{p}{(}\PY{n}{nodo}\PY{p}{,} \PY{n}{k}\PY{p}{)} \PY{o+ow}{in} \PY{n}{G}\PY{o}{.}\PY{n}{out\PYZus{}degree}\PY{p}{)} \PY{o}{/} \PY{n}{N}
             \PY{k}{else}\PY{p}{:}
                 \PY{n}{kin\PYZus{}med}\PY{p}{,} \PY{n}{kout\PYZus{}med} \PY{o}{=} \PY{l+m+mi}{0}\PY{p}{,} \PY{l+m+mi}{0}
             \PY{n}{k\PYZus{}med} \PY{o}{=} \PY{n+nb}{sum}\PY{p}{(}\PY{n}{k} \PY{k}{for} \PY{p}{(}\PY{n}{nodo}\PY{p}{,} \PY{n}{k}\PY{p}{)} \PY{o+ow}{in} \PY{n}{G}\PY{o}{.}\PY{n}{degree}\PY{p}{)} \PY{o}{/} \PY{n}{N}
             \PY{k}{return} \PY{n}{kin\PYZus{}med}\PY{p}{,} \PY{n}{kout\PYZus{}med}\PY{p}{,} \PY{n}{k\PYZus{}med}
         
         \PY{k}{def} \PY{n+nf}{k\PYZus{}extremos}\PY{p}{(}\PY{n}{G}\PY{p}{)}\PY{p}{:}
             \PY{n}{k\PYZus{}min} \PY{o}{=} \PY{n+nb}{min}\PY{p}{(}\PY{n}{k} \PY{k}{for} \PY{p}{(}\PY{n}{nodo}\PY{p}{,} \PY{n}{k}\PY{p}{)} \PY{o+ow}{in} \PY{n}{G}\PY{o}{.}\PY{n}{degree}\PY{p}{)}
             \PY{n}{k\PYZus{}max} \PY{o}{=} \PY{n+nb}{max}\PY{p}{(}\PY{n}{k} \PY{k}{for} \PY{p}{(}\PY{n}{nodo}\PY{p}{,} \PY{n}{k}\PY{p}{)} \PY{o+ow}{in} \PY{n}{G}\PY{o}{.}\PY{n}{degree}\PY{p}{)}
             \PY{k}{return} \PY{n}{k\PYZus{}min}\PY{p}{,} \PY{n}{k\PYZus{}max}
         
         \PY{k}{def} \PY{n+nf}{clustering\PYZus{}medio}\PY{p}{(}\PY{n}{G}\PY{p}{)}\PY{p}{:}
             \PY{k}{return} \PY{n}{np}\PY{o}{.}\PY{n}{average}\PY{p}{(}\PY{n+nb}{list}\PY{p}{(}\PY{n+nb}{dict}\PY{p}{(}\PY{n}{nx}\PY{o}{.}\PY{n}{clustering}\PY{p}{(}\PY{n}{G}\PY{p}{)}\PY{p}{)}\PY{o}{.}\PY{n}{values}\PY{p}{(}\PY{p}{)}\PY{p}{)}\PY{p}{)}
\end{Verbatim}


    Importamos las redes.

    \begin{Verbatim}[commandchars=\\\{\}]
{\color{incolor}In [{\color{incolor}22}]:} \PY{n}{y2h} \PY{o}{=} \PY{n}{ldata}\PY{p}{(}\PY{l+s+s1}{\PYZsq{}}\PY{l+s+s1}{tc01\PYZus{}data/yeast\PYZus{}Y2H.txt}\PY{l+s+s1}{\PYZsq{}}\PY{p}{)}
         \PY{n}{apms} \PY{o}{=} \PY{n}{ldata}\PY{p}{(}\PY{l+s+s1}{\PYZsq{}}\PY{l+s+s1}{tc01\PYZus{}data/yeast\PYZus{}AP\PYZhy{}MS.txt}\PY{l+s+s1}{\PYZsq{}}\PY{p}{)}
         \PY{n}{lit} \PY{o}{=} \PY{n}{ldata}\PY{p}{(}\PY{l+s+s1}{\PYZsq{}}\PY{l+s+s1}{tc01\PYZus{}data/yeast\PYZus{}LIT.txt}\PY{l+s+s1}{\PYZsq{}}\PY{p}{)}
\end{Verbatim}


    \begin{Verbatim}[commandchars=\\\{\}]
{\color{incolor}In [{\color{incolor}23}]:} \PY{n}{g\PYZus{}y2h} \PY{o}{=} \PY{n}{nx}\PY{o}{.}\PY{n}{Graph}\PY{p}{(}\PY{p}{)}
         \PY{n}{g\PYZus{}y2h}\PY{o}{.}\PY{n}{add\PYZus{}edges\PYZus{}from}\PY{p}{(}\PY{n}{y2h}\PY{p}{)}
         
         \PY{n}{g\PYZus{}apms} \PY{o}{=} \PY{n}{nx}\PY{o}{.}\PY{n}{Graph}\PY{p}{(}\PY{p}{)}
         \PY{n}{g\PYZus{}apms}\PY{o}{.}\PY{n}{add\PYZus{}edges\PYZus{}from}\PY{p}{(}\PY{n}{apms}\PY{p}{)}
         
         \PY{n}{g\PYZus{}lit} \PY{o}{=} \PY{n}{nx}\PY{o}{.}\PY{n}{Graph}\PY{p}{(}\PY{p}{)}
         \PY{n}{g\PYZus{}lit}\PY{o}{.}\PY{n}{add\PYZus{}edges\PYZus{}from}\PY{p}{(}\PY{n}{lit}\PY{p}{)}
\end{Verbatim}


    \subsubsection{Punto a}\label{punto-a}

    En primer lugar visualizamos las redes que hemos importado. Según nos
indica la consigna, se trata de redes de interacción de proteínas
relevadas para levadura, por lo cual podemos decir que las tres redes
representan un mismo sistema físico.

    \begin{Verbatim}[commandchars=\\\{\}]
{\color{incolor}In [{\color{incolor}29}]:} \PY{n}{fig}\PY{p}{,} \PY{p}{(}\PY{n}{ax1}\PY{p}{,} \PY{n}{ax2}\PY{p}{,} \PY{n}{ax3}\PY{p}{)} \PY{o}{=} \PY{n}{plt}\PY{o}{.}\PY{n}{subplots}\PY{p}{(}\PY{l+m+mi}{1}\PY{p}{,} \PY{l+m+mi}{3}\PY{p}{,} \PY{n}{figsize}\PY{o}{=}\PY{p}{(}\PY{l+m+mi}{16}\PY{p}{,} \PY{l+m+mi}{6}\PY{p}{)}\PY{p}{)}
         
         \PY{n}{ax1}\PY{o}{.}\PY{n}{set\PYZus{}title}\PY{p}{(}\PY{l+s+s1}{\PYZsq{}}\PY{l+s+s1}{Y2h}\PY{l+s+s1}{\PYZsq{}}\PY{p}{)}
         \PY{n}{nx}\PY{o}{.}\PY{n}{draw}\PY{p}{(}\PY{n}{g\PYZus{}y2h}\PY{p}{,} \PY{n}{node\PYZus{}size} \PY{o}{=} \PY{l+m+mi}{10}\PY{p}{,} \PY{n}{ax}\PY{o}{=}\PY{n}{ax1}\PY{p}{)}
         
         \PY{n}{ax2}\PY{o}{.}\PY{n}{set\PYZus{}title}\PY{p}{(}\PY{l+s+s1}{\PYZsq{}}\PY{l+s+s1}{APMS}\PY{l+s+s1}{\PYZsq{}}\PY{p}{)}
         \PY{n}{nx}\PY{o}{.}\PY{n}{draw}\PY{p}{(}\PY{n}{g\PYZus{}apms}\PY{p}{,} \PY{n}{node\PYZus{}size} \PY{o}{=} \PY{l+m+mi}{10}\PY{p}{,} \PY{n}{ax}\PY{o}{=}\PY{n}{ax2}\PY{p}{)}
         
         \PY{n}{ax3}\PY{o}{.}\PY{n}{set\PYZus{}title}\PY{p}{(}\PY{l+s+s1}{\PYZsq{}}\PY{l+s+s1}{Lit}\PY{l+s+s1}{\PYZsq{}}\PY{p}{)}
         \PY{n}{nx}\PY{o}{.}\PY{n}{draw}\PY{p}{(}\PY{n}{g\PYZus{}lit}\PY{p}{,} \PY{n}{node\PYZus{}size} \PY{o}{=} \PY{l+m+mi}{10}\PY{p}{,} \PY{n}{ax}\PY{o}{=}\PY{n}{ax3}\PY{p}{)}
         
         \PY{n}{plt}\PY{o}{.}\PY{n}{show}\PY{p}{(}\PY{p}{)}
\end{Verbatim}


    \begin{center}
    \adjustimage{max size={0.9\linewidth}{0.9\paperheight}}{output_12_0.png}
    \end{center}
    { \hspace*{\fill} \\}
    
    Podemos ver que se trata de redes grandes, que presentan una componente
gigante. Graficando solo estas últimas:

    \begin{Verbatim}[commandchars=\\\{\}]
{\color{incolor}In [{\color{incolor}28}]:} \PY{c+c1}{\PYZsh{} Construimos los subgrafos correspondientes a las componentes gigantes}
         \PY{n}{gc\PYZus{}y2h} \PY{o}{=} \PY{n+nb}{max}\PY{p}{(}\PY{n}{nx}\PY{o}{.}\PY{n}{connected\PYZus{}component\PYZus{}subgraphs}\PY{p}{(}\PY{n}{g\PYZus{}y2h}\PY{p}{)}\PY{p}{,} \PY{n}{key}\PY{o}{=}\PY{n+nb}{len}\PY{p}{)}
         \PY{n}{gc\PYZus{}apms} \PY{o}{=} \PY{n+nb}{max}\PY{p}{(}\PY{n}{nx}\PY{o}{.}\PY{n}{connected\PYZus{}component\PYZus{}subgraphs}\PY{p}{(}\PY{n}{g\PYZus{}apms}\PY{p}{)}\PY{p}{,} \PY{n}{key}\PY{o}{=}\PY{n+nb}{len}\PY{p}{)}
         \PY{n}{gc\PYZus{}lit} \PY{o}{=} \PY{n+nb}{max}\PY{p}{(}\PY{n}{nx}\PY{o}{.}\PY{n}{connected\PYZus{}component\PYZus{}subgraphs}\PY{p}{(}\PY{n}{g\PYZus{}lit}\PY{p}{)}\PY{p}{,} \PY{n}{key}\PY{o}{=}\PY{n+nb}{len}\PY{p}{)}
\end{Verbatim}


    \begin{Verbatim}[commandchars=\\\{\}]
{\color{incolor}In [{\color{incolor}30}]:} \PY{c+c1}{\PYZsh{} Los graficamos}
         \PY{n}{fig}\PY{p}{,} \PY{p}{(}\PY{n}{ax1}\PY{p}{,} \PY{n}{ax2}\PY{p}{,} \PY{n}{ax3}\PY{p}{)} \PY{o}{=} \PY{n}{plt}\PY{o}{.}\PY{n}{subplots}\PY{p}{(}\PY{l+m+mi}{1}\PY{p}{,} \PY{l+m+mi}{3}\PY{p}{,} \PY{n}{figsize}\PY{o}{=}\PY{p}{(}\PY{l+m+mi}{16}\PY{p}{,} \PY{l+m+mi}{6}\PY{p}{)}\PY{p}{)}
         
         \PY{n}{ax1}\PY{o}{.}\PY{n}{set\PYZus{}title}\PY{p}{(}\PY{l+s+s1}{\PYZsq{}}\PY{l+s+s1}{Y2h}\PY{l+s+s1}{\PYZsq{}}\PY{p}{)}
         \PY{n}{nx}\PY{o}{.}\PY{n}{draw}\PY{p}{(}\PY{n}{gc\PYZus{}y2h}\PY{p}{,} \PY{n}{node\PYZus{}size} \PY{o}{=} \PY{l+m+mi}{10}\PY{p}{,} \PY{n}{ax}\PY{o}{=}\PY{n}{ax1}\PY{p}{)}
         
         \PY{n}{ax2}\PY{o}{.}\PY{n}{set\PYZus{}title}\PY{p}{(}\PY{l+s+s1}{\PYZsq{}}\PY{l+s+s1}{APMS}\PY{l+s+s1}{\PYZsq{}}\PY{p}{)}
         \PY{n}{nx}\PY{o}{.}\PY{n}{draw}\PY{p}{(}\PY{n}{gc\PYZus{}apms}\PY{p}{,} \PY{n}{node\PYZus{}size} \PY{o}{=} \PY{l+m+mi}{10}\PY{p}{,} \PY{n}{ax}\PY{o}{=}\PY{n}{ax2}\PY{p}{)}
         
         \PY{n}{ax3}\PY{o}{.}\PY{n}{set\PYZus{}title}\PY{p}{(}\PY{l+s+s1}{\PYZsq{}}\PY{l+s+s1}{Lit}\PY{l+s+s1}{\PYZsq{}}\PY{p}{)}
         \PY{n}{nx}\PY{o}{.}\PY{n}{draw}\PY{p}{(}\PY{n}{gc\PYZus{}lit}\PY{p}{,} \PY{n}{node\PYZus{}size} \PY{o}{=} \PY{l+m+mi}{10}\PY{p}{,} \PY{n}{ax}\PY{o}{=}\PY{n}{ax3}\PY{p}{)}
         
         \PY{n}{plt}\PY{o}{.}\PY{n}{show}\PY{p}{(}\PY{p}{)}
\end{Verbatim}


    \begin{center}
    \adjustimage{max size={0.9\linewidth}{0.9\paperheight}}{output_15_0.png}
    \end{center}
    { \hspace*{\fill} \\}
    
    Vemos que el algoritmo de visualización por defecto de \texttt{networkx}
(posicionamiento de nodos por resortes) grafica las 3 redes de maneras
bastante diferentes, con lo cual podemos suponer que hay diferencias
significativas en los datos aún si representan al mismo sistema físico.

    \subsubsection{Punto b}\label{punto-b}

    \paragraph{Sobre si las redes son o no son
dirigidas}\label{sobre-si-las-redes-son-o-no-son-dirigidas}

    En principio, al comenzar a trabajar con una red es esencial que se nos
informe si la red es o no dirigida, dado que esta información no está
contenida en la lista de enlaces, y cambia muchísimo la interpretación
física de la red.

Ante la ausencia de un metadato claro sobre si los enlaces importados
deben ser considerados dirigidos o no dirigidos, se implementó la
función \texttt{es\_dirigido} que permite evaluar un criterio posible
para determinar, para cada red, de qué tipo de enlaces se trata. El
criterio es simple, si bien falible: si un mismo par de nodos aparece
dos veces en la lista de enlaces, una en cada orden posible, entonces la
red debe ser dirigida, puesto que de lo contrario se nos estaría
proveyendo información redundante. Puede fallar debido a que también es
posible que los datos simplemente estén presentados de manera poco
prolija.

Aplicando la función, obtenemos lo siguiente.

    \begin{Verbatim}[commandchars=\\\{\}]
{\color{incolor}In [{\color{incolor}41}]:} \PY{k}{for} \PY{n}{nombre}\PY{p}{,} \PY{n}{datos} \PY{o+ow}{in} \PY{p}{[}\PY{p}{(}\PY{l+s+s1}{\PYZsq{}}\PY{l+s+s1}{Y2H}\PY{l+s+s1}{\PYZsq{}}\PY{p}{,} \PY{n}{y2h}\PY{p}{)}\PY{p}{,} \PY{p}{(}\PY{l+s+s1}{\PYZsq{}}\PY{l+s+s1}{AP\PYZhy{}MS}\PY{l+s+s1}{\PYZsq{}}\PY{p}{,} \PY{n}{apms}\PY{p}{)}\PY{p}{,} \PY{p}{(}\PY{l+s+s1}{\PYZsq{}}\PY{l+s+s1}{Lit}\PY{l+s+s1}{\PYZsq{}}\PY{p}{,} \PY{n}{lit}\PY{p}{)}\PY{p}{]}\PY{p}{:}
             \PY{k}{print}\PY{p}{(}\PY{l+s+s1}{\PYZsq{}}\PY{l+s+s1}{Los datos de la red \PYZob{}\PYZcb{} tienen \PYZob{}\PYZcb{} pares de nodos repetidos.}\PY{l+s+s1}{\PYZsq{}}\PY{o}{.}\PY{n}{format}\PY{p}{(}\PY{n}{nombre}\PY{p}{,} \PY{n}{es\PYZus{}dirigido}\PY{p}{(}\PY{n}{datos}\PY{p}{)}\PY{p}{)}\PY{p}{)}
\end{Verbatim}


    \begin{Verbatim}[commandchars=\\\{\}]
Los datos de la red Y2H tienen 112.5 pares de nodos repetidos.
Los datos de la red AP-MS tienen 0.0 pares de nodos repetidos.
Los datos de la red Lit tienen 40.5 pares de nodos repetidos.

    \end{Verbatim}

    Este criterio parecería indicar que las redes Y2H y Lit son dirigidas.
Sin embargo, la cantidad de pares de nodos que aparecen repetidos es muy
chica respecto del tamaño total de la red (ver la tabla más abajo), por
lo cual también es factible que se trate de datos mal presentados.

En última instancia, se recurrió al significado biológico de las redes
para determinar su naturaleza: debido a que estamos hablando de
interacciones entre proteínas, relacionadas con el posible acoplamiento
mecánico entre las mismas, no tiene mucho sentido considerar que dichas
interacciones puedan ser dirigidas.

De esta manera, se trabajó con las redes considerándolas no dirigidas,
salvo que se diga lo contrario.

    Presentamos entonces una tabla con todos los datos pedidos, y luego
procedemos a mostrar cómo calculamos cada uno de ellos.

    \begin{Verbatim}[commandchars=\\\{\}]
{\color{incolor}In [{\color{incolor}27}]:} \PY{n}{data} \PY{o}{=} \PY{n}{pd}\PY{o}{.}\PY{n}{DataFrame}\PY{p}{(}\PY{p}{\PYZob{}}\PY{l+s+s2}{\PYZdq{}}\PY{l+s+s2}{Nombre de la red}\PY{l+s+s2}{\PYZdq{}}\PY{p}{:} \PY{p}{[}\PY{l+s+s1}{\PYZsq{}}\PY{l+s+s1}{Y2H}\PY{l+s+s1}{\PYZsq{}}\PY{p}{,}\PY{l+s+s1}{\PYZsq{}}\PY{l+s+s1}{AP\PYZhy{}MS}\PY{l+s+s1}{\PYZsq{}}\PY{p}{,}\PY{l+s+s1}{\PYZsq{}}\PY{l+s+s1}{Lit}\PY{l+s+s1}{\PYZsq{}}\PY{p}{]}\PY{p}{,}
                              \PY{l+s+s2}{\PYZdq{}}\PY{l+s+s2}{N}\PY{l+s+s2}{\PYZdq{}}\PY{p}{:}\PY{p}{[}\PY{l+m+mi}{2018}\PY{p}{,}\PY{l+m+mi}{1622}\PY{p}{,}\PY{l+m+mi}{1536}\PY{p}{]}\PY{p}{,}
                              \PY{l+s+s2}{\PYZdq{}}\PY{l+s+s2}{L}\PY{l+s+s2}{\PYZdq{}}\PY{p}{:}\PY{p}{[}\PY{l+m+mi}{2930}\PY{p}{,}\PY{l+m+mi}{9070}\PY{p}{,}\PY{l+m+mi}{2925}\PY{p}{]}\PY{p}{,}
                              \PY{l+s+s2}{\PYZdq{}}\PY{l+s+s2}{Dirigida?}\PY{l+s+s2}{\PYZdq{}}\PY{p}{:}\PY{p}{[}\PY{l+s+s1}{\PYZsq{}}\PY{l+s+s1}{No}\PY{l+s+s1}{\PYZsq{}}\PY{p}{,}\PY{l+s+s1}{\PYZsq{}}\PY{l+s+s1}{No}\PY{l+s+s1}{\PYZsq{}}\PY{p}{,}\PY{l+s+s1}{\PYZsq{}}\PY{l+s+s1}{No}\PY{l+s+s1}{\PYZsq{}}\PY{p}{]}\PY{p}{,}
                              \PY{l+s+s2}{\PYZdq{}}\PY{l+s+s2}{\PYZdl{}\PYZdl{}}\PY{l+s+s2}{\PYZbs{}}\PY{l+s+s2}{overline\PYZob{}K\PYZcb{}\PYZdl{}\PYZdl{}}\PY{l+s+s2}{\PYZdq{}}\PY{p}{:}\PY{p}{[}\PY{l+m+mf}{2.9038652130822595}\PY{p}{,}\PY{l+m+mf}{11.183723797780518}\PY{p}{,}\PY{l+m+mf}{3.80859375}\PY{p}{]}\PY{p}{,}
                              \PY{l+s+s2}{\PYZdq{}}\PY{l+s+s2}{\PYZdl{}\PYZdl{}}\PY{l+s+s2}{\PYZbs{}}\PY{l+s+s2}{overline\PYZob{}K\PYZus{}\PYZob{}in\PYZcb{}\PYZcb{}\PYZdl{}\PYZdl{}}\PY{l+s+s2}{\PYZdq{}}\PY{p}{:}\PY{p}{[}\PY{l+m+mf}{1.4519326065411298}\PY{p}{,}\PY{l+s+s1}{\PYZsq{}}\PY{l+s+s1}{\PYZhy{}}\PY{l+s+s1}{\PYZsq{}}\PY{p}{,}\PY{l+m+mf}{1.904296875}\PY{p}{]}\PY{p}{,}
                              \PY{l+s+s2}{\PYZdq{}}\PY{l+s+s2}{\PYZdl{}\PYZdl{}}\PY{l+s+s2}{\PYZbs{}}\PY{l+s+s2}{overline\PYZob{}K\PYZus{}\PYZob{}out\PYZcb{}\PYZcb{}\PYZdl{}\PYZdl{}}\PY{l+s+s2}{\PYZdq{}}\PY{p}{:}\PY{p}{[}\PY{l+m+mf}{1.4519326065411298}\PY{p}{,}\PY{l+s+s1}{\PYZsq{}}\PY{l+s+s1}{\PYZhy{}}\PY{l+s+s1}{\PYZsq{}}\PY{p}{,}\PY{l+m+mf}{1.904296875}\PY{p}{]}\PY{p}{,}
                              \PY{l+s+s2}{\PYZdq{}}\PY{l+s+s2}{\PYZdl{}\PYZdl{}K\PYZus{}\PYZob{}min\PYZcb{}\PYZdl{}\PYZdl{}}\PY{l+s+s2}{\PYZdq{}}\PY{p}{:}\PY{p}{[}\PY{l+m+mi}{1}\PY{p}{,}\PY{l+m+mi}{1}\PY{p}{,}\PY{l+m+mi}{1}\PY{p}{]}\PY{p}{,}
                              \PY{l+s+s2}{\PYZdq{}}\PY{l+s+s2}{\PYZdl{}\PYZdl{}K\PYZus{}\PYZob{}max\PYZcb{}\PYZdl{}\PYZdl{}}\PY{l+s+s2}{\PYZdq{}}\PY{p}{:}\PY{p}{[}\PY{l+m+mi}{91}\PY{p}{,}\PY{l+m+mi}{127}\PY{p}{,}\PY{l+m+mi}{40}\PY{p}{]}\PY{p}{,}
                              \PY{l+s+s2}{\PYZdq{}}\PY{l+s+s2}{Densidad}\PY{l+s+s2}{\PYZdq{}}\PY{p}{:}\PY{p}{[}\PY{l+m+mf}{0.0014396951973635397}\PY{p}{,}\PY{l+m+mf}{0.006899274397150227}\PY{p}{,}\PY{l+m+mf}{0.002481168566775244}\PY{p}{]}\PY{p}{,}
                              \PY{l+s+s2}{\PYZdq{}}\PY{l+s+s2}{\PYZdl{}\PYZdl{}\PYZlt{}C\PYZus{}\PYZob{}i\PYZcb{}\PYZgt{}\PYZdl{}\PYZdl{}}\PY{l+s+s2}{\PYZdq{}}\PY{p}{:}\PY{p}{[}\PY{l+m+mf}{0.046194001297365124}\PY{p}{,}\PY{l+m+mf}{0.5546360657013015}\PY{p}{,}\PY{l+m+mf}{0.2924923005815713}\PY{p}{]}\PY{p}{,}
                              \PY{l+s+s2}{\PYZdq{}}\PY{l+s+s2}{\PYZdl{}\PYZdl{}C\PYZus{}\PYZob{}}\PY{l+s+s2}{\PYZbs{}}\PY{l+s+s2}{Delta\PYZcb{}\PYZdl{}\PYZdl{}}\PY{l+s+s2}{\PYZdq{}}\PY{p}{:}\PY{p}{[}\PY{l+m+mf}{0.02361415364051535}\PY{p}{,} \PY{l+m+mf}{0.6185901626483971}\PY{p}{,} \PY{l+m+mf}{0.3461926495315878}\PY{p}{]}\PY{p}{,}
                              \PY{l+s+s2}{\PYZdq{}}\PY{l+s+s2}{Diametro (para la componente gigante)}\PY{l+s+s2}{\PYZdq{}}\PY{p}{:}\PY{p}{[}\PY{l+m+mi}{14}\PY{p}{,}\PY{l+m+mi}{15}\PY{p}{,}\PY{l+m+mi}{19}\PY{p}{]}\PY{p}{,}
                             \PY{p}{\PYZcb{}}\PY{p}{)}\PY{c+c1}{\PYZsh{}empty dataframe}
         \PY{n}{data}
\end{Verbatim}


\begin{Verbatim}[commandchars=\\\{\}]
{\color{outcolor}Out[{\color{outcolor}27}]:}   Nombre de la red     N     L Dirigida?  \$\$\textbackslash{}overline\{K\}\$\$  \textbackslash{}
         0              Y2H  2018  2930        No          2.903865   
         1            AP-MS  1622  9070        No         11.183724   
         2              Lit  1536  2925        No          3.808594   
         
           \$\$\textbackslash{}overline\{K\_\{in\}\}\$\$ \$\$\textbackslash{}overline\{K\_\{out\}\}\$\$  \$\$K\_\{min\}\$\$  \$\$K\_\{max\}\$\$  \textbackslash{}
         0               1.45193                1.45193            1           91   
         1                     -                      -            1          127   
         2                1.9043                 1.9043            1           40   
         
            Densidad  \$\$<C\_\{i\}>\$\$  \$\$C\_\{\textbackslash{}Delta\}\$\$  \textbackslash{}
         0  0.001440     0.046194        0.023614   
         1  0.006899     0.554636        0.618590   
         2  0.002481     0.292492        0.346193   
         
            Diametro (para la componente gigante)  
         0                                     14  
         1                                     15  
         2                                     19  
\end{Verbatim}
            
    Para calcular cuantos nodos hay:

    \begin{Verbatim}[commandchars=\\\{\}]
{\color{incolor}In [{\color{incolor} }]:} \PY{k}{print}\PY{p}{(}\PY{l+s+s1}{\PYZsq{}}\PY{l+s+s1}{El número de nodos de cada grafo es}\PY{l+s+s1}{\PYZsq{}}\PY{p}{,}
              \PY{n}{g\PYZus{}y2h}\PY{o}{.}\PY{n}{order}\PY{p}{(}\PY{p}{)}\PY{p}{,} \PY{n}{g\PYZus{}apms}\PY{o}{.}\PY{n}{order}\PY{p}{(}\PY{p}{)}\PY{p}{,} \PY{n}{g\PYZus{}lit}\PY{o}{.}\PY{n}{order}\PY{p}{(}\PY{p}{)}\PY{p}{)} 
\end{Verbatim}


    Para calcular cuantos enlaces hay:

    \begin{Verbatim}[commandchars=\\\{\}]
{\color{incolor}In [{\color{incolor} }]:} \PY{k}{print}\PY{p}{(}\PY{l+s+s1}{\PYZsq{}}\PY{l+s+s1}{El número de enlaces para cada grafo es}\PY{l+s+s1}{\PYZsq{}}\PY{p}{,}
              \PY{n}{g\PYZus{}y2h}\PY{o}{.}\PY{n}{size}\PY{p}{(}\PY{p}{)}\PY{p}{,} \PY{n}{g\PYZus{}apms}\PY{o}{.}\PY{n}{size}\PY{p}{(}\PY{p}{)}\PY{p}{,} \PY{n}{g\PYZus{}lit}\PY{o}{.}\PY{n}{size}\PY{p}{(}\PY{p}{)}\PY{p}{)}
\end{Verbatim}


    Para calcular el grado medio, lo hacemos considerando a las redes Y2H y
Lit como no dirigidas y también como dirigidas, tal como habría sugerido
el criterio considerado inicialmente. De esta manera, calculamos el
grado medio (sin tener en cuenta la dirección de los enlaces) para las 3
redes, y además el grado medio de entrada \(\overline{K_{in}}\) y de
salida \(\overline{K_{out}}\) para esas dos redes dirigidas.

    \begin{Verbatim}[commandchars=\\\{\}]
{\color{incolor}In [{\color{incolor}55}]:} \PY{c+c1}{\PYZsh{} Creamos versiones dirigidas de las redes Y2H y Lit}
         \PY{n}{g\PYZus{}lit\PYZus{}Di} \PY{o}{=} \PY{n}{nx}\PY{o}{.}\PY{n}{DiGraph}\PY{p}{(}\PY{p}{)}
         \PY{n}{g\PYZus{}lit\PYZus{}Di}\PY{o}{.}\PY{n}{add\PYZus{}edges\PYZus{}from}\PY{p}{(}\PY{n}{lit}\PY{p}{)}
         
         \PY{n}{g\PYZus{}y2h\PYZus{}Di} \PY{o}{=} \PY{n}{nx}\PY{o}{.}\PY{n}{DiGraph}\PY{p}{(}\PY{p}{)}
         \PY{n}{g\PYZus{}y2h\PYZus{}Di}\PY{o}{.}\PY{n}{add\PYZus{}edges\PYZus{}from}\PY{p}{(}\PY{n}{y2h}\PY{p}{)}
\end{Verbatim}


    \begin{Verbatim}[commandchars=\\\{\}]
{\color{incolor}In [{\color{incolor}56}]:} \PY{n}{kin\PYZus{}medio\PYZus{}lit}\PY{p}{,} \PY{n}{kout\PYZus{}medio\PYZus{}lit}\PY{p}{,} \PY{n}{k\PYZus{}medio\PYZus{}lit} \PY{o}{=} \PY{n}{k\PYZus{}medio}\PY{p}{(}\PY{n}{g\PYZus{}lit\PYZus{}Di}\PY{p}{)}
         \PY{n}{kin\PYZus{}medio\PYZus{}y2h}\PY{p}{,} \PY{n}{kout\PYZus{}medio\PYZus{}y2h}\PY{p}{,} \PY{n}{k\PYZus{}medio\PYZus{}y2h} \PY{o}{=} \PY{n}{k\PYZus{}medio}\PY{p}{(}\PY{n}{g\PYZus{}y2h\PYZus{}Di}\PY{p}{)}
         \PY{k}{print}\PY{p}{(}\PY{l+s+s1}{\PYZsq{}}\PY{l+s+s1}{Grados medios calculados usando grafos dirigidos (Y2h y Lit)}\PY{l+s+s1}{\PYZsq{}}\PY{p}{)}
         \PY{k}{print}\PY{p}{(}\PY{l+s+s1}{\PYZsq{}}\PY{l+s+s1}{Grados medios}\PY{l+s+s1}{\PYZsq{}}\PY{p}{)}
         \PY{k}{print}\PY{p}{(}\PY{n}{k\PYZus{}medio\PYZus{}y2h}\PY{p}{,} \PY{n}{k\PYZus{}medio\PYZus{}lit}\PY{p}{)}
         \PY{k}{print}\PY{p}{(}\PY{l+s+s1}{\PYZsq{}}\PY{l+s+s1}{Grados medios in}\PY{l+s+s1}{\PYZsq{}}\PY{p}{)}
         \PY{k}{print}\PY{p}{(}\PY{n}{kin\PYZus{}medio\PYZus{}y2h}\PY{p}{,} \PY{n}{kin\PYZus{}medio\PYZus{}lit}\PY{p}{)}
         \PY{k}{print}\PY{p}{(}\PY{l+s+s1}{\PYZsq{}}\PY{l+s+s1}{Grados medios out}\PY{l+s+s1}{\PYZsq{}}\PY{p}{)}
         \PY{k}{print}\PY{p}{(}\PY{n}{kout\PYZus{}medio\PYZus{}y2h}\PY{p}{,} \PY{n}{kout\PYZus{}medio\PYZus{}lit}\PY{p}{)}
         
         \PY{k}{print}\PY{p}{(}\PY{p}{)}
         \PY{n}{\PYZus{}}\PY{p}{,} \PY{n}{\PYZus{}}\PY{p}{,} \PY{n}{k\PYZus{}medio\PYZus{}apms} \PY{o}{=} \PY{n}{k\PYZus{}medio}\PY{p}{(}\PY{n}{g\PYZus{}apms}\PY{p}{)}
         \PY{n}{\PYZus{}}\PY{p}{,} \PY{n}{\PYZus{}}\PY{p}{,} \PY{n}{k\PYZus{}medio\PYZus{}lit} \PY{o}{=} \PY{n}{k\PYZus{}medio}\PY{p}{(}\PY{n}{g\PYZus{}lit}\PY{p}{)}
         \PY{n}{\PYZus{}}\PY{p}{,} \PY{n}{\PYZus{}}\PY{p}{,} \PY{n}{k\PYZus{}medio\PYZus{}y2h} \PY{o}{=} \PY{n}{k\PYZus{}medio}\PY{p}{(}\PY{n}{g\PYZus{}y2h}\PY{p}{)}
         \PY{k}{print}\PY{p}{(}\PY{l+s+s1}{\PYZsq{}}\PY{l+s+s1}{Grados medios calculados usando grafos no dirigidos:}\PY{l+s+s1}{\PYZsq{}}\PY{p}{)}
         \PY{k}{print}\PY{p}{(}\PY{l+s+s1}{\PYZsq{}}\PY{l+s+s1}{Y2H, AP\PYZhy{}MS, Lit}\PY{l+s+s1}{\PYZsq{}}\PY{p}{)}
         \PY{k}{print}\PY{p}{(}\PY{n}{k\PYZus{}medio\PYZus{}y2h}\PY{p}{,} \PY{n}{k\PYZus{}medio\PYZus{}apms}\PY{p}{,} \PY{n}{k\PYZus{}medio\PYZus{}lit}\PY{p}{)}
\end{Verbatim}


    \begin{Verbatim}[commandchars=\\\{\}]
Grados medios calculados usando grafos dirigidos (Y2h y Lit)
Grados medios
2.9038652130822595 3.80859375
Grados medios in
1.4519326065411298 1.904296875
Grados medios out
1.4519326065411298 1.904296875

Grados medios calculados usando grafos no dirigidos:
Y2H, AP-MS, Lit
2.9038652130822595 11.183723797780518 3.80859375

    \end{Verbatim}

    Comparando los valores para redes dirigidas y no dirigidas, vemos que -
Para las redes dirigidas, \(\overline{K_{in}}\) y \(\overline{K_{out}}\)
son iguales. Esto es cierto un poco trivialmente, dado que cada enlace
dirigido es un enlace de salida de un nodo y un enlace de entrada para
otro. Se puede entender también así: si pensamos en una red en donde
existen 2 nodos que no estan conectados entre sí, y ahora tendemos un
enlace dirigido entre ambos, tanto \(\overline{K_{in}}\) como
\(\overline{K_{out}}\) aumentaran. - La
\(\overline{K_{in}} + \overline{K_{out}} = \overline{K}\) donde
\(\overline{K}\) es el grado medio de la red no dirigida
correspondiente.

    Para calcular los grados máximo y el grado mínimo:

    \begin{Verbatim}[commandchars=\\\{\}]
{\color{incolor}In [{\color{incolor}57}]:} \PY{n}{k\PYZus{}min\PYZus{}apms}\PY{p}{,} \PY{n}{k\PYZus{}max\PYZus{}apms} \PY{o}{=} \PY{n}{k\PYZus{}extremos}\PY{p}{(}\PY{n}{g\PYZus{}apms}\PY{p}{)}
         \PY{n}{k\PYZus{}min\PYZus{}lit}\PY{p}{,} \PY{n}{k\PYZus{}max\PYZus{}lit} \PY{o}{=} \PY{n}{k\PYZus{}extremos}\PY{p}{(}\PY{n}{g\PYZus{}lit}\PY{p}{)}
         \PY{n}{k\PYZus{}min\PYZus{}y2h}\PY{p}{,} \PY{n}{k\PYZus{}max\PYZus{}y2h} \PY{o}{=} \PY{n}{k\PYZus{}extremos}\PY{p}{(}\PY{n}{g\PYZus{}y2h}\PY{p}{)}
         \PY{k}{print}\PY{p}{(}\PY{l+s+s1}{\PYZsq{}}\PY{l+s+s1}{Grados extremos}\PY{l+s+s1}{\PYZsq{}}\PY{p}{)}
         \PY{k}{print}\PY{p}{(}\PY{l+s+s1}{\PYZsq{}}\PY{l+s+s1}{K\PYZus{}min:}\PY{l+s+s1}{\PYZsq{}}\PY{p}{,} \PY{n}{k\PYZus{}min\PYZus{}y2h}\PY{p}{,}  \PY{n}{k\PYZus{}min\PYZus{}apms}\PY{p}{,} \PY{n}{k\PYZus{}min\PYZus{}lit}\PY{p}{)}
         \PY{k}{print}\PY{p}{(}\PY{l+s+s1}{\PYZsq{}}\PY{l+s+s1}{K\PYZus{}max:}\PY{l+s+s1}{\PYZsq{}}\PY{p}{,} \PY{n}{k\PYZus{}max\PYZus{}y2h}\PY{p}{,}  \PY{n}{k\PYZus{}max\PYZus{}apms}\PY{p}{,} \PY{n}{k\PYZus{}max\PYZus{}lit}\PY{p}{)}
\end{Verbatim}


    \begin{Verbatim}[commandchars=\\\{\}]
Grados extremos
K\_min: 1 1 1
K\_max: 91 127 40

    \end{Verbatim}

    Para calcular la densidad de la red, usamos la función nx.density.

    \begin{Verbatim}[commandchars=\\\{\}]
{\color{incolor}In [{\color{incolor}58}]:} \PY{k}{print}\PY{p}{(}\PY{l+s+s1}{\PYZsq{}}\PY{l+s+s1}{La densidad de las redes es}\PY{l+s+s1}{\PYZsq{}}\PY{p}{,} \PY{n}{nx}\PY{o}{.}\PY{n}{density}\PY{p}{(}\PY{n}{g\PYZus{}y2h}\PY{p}{)}\PY{p}{,} \PY{n}{nx}\PY{o}{.}\PY{n}{density}\PY{p}{(}\PY{n}{g\PYZus{}apms}\PY{p}{)}\PY{p}{,} \PY{n}{nx}\PY{o}{.}\PY{n}{density}\PY{p}{(}\PY{n}{g\PYZus{}lit}\PY{p}{)}\PY{p}{)}
\end{Verbatim}


    \begin{Verbatim}[commandchars=\\\{\}]
La densidad de las redes es 0.0014396951973635397 0.006899274397150227 0.002481168566775244

    \end{Verbatim}

    En relación al clustering, vamos a calcular dos coeficientes diferentes.
El primero (\(\left<C_i\right>\)) es un promedio entre los coeficientes
de cluesting local de cada nodo de la red. El segundo (\(C_\Delta\)) es
un coeficiente de clustering global que mide la proporción de triganulo
formados en la red sobre el número de triadas de la misma.

\begin{itemize}
\tightlist
\item
  \(\left<C_i\right>\) responde a la pregunta: ``¿En qué medida pares de
  vecinos de un nodo tomado al azar en la red son vecinos entre sí?''
\item
  \(C_\Delta\) responde a la pregunta: ``¿Qué tan probable es encontrar
  triangulos (a.k.a. clausura transitiva) en la red?''
\end{itemize}

    \begin{Verbatim}[commandchars=\\\{\}]
{\color{incolor}In [{\color{incolor}59}]:} \PY{k}{print}\PY{p}{(}\PY{l+s+s1}{\PYZsq{}}\PY{l+s+s1}{C\PYZus{}Δ}\PY{l+s+s1}{\PYZsq{}}\PY{p}{)}
         \PY{k}{print} \PY{p}{(}\PY{n}{nx}\PY{o}{.}\PY{n}{transitivity}\PY{p}{(}\PY{n}{g\PYZus{}y2h}\PY{p}{)}\PY{p}{,} \PY{n}{nx}\PY{o}{.}\PY{n}{transitivity}\PY{p}{(}\PY{n}{g\PYZus{}apms}\PY{p}{)}\PY{p}{,}
                \PY{n}{nx}\PY{o}{.}\PY{n}{transitivity}\PY{p}{(}\PY{n}{g\PYZus{}lit}\PY{p}{)}\PY{p}{)}
         
         \PY{k}{print}\PY{p}{(}\PY{l+s+s1}{\PYZsq{}}\PY{l+s+s1}{\PYZlt{}C\PYZus{}i\PYZgt{}}\PY{l+s+s1}{\PYZsq{}}\PY{p}{)}
         \PY{k}{print}\PY{p}{(}\PY{n}{clustering\PYZus{}medio}\PY{p}{(}\PY{n}{g\PYZus{}y2h}\PY{p}{)}\PY{p}{)}
         \PY{k}{print}\PY{p}{(}\PY{n}{clustering\PYZus{}medio}\PY{p}{(}\PY{n}{g\PYZus{}apms}\PY{p}{)}\PY{p}{)}
         \PY{k}{print}\PY{p}{(}\PY{n}{clustering\PYZus{}medio}\PY{p}{(}\PY{n}{g\PYZus{}lit}\PY{p}{)}\PY{p}{)}
\end{Verbatim}


    \begin{Verbatim}[commandchars=\\\{\}]
C\_Δ
0.02361415364051535 0.6185901626483971 0.3461926495315878
<C\_i>
0.046194001297365124
0.5546360657013015
0.2924923005815713

    \end{Verbatim}

    Observación: El método \(C_\Delta\) no diferencia entre enlaces
entrantes y salientes en redes dirigidas. El método \(\left<C_i\right>\)
no está definido para redes dirigidas.

    Para calcular el diámetro de la red, utilizamos la función nx.diameter.
Dado que los 3 grafos son no conexos, como se puede ver en las
visualizaciones del principio, su diámetro es convencionalmente definido
como infinito. Para obtener un resultado más informativo, consideramos
únicamente sus componentes gigantes, las cuales ya fueron construidas
más arriba.

    \begin{Verbatim}[commandchars=\\\{\}]
{\color{incolor}In [{\color{incolor}61}]:} \PY{k}{print}\PY{p}{(}\PY{l+s+s1}{\PYZsq{}}\PY{l+s+s1}{Diámetro para la red Y2H}\PY{l+s+s1}{\PYZsq{}}\PY{p}{,} \PY{n}{nx}\PY{o}{.}\PY{n}{diameter}\PY{p}{(}\PY{n}{gc\PYZus{}y2h}\PY{p}{)}\PY{p}{)}
         \PY{k}{print}\PY{p}{(}\PY{l+s+s1}{\PYZsq{}}\PY{l+s+s1}{Diámetro para la red AP\PYZhy{}MS}\PY{l+s+s1}{\PYZsq{}}\PY{p}{,} \PY{n}{nx}\PY{o}{.}\PY{n}{diameter}\PY{p}{(}\PY{n}{gc\PYZus{}apms}\PY{p}{)}\PY{p}{)}
         \PY{k}{print}\PY{p}{(}\PY{l+s+s1}{\PYZsq{}}\PY{l+s+s1}{Diámetro para la red Lit}\PY{l+s+s1}{\PYZsq{}}\PY{p}{,} \PY{n}{nx}\PY{o}{.}\PY{n}{diameter}\PY{p}{(}\PY{n}{gc\PYZus{}lit}\PY{p}{)}\PY{p}{)}
\end{Verbatim}


    \begin{Verbatim}[commandchars=\\\{\}]
Diámetro para la red Y2H 14
Diámetro para la red AP-MS 15
Diámetro para la red Lit 19

    \end{Verbatim}

    \paragraph{Discusión}\label{discusiuxf3n}

\begin{itemize}
\tightlist
\item
  La red Y2H tiene casi un 25\% más de nodos que las otras dos.
\item
  La red AP-MS tiene muchísimos más enlaces que las otras dos, lo cual
  queda cuantificado por su densidad, la cual es aproximadamente 3 veces
  más grande que las demás. Esto se refleja en su grado medio, así como
  en sus coeficientes de clustering y el grado máximo alcanzado.
\item
  La red Lit es la red más "larga", en el sentido de que tiene un mayor
  diámetro que las otras dos.
\end{itemize}

    \subsubsection{Puntos c y d}\label{puntos-c-y-d}

    Según lo hablado en la práctica, en vez de encarar estos puntos tal como
están descritos en la guía, nos proponemos decidir entre dos métodos
posibles que podrían haber sido implementados a la hora de construir la
red AP-MS.

    El aspecto relevante del método experimental para generar una red como
nuestra red AP-MS es el siguiente. Se introducen en la célula
anticuerpos que se "enganchan" a una determinada proteína A, y al
hacerlo se terminan acoplando a todo un complejo de proteínas (e.g.
ABCD). Luego se extrae de la célula lo que sea que el anticuerpo
"atrapó", y se identifican las proteínas presentes en el complejo
extraído (A, B, C y D por separado).

La ambigüedad se encuentra en que una red codifica interacciones
binarias, mientras que este método permite relevar acoplamiento mecánico
entre múltiples proteínas sin identificar realmente cuál se acopla con
cuál. Dos maneras para meter esta interacción simultánea entre múltiples
proteínas dentro de un contexto de relaciones binarias:

\begin{enumerate}
\def\labelenumi{\arabic{enumi})}
\item
  \(G=\{A\leftrightarrow B,A\leftrightarrow C,A\leftrightarrow D\}\)
\item
  \(G=GrafoCompleto(A,B,C,D)\).
\end{enumerate}

El primer método posiblemente subestima la cantidad real de
interacciones mecánicas, mientras que el segundo la sobreestima.

    Nos preguntamos entonces cuál de estos dos es el que se aplicó para
obtener la red AP-MS bajo consideración.

En primer lugar, se observa que los coeficientes de clustering y la
densidad de la red AP-MS es mayor que la de la red Y2H y la red Lit, que
a su vez poseen densidad y clusterizacion similar.

Por otro lado, visualizando distintos sectores de la red,
cualitativamente se observa que los enlaces presentes parecerían ser
consistentes con haber sido generados mediante una superposición de
cliques, i.e. se observan grupos de tamaño del orden de 10 nodos
conectados completamente entre sí, y adicionalmente conectados mediante
uno o dos nodos al resto de la red.

Estas dos observaciones sugieren fuertemente que el método adoptado en
este caso fue el segundo, sobreestimando la cantidad de enlaces
presentes en la red.

    \subsection{Ejercicio 2}\label{ejercicio-2}

    Importamos las librerías.

    \begin{Verbatim}[commandchars=\\\{\}]
{\color{incolor}In [{\color{incolor}65}]:} \PY{k+kn}{from} \PY{n+nn}{\PYZus{}\PYZus{}future\PYZus{}\PYZus{}} \PY{k+kn}{import} \PY{n}{division}
         \PY{k+kn}{import} \PY{n+nn}{networkx} \PY{k+kn}{as} \PY{n+nn}{nx}
         \PY{k+kn}{import} \PY{n+nn}{numpy} \PY{k+kn}{as} \PY{n+nn}{np}
         \PY{k+kn}{import} \PY{n+nn}{matplotlib.pyplot} \PY{k+kn}{as} \PY{n+nn}{plt}
         \PY{k+kn}{from} \PY{n+nn}{networkx.readwrite.gml} \PY{k+kn}{import} \PY{n}{read\PYZus{}gml}
         \PY{k+kn}{from} \PY{n+nn}{random} \PY{k+kn}{import} \PY{n}{sample}
         
         \PY{k+kn}{from} \PY{n+nn}{lectura} \PY{k+kn}{import} \PY{n}{ldata}
         \PY{k+kn}{from} \PY{n+nn}{histograma} \PY{k+kn}{import} \PY{n}{histograma}
         \PY{k+kn}{from} \PY{n+nn}{graficar\PYZus{}multipartito} \PY{k+kn}{import} \PY{o}{*}
         
         \PY{k+kn}{from} \PY{n+nn}{collections} \PY{k+kn}{import} \PY{n}{Counter}
\end{Verbatim}


    Definimos las funciones que vamos a utilizar. Algunas funciones de
visualización empleadas las importamos de archivos de código separados,
para no congestionar el notebook.

    \begin{Verbatim}[commandchars=\\\{\}]
{\color{incolor}In [{\color{incolor}73}]:} \PY{c+c1}{\PYZsh{}Para graficar}
         \PY{k}{def} \PY{n+nf}{genero\PYZus{}a\PYZus{}color}\PY{p}{(}\PY{n}{gender}\PY{p}{)}\PY{p}{:}
             \PY{k}{if} \PY{n}{gender}\PY{o}{==}\PY{l+s+s1}{\PYZsq{}}\PY{l+s+s1}{m}\PY{l+s+s1}{\PYZsq{}}\PY{p}{:}
                 \PY{k}{return} \PY{l+s+s1}{\PYZsq{}}\PY{l+s+s1}{red}\PY{l+s+s1}{\PYZsq{}}
             \PY{k}{elif} \PY{n}{gender}\PY{o}{==}\PY{l+s+s1}{\PYZsq{}}\PY{l+s+s1}{f}\PY{l+s+s1}{\PYZsq{}}\PY{p}{:}
                 \PY{k}{return} \PY{l+s+s1}{\PYZsq{}}\PY{l+s+s1}{dodgerblue}\PY{l+s+s1}{\PYZsq{}}
             \PY{k}{else}\PY{p}{:}
                 \PY{k}{return} \PY{l+s+s1}{\PYZsq{}}\PY{l+s+s1}{green}\PY{l+s+s1}{\PYZsq{}}
             
         \PY{k}{def} \PY{n+nf}{particionar\PYZus{}por\PYZus{}genero}\PY{p}{(}\PY{n}{G}\PY{p}{,} \PY{n}{orden}\PY{o}{=}\PY{p}{\PYZob{}}\PY{l+s+s1}{\PYZsq{}}\PY{l+s+s1}{f}\PY{l+s+s1}{\PYZsq{}}\PY{p}{:}\PY{l+m+mi}{0}\PY{p}{,} \PY{l+s+s1}{\PYZsq{}}\PY{l+s+s1}{NA}\PY{l+s+s1}{\PYZsq{}}\PY{p}{:}\PY{l+m+mi}{1}\PY{p}{,} \PY{l+s+s1}{\PYZsq{}}\PY{l+s+s1}{m}\PY{l+s+s1}{\PYZsq{}}\PY{p}{:}\PY{l+m+mi}{2}\PY{p}{\PYZcb{}}\PY{p}{)}\PY{p}{:}
             \PY{n}{particiones} \PY{o}{=} \PY{p}{[}\PY{p}{[}\PY{p}{]}\PY{p}{,} \PY{p}{[}\PY{p}{]}\PY{p}{,} \PY{p}{[}\PY{p}{]}\PY{p}{]}
             \PY{k}{for} \PY{n}{key} \PY{o+ow}{in} \PY{n+nb}{dict}\PY{p}{(}\PY{n}{G}\PY{o}{.}\PY{n}{nodes}\PY{p}{)}\PY{o}{.}\PY{n}{keys}\PY{p}{(}\PY{p}{)}\PY{p}{:}
                 \PY{n}{gender} \PY{o}{=} \PY{n}{G}\PY{o}{.}\PY{n}{nodes}\PY{p}{[}\PY{n}{key}\PY{p}{]}\PY{p}{[}\PY{l+s+s1}{\PYZsq{}}\PY{l+s+s1}{gender}\PY{l+s+s1}{\PYZsq{}}\PY{p}{]}
                 \PY{k}{if} \PY{n}{gender}\PY{o}{==}\PY{l+s+s1}{\PYZsq{}}\PY{l+s+s1}{f}\PY{l+s+s1}{\PYZsq{}}\PY{p}{:}
                     \PY{n}{particiones}\PY{p}{[}\PY{n}{orden}\PY{p}{[}\PY{n}{gender}\PY{p}{]}\PY{p}{]}\PY{o}{.}\PY{n}{append}\PY{p}{(}\PY{n}{key}\PY{p}{)}
                 \PY{k}{elif} \PY{n}{gender}\PY{o}{==}\PY{l+s+s1}{\PYZsq{}}\PY{l+s+s1}{m}\PY{l+s+s1}{\PYZsq{}}\PY{p}{:}
                     \PY{n}{particiones}\PY{p}{[}\PY{n}{orden}\PY{p}{[}\PY{n}{gender}\PY{p}{]}\PY{p}{]}\PY{o}{.}\PY{n}{append}\PY{p}{(}\PY{n}{key}\PY{p}{)}
                 \PY{k}{else}\PY{p}{:}
                     \PY{n}{particiones}\PY{p}{[}\PY{n}{orden}\PY{p}{[}\PY{n}{gender}\PY{p}{]}\PY{p}{]}\PY{o}{.}\PY{n}{append}\PY{p}{(}\PY{n}{key}\PY{p}{)}
             \PY{k}{return} \PY{n}{particiones}
         
         \PY{k}{def} \PY{n+nf}{crear\PYZus{}leyenda}\PY{p}{(}\PY{n}{ax}\PY{p}{)}\PY{p}{:}
             \PY{n}{ax}\PY{o}{.}\PY{n}{plot}\PY{p}{(}\PY{p}{[}\PY{p}{]}\PY{p}{,}\PY{p}{[}\PY{p}{]}\PY{p}{,} \PY{l+s+s1}{\PYZsq{}}\PY{l+s+s1}{o}\PY{l+s+s1}{\PYZsq{}}\PY{p}{,} \PY{n}{color}\PY{o}{=}\PY{l+s+s1}{\PYZsq{}}\PY{l+s+s1}{dodgerblue}\PY{l+s+s1}{\PYZsq{}}\PY{p}{,} \PY{n}{label}\PY{o}{=}\PY{l+s+s1}{\PYZsq{}}\PY{l+s+s1}{Hembras}\PY{l+s+s1}{\PYZsq{}}\PY{p}{)}
             \PY{n}{ax}\PY{o}{.}\PY{n}{plot}\PY{p}{(}\PY{p}{[}\PY{p}{]}\PY{p}{,}\PY{p}{[}\PY{p}{]}\PY{p}{,} \PY{l+s+s1}{\PYZsq{}}\PY{l+s+s1}{o}\PY{l+s+s1}{\PYZsq{}}\PY{p}{,} \PY{n}{color}\PY{o}{=}\PY{l+s+s1}{\PYZsq{}}\PY{l+s+s1}{red}\PY{l+s+s1}{\PYZsq{}}\PY{p}{,} \PY{n}{label}\PY{o}{=}\PY{l+s+s1}{\PYZsq{}}\PY{l+s+s1}{Machos}\PY{l+s+s1}{\PYZsq{}}\PY{p}{)}
             \PY{n}{ax}\PY{o}{.}\PY{n}{plot}\PY{p}{(}\PY{p}{[}\PY{p}{]}\PY{p}{,}\PY{p}{[}\PY{p}{]}\PY{p}{,} \PY{l+s+s1}{\PYZsq{}}\PY{l+s+s1}{o}\PY{l+s+s1}{\PYZsq{}}\PY{p}{,} \PY{n}{color}\PY{o}{=}\PY{l+s+s1}{\PYZsq{}}\PY{l+s+s1}{black}\PY{l+s+s1}{\PYZsq{}}\PY{p}{,} \PY{n}{label}\PY{o}{=}\PY{l+s+s1}{\PYZsq{}}\PY{l+s+s1}{NA}\PY{l+s+s1}{\PYZsq{}}\PY{p}{)}
             \PY{n}{ax}\PY{o}{.}\PY{n}{legend}\PY{p}{(}\PY{p}{)}
\end{Verbatim}


    \begin{Verbatim}[commandchars=\\\{\}]
{\color{incolor}In [{\color{incolor}66}]:} \PY{c+c1}{\PYZsh{}Para cuantificar homofilia.}
         \PY{k}{def} \PY{n+nf}{contar\PYZus{}clases}\PY{p}{(}\PY{n}{g}\PY{p}{,} \PY{n}{atributo}\PY{p}{,} \PY{n}{valores\PYZus{}posibles}\PY{p}{)}\PY{p}{:}
             \PY{n}{ns} \PY{o}{=} \PY{p}{[}\PY{p}{]}
             \PY{k}{for} \PY{n}{valor} \PY{o+ow}{in} \PY{n}{valores\PYZus{}posibles}\PY{p}{:}
                 \PY{n}{n} \PY{o}{=} \PY{n+nb}{len}\PY{p}{(}\PY{p}{[}\PY{n}{n} \PY{k}{for} \PY{n}{n}\PY{p}{,} \PY{n}{attrdict} \PY{o+ow}{in} \PY{n+nb}{dict}\PY{p}{(}\PY{n}{g}\PY{o}{.}\PY{n}{nodes}\PY{p}{)}\PY{o}{.}\PY{n}{items}\PY{p}{(}\PY{p}{)} \PY{k}{if} \PY{n}{attrdict}\PY{p}{[}\PY{n}{atributo}\PY{p}{]}\PY{o}{==}\PY{n}{valor}\PY{p}{]}\PY{p}{)}
                 \PY{n}{ns}\PY{o}{.}\PY{n}{append}\PY{p}{(}\PY{n}{n}\PY{p}{)}
                 \PY{k}{print}\PY{p}{(}\PY{l+s+s1}{\PYZsq{}}\PY{l+s+s1}{Hay \PYZob{}\PYZcb{} nodos con \PYZob{}\PYZcb{}=\PYZob{}\PYZcb{}}\PY{l+s+s1}{\PYZsq{}}\PY{o}{.}\PY{n}{format}\PY{p}{(}\PY{n}{n}\PY{p}{,} \PY{n}{atributo}\PY{p}{,} \PY{n}{valor}\PY{p}{)}\PY{p}{)}
             \PY{k}{return} \PY{n}{ns}
         
         \PY{k}{def} \PY{n+nf}{contar\PYZus{}enlaces\PYZus{}internos}\PY{p}{(}\PY{n}{g}\PY{p}{,} \PY{n}{atributo}\PY{p}{,} \PY{n}{valor}\PY{p}{)}\PY{p}{:}
             \PY{l+s+sd}{\PYZdq{}\PYZdq{}\PYZdq{}Cuenta los enlaces internos en el grupo de nodos que tienen}
         \PY{l+s+sd}{    atributo=valor. g debe ser objeto Graph de NetworkX.}
         \PY{l+s+sd}{    Ejemplo con delfines: atributo=\PYZsq{}gender\PYZsq{}, valor puede ser \PYZsq{}f\PYZsq{}, \PYZsq{}m\PYZsq{} o \PYZsq{}NA\PYZsq{}}
         \PY{l+s+sd}{    \PYZdq{}\PYZdq{}\PYZdq{}}
             \PY{n}{nodos} \PY{o}{=} \PY{p}{[}\PY{n}{n} \PY{k}{for} \PY{n}{n}\PY{p}{,} \PY{n}{attrdict} \PY{o+ow}{in} \PY{n+nb}{dict}\PY{p}{(}\PY{n}{g}\PY{o}{.}\PY{n}{nodes}\PY{p}{)}\PY{o}{.}\PY{n}{items}\PY{p}{(}\PY{p}{)} \PY{k}{if} \PY{n}{attrdict}\PY{p}{[}\PY{n}{atributo}\PY{p}{]}\PY{o}{==}\PY{n}{valor}\PY{p}{]}
             \PY{n}{grupo} \PY{o}{=} \PY{n}{nx}\PY{o}{.}\PY{n}{subgraph}\PY{p}{(}\PY{n}{g}\PY{p}{,} \PY{n}{nodos}\PY{p}{)}
             \PY{k}{return} \PY{n}{grupo}\PY{o}{.}\PY{n}{size}\PY{p}{(}\PY{p}{)}
         
         \PY{k}{def} \PY{n+nf}{contar\PYZus{}enlaces\PYZus{}entre\PYZus{}grupos}\PY{p}{(}\PY{n}{g}\PY{p}{,} \PY{n}{atributo}\PY{p}{)}\PY{p}{:}
             \PY{l+s+sd}{\PYZdq{}\PYZdq{}\PYZdq{}Cuenta los enlaces que conectan grupos con valores distintos de}
         \PY{l+s+sd}{    atributo. g debe ser objeto Graph de NetworkX.}
         \PY{l+s+sd}{    Ejemplo con delfines: atributo=\PYZsq{}gender\PYZsq{}.}
         \PY{l+s+sd}{    \PYZdq{}\PYZdq{}\PYZdq{}}
             \PY{n}{n} \PY{o}{=} \PY{l+m+mi}{0}
             \PY{k}{for} \PY{n}{edge} \PY{o+ow}{in} \PY{n}{g}\PY{o}{.}\PY{n}{edges}\PY{p}{(}\PY{p}{)}\PY{p}{:}
                 \PY{n}{a}\PY{p}{,} \PY{n}{b} \PY{o}{=} \PY{n}{edge}\PY{p}{[}\PY{l+m+mi}{0}\PY{p}{]}\PY{p}{,} \PY{n}{edge}\PY{p}{[}\PY{l+m+mi}{1}\PY{p}{]}
                 \PY{k}{if} \PY{n}{g}\PY{o}{.}\PY{n}{nodes}\PY{p}{(}\PY{p}{)}\PY{p}{[}\PY{n}{a}\PY{p}{]}\PY{p}{[}\PY{n}{atributo}\PY{p}{]} \PY{o}{!=} \PY{n}{g}\PY{o}{.}\PY{n}{nodes}\PY{p}{(}\PY{p}{)}\PY{p}{[}\PY{n}{b}\PY{p}{]}\PY{p}{[}\PY{n}{atributo}\PY{p}{]}\PY{p}{:}
                     \PY{n}{n} \PY{o}{=} \PY{n}{n} \PY{o}{+} \PY{l+m+mi}{1}
             \PY{k}{return} \PY{n}{n}
\end{Verbatim}


    \begin{Verbatim}[commandchars=\\\{\}]
{\color{incolor}In [{\color{incolor}68}]:} \PY{c+c1}{\PYZsh{}Para el calculo de la modularidad}
         \PY{k}{def} \PY{n+nf}{modularidad}\PY{p}{(}\PY{n}{g}\PY{p}{,} \PY{n}{atributo}\PY{p}{)}\PY{p}{:}
             \PY{n}{A} \PY{o}{=} \PY{n}{nx}\PY{o}{.}\PY{n}{adjacency\PYZus{}matrix}\PY{p}{(}\PY{n}{g}\PY{p}{)}
             \PY{n}{B} \PY{o}{=} \PY{l+m+mi}{0}
             \PY{n}{m} \PY{o}{=} \PY{n+nb}{len}\PY{p}{(}\PY{n}{g}\PY{o}{.}\PY{n}{edges}\PY{p}{(}\PY{p}{)}\PY{p}{)}
             \PY{n}{k} \PY{o}{=} \PY{p}{[}\PY{p}{]}
             \PY{n}{name} \PY{o}{=} \PY{p}{[}\PY{p}{]}
             \PY{k}{for} \PY{n}{key}\PY{p}{,} \PY{n}{value} \PY{o+ow}{in} \PY{n}{g}\PY{o}{.}\PY{n}{degree}\PY{p}{(}\PY{p}{)}\PY{p}{:}
                 \PY{n}{k}\PY{o}{.}\PY{n}{append}\PY{p}{(}\PY{n}{value}\PY{p}{)}
                 \PY{n}{name}\PY{o}{.}\PY{n}{append}\PY{p}{(}\PY{n}{key}\PY{p}{)}
             \PY{k}{for} \PY{n}{i} \PY{o+ow}{in} \PY{n+nb}{range}\PY{p}{(}\PY{n+nb}{len}\PY{p}{(}\PY{n}{k}\PY{p}{)}\PY{p}{)}\PY{p}{:}
                 \PY{k}{for} \PY{n}{j} \PY{o+ow}{in} \PY{n+nb}{range}\PY{p}{(}\PY{n+nb}{len}\PY{p}{(}\PY{n}{k}\PY{p}{)}\PY{p}{)}\PY{p}{:}
                     \PY{k}{if} \PY{n}{g}\PY{o}{.}\PY{n}{node}\PY{p}{[}\PY{n}{name}\PY{p}{[}\PY{n}{i}\PY{p}{]}\PY{p}{]}\PY{p}{[}\PY{n}{atributo}\PY{p}{]} \PY{o}{==} \PY{n}{g}\PY{o}{.}\PY{n}{node}\PY{p}{[}\PY{n}{name}\PY{p}{[}\PY{n}{j}\PY{p}{]}\PY{p}{]}\PY{p}{[}\PY{n}{atributo}\PY{p}{]}\PY{p}{:}
                         \PY{n}{B} \PY{o}{+}\PY{o}{=} \PY{n}{A}\PY{p}{[}\PY{n}{i}\PY{p}{,} \PY{n}{j}\PY{p}{]} \PY{o}{\PYZhy{}} \PY{n}{k}\PY{p}{[}\PY{n}{i}\PY{p}{]}\PY{o}{*}\PY{n}{k}\PY{p}{[}\PY{n}{j}\PY{p}{]}\PY{o}{/}\PY{p}{(}\PY{l+m+mi}{2}\PY{o}{*}\PY{n}{m}\PY{p}{)}
             \PY{k}{return} \PY{n+nb}{float}\PY{p}{(}\PY{n}{B}\PY{p}{)}\PY{o}{/}\PY{p}{(}\PY{l+m+mi}{2}\PY{o}{*}\PY{n}{m}\PY{p}{)}
\end{Verbatim}


    \begin{Verbatim}[commandchars=\\\{\}]
{\color{incolor}In [{\color{incolor}128}]:} \PY{c+c1}{\PYZsh{} Para calcular el p\PYZhy{}value del número de enlaces entre delfines de distinto}
          \PY{c+c1}{\PYZsh{} género, y de la modularidad de la red, bajo la hipótesis nula.}
          
          \PY{c+c1}{\PYZsh{}def p\PYZus{}value(datos, valor\PYZus{}observado):}
          \PY{c+c1}{\PYZsh{}    enlace = list(Counter(datos).keys())}
          \PY{c+c1}{\PYZsh{}    cuantos = list(Counter(datos).values())}
          \PY{c+c1}{\PYZsh{}    indices = np.where(np.array(enlace)\PYZlt{}= valor\PYZus{}real)[0] }
          \PY{c+c1}{\PYZsh{}    a = [cuantos[i] for i in indices]}
          \PY{c+c1}{\PYZsh{}    return 2 * len(a)/len(datos)}
          
          \PY{k}{def} \PY{n+nf}{p\PYZus{}value}\PY{p}{(}\PY{n}{datos}\PY{p}{,} \PY{n}{valor\PYZus{}observado}\PY{p}{,} \PY{n}{cola}\PY{p}{,} \PY{n}{doscolas}\PY{o}{=}\PY{n+nb+bp}{False}\PY{p}{)}\PY{p}{:}
              \PY{k}{assert} \PY{n}{cola} \PY{o+ow}{in} \PY{p}{[}\PY{l+s+s1}{\PYZsq{}}\PY{l+s+s1}{izq}\PY{l+s+s1}{\PYZsq{}}\PY{p}{,} \PY{l+s+s1}{\PYZsq{}}\PY{l+s+s1}{der}\PY{l+s+s1}{\PYZsq{}}\PY{p}{]}
              \PY{k}{if} \PY{n}{cola} \PY{o}{==} \PY{l+s+s1}{\PYZsq{}}\PY{l+s+s1}{izq}\PY{l+s+s1}{\PYZsq{}}\PY{p}{:}
                  \PY{n}{frac} \PY{o}{=} \PY{n}{np}\PY{o}{.}\PY{n}{sum}\PY{p}{(}\PY{n}{datos} \PY{o}{\PYZlt{}}\PY{o}{=} \PY{n}{valor\PYZus{}observado}\PY{p}{)} \PY{o}{/} \PY{n+nb}{len}\PY{p}{(}\PY{n}{datos}\PY{p}{)}
              \PY{k}{else}\PY{p}{:}
                  \PY{n}{frac} \PY{o}{=} \PY{n}{np}\PY{o}{.}\PY{n}{sum}\PY{p}{(}\PY{n}{datos} \PY{o}{\PYZgt{}}\PY{o}{=} \PY{n}{valor\PYZus{}observado}\PY{p}{)} \PY{o}{/} \PY{n+nb}{len}\PY{p}{(}\PY{n}{datos}\PY{p}{)}
              \PY{k}{if} \PY{n}{doscolas}\PY{p}{:}
                  \PY{k}{return} \PY{n}{frac} \PY{o}{*} \PY{l+m+mi}{2}
              \PY{k}{else}\PY{p}{:}
                  \PY{k}{return} \PY{n}{frac}
\end{Verbatim}


    Importamos la red de delfines y a cada uno le asignamos su género.

    \begin{Verbatim}[commandchars=\\\{\}]
{\color{incolor}In [{\color{incolor}129}]:} \PY{n}{dolph} \PY{o}{=} \PY{n}{read\PYZus{}gml}\PY{p}{(}\PY{l+s+s1}{\PYZsq{}}\PY{l+s+s1}{tc01\PYZus{}data/new\PYZus{}dolphins.gml}\PY{l+s+s1}{\PYZsq{}}\PY{p}{)}
          \PY{n}{genders} \PY{o}{=} \PY{n+nb}{dict}\PY{p}{(}\PY{n}{ldata}\PY{p}{(}\PY{l+s+s1}{\PYZsq{}}\PY{l+s+s1}{tc01\PYZus{}data/dolphinsGender.txt}\PY{l+s+s1}{\PYZsq{}}\PY{p}{)}\PY{p}{)}
          
          \PY{c+c1}{\PYZsh{} Agregamos los géneros a los dicts de cada delfín}
          \PY{k}{for} \PY{n}{nodo}\PY{p}{,} \PY{n}{dict\PYZus{}nodo} \PY{o+ow}{in} \PY{n+nb}{dict}\PY{p}{(}\PY{n}{dolph}\PY{o}{.}\PY{n}{nodes}\PY{p}{)}\PY{o}{.}\PY{n}{items}\PY{p}{(}\PY{p}{)}\PY{p}{:}
              \PY{n}{dict\PYZus{}nodo}\PY{p}{[}\PY{l+s+s1}{\PYZsq{}}\PY{l+s+s1}{gender}\PY{l+s+s1}{\PYZsq{}}\PY{p}{]} \PY{o}{=} \PY{n}{genders}\PY{p}{[}\PY{n}{nodo}\PY{p}{]} \PY{c+c1}{\PYZsh{} agrego el género del delfín a su dict}
          \PY{c+c1}{\PYZsh{}    print(\PYZsq{}Key = \PYZob{}\PYZcb{}, Value = \PYZob{}\PYZcb{}\PYZsq{}.format(nodo, dict\PYZus{}nodo)) \PYZsh{} para chequear que anda}
          
          \PY{n}{particiones} \PY{o}{=} \PY{n}{particionar\PYZus{}por\PYZus{}genero}\PY{p}{(}\PY{n}{dolph}\PY{p}{)}
          \PY{n}{colores} \PY{o}{=} \PY{p}{[}\PY{n}{genero\PYZus{}a\PYZus{}color}\PY{p}{(}\PY{n}{g}\PY{p}{)} \PY{k}{for} \PY{n}{g} \PY{o+ow}{in} \PY{n}{nx}\PY{o}{.}\PY{n}{get\PYZus{}node\PYZus{}attributes}\PY{p}{(}\PY{n}{dolph}\PY{p}{,} \PY{l+s+s2}{\PYZdq{}}\PY{l+s+s2}{gender}\PY{l+s+s2}{\PYZdq{}}\PY{p}{)}\PY{o}{.}\PY{n}{values}\PY{p}{(}\PY{p}{)}\PY{p}{]}    
\end{Verbatim}


    Hay 34 delfines macho, 24 delfines hembra y 4 delfines sin información.

    \subsubsection{Punto a}\label{punto-a}

    Exploramos distintos tipos de visualizaciones, intentando que las mismas
nos permitan observar cómo es el conexionado entre los delfines de
distinto género, así como el conexionado interno de cada grupo. Para
esto graficamos de 9 maneras diferentes.

    \begin{Verbatim}[commandchars=\\\{\}]
{\color{incolor}In [{\color{incolor}130}]:} \PY{n}{fig}\PY{p}{,} \PY{n}{axes} \PY{o}{=} \PY{n}{plt}\PY{o}{.}\PY{n}{subplots}\PY{p}{(}\PY{l+m+mi}{3}\PY{p}{,}\PY{l+m+mi}{3}\PY{p}{,} \PY{n}{figsize}\PY{o}{=}\PY{p}{(}\PY{l+m+mi}{16}\PY{p}{,} \PY{l+m+mi}{16}\PY{p}{)}\PY{p}{)}
          \PY{n}{axes} \PY{o}{=} \PY{n}{axes}\PY{o}{.}\PY{n}{flatten}\PY{p}{(}\PY{p}{)}
          \PY{n}{ns} \PY{o}{=} \PY{l+m+mi}{25} \PY{c+c1}{\PYZsh{} Node size}
          
          \PY{c+c1}{\PYZsh{} Posicionamiento multipartito (es engañoso: no se ven los links homofílicos)}
          \PY{n}{multipartite\PYZus{}pos} \PY{o}{=} \PY{n}{position\PYZus{}multipartito}\PY{p}{(}\PY{n}{dolph}\PY{p}{,} \PY{p}{[}\PY{l+s+s1}{\PYZsq{}}\PY{l+s+s1}{f}\PY{l+s+s1}{\PYZsq{}}\PY{p}{,} \PY{l+s+s1}{\PYZsq{}}\PY{l+s+s1}{NA}\PY{l+s+s1}{\PYZsq{}}\PY{p}{,} \PY{l+s+s1}{\PYZsq{}}\PY{l+s+s1}{m}\PY{l+s+s1}{\PYZsq{}}\PY{p}{]}\PY{p}{,} \PY{l+s+s1}{\PYZsq{}}\PY{l+s+s1}{gender}\PY{l+s+s1}{\PYZsq{}}\PY{p}{)}
          \PY{n}{nx}\PY{o}{.}\PY{n}{draw}\PY{p}{(}\PY{n}{dolph}\PY{p}{,} \PY{n}{ax} \PY{o}{=} \PY{n}{axes}\PY{p}{[}\PY{l+m+mi}{0}\PY{p}{]}\PY{p}{,} \PY{n}{node\PYZus{}size} \PY{o}{=} \PY{n}{ns}\PY{p}{,} \PY{n}{node\PYZus{}color}\PY{o}{=}\PY{n}{colores}\PY{p}{,}
                  \PY{n}{pos}\PY{o}{=}\PY{n}{multipartite\PYZus{}pos}\PY{p}{)}
          \PY{n}{axes}\PY{p}{[}\PY{l+m+mi}{0}\PY{p}{]}\PY{o}{.}\PY{n}{set\PYZus{}title}\PY{p}{(}\PY{l+s+s1}{\PYZsq{}}\PY{l+s+s1}{Posicionamiento multipartito}\PY{l+s+s1}{\PYZsq{}}\PY{p}{)}
          
          \PY{c+c1}{\PYZsh{} Posicionamiento en un círculo}
          \PY{n}{nx}\PY{o}{.}\PY{n}{draw\PYZus{}circular}\PY{p}{(}\PY{n}{dolph}\PY{p}{,} \PY{n}{ax} \PY{o}{=} \PY{n}{axes}\PY{p}{[}\PY{l+m+mi}{1}\PY{p}{]}\PY{p}{,} \PY{n}{node\PYZus{}size} \PY{o}{=} \PY{n}{ns}\PY{p}{,} \PY{n}{node\PYZus{}color}\PY{o}{=}\PY{n}{colores}\PY{p}{)}
          \PY{n}{axes}\PY{p}{[}\PY{l+m+mi}{1}\PY{p}{]}\PY{o}{.}\PY{n}{set\PYZus{}title}\PY{p}{(}\PY{l+s+s1}{\PYZsq{}}\PY{l+s+s1}{Posicionamiento en un círculo}\PY{l+s+s1}{\PYZsq{}}\PY{p}{)}
          
          \PY{c+c1}{\PYZsh{} Posicionamiento en círculos concéntricos}
          \PY{n}{shell\PYZus{}pos} \PY{o}{=} \PY{n}{nx}\PY{o}{.}\PY{n}{drawing}\PY{o}{.}\PY{n}{layout}\PY{o}{.}\PY{n}{shell\PYZus{}layout}\PY{p}{(}\PY{n}{dolph}\PY{p}{,} \PY{n}{particiones}\PY{p}{)}
          \PY{n}{nx}\PY{o}{.}\PY{n}{draw}\PY{p}{(}\PY{n}{dolph}\PY{p}{,} \PY{n}{ax} \PY{o}{=} \PY{n}{axes}\PY{p}{[}\PY{l+m+mi}{2}\PY{p}{]}\PY{p}{,} \PY{n}{node\PYZus{}size} \PY{o}{=} \PY{n}{ns}\PY{p}{,} \PY{n}{node\PYZus{}color}\PY{o}{=}\PY{n}{colores}\PY{p}{,}
                  \PY{n}{pos}\PY{o}{=}\PY{n}{shell\PYZus{}pos}\PY{p}{)}
          \PY{n}{axes}\PY{p}{[}\PY{l+m+mi}{2}\PY{p}{]}\PY{o}{.}\PY{n}{set\PYZus{}title}\PY{p}{(}\PY{l+s+s1}{\PYZsq{}}\PY{l+s+s1}{Posicionamiento en círculos concéntricos}\PY{l+s+s1}{\PYZsq{}}\PY{p}{)}
          
          \PY{c+c1}{\PYZsh{} Posicionamiento al azar}
          \PY{n}{nx}\PY{o}{.}\PY{n}{draw\PYZus{}random}\PY{p}{(}\PY{n}{dolph}\PY{p}{,} \PY{n}{ax} \PY{o}{=} \PY{n}{axes}\PY{p}{[}\PY{l+m+mi}{3}\PY{p}{]}\PY{p}{,} \PY{n}{node\PYZus{}size} \PY{o}{=} \PY{n}{ns}\PY{p}{,} \PY{n}{node\PYZus{}color}\PY{o}{=}\PY{n}{colores}\PY{p}{)}
          \PY{n}{axes}\PY{p}{[}\PY{l+m+mi}{3}\PY{p}{]}\PY{o}{.}\PY{n}{set\PYZus{}title}\PY{p}{(}\PY{l+s+s1}{\PYZsq{}}\PY{l+s+s1}{Posicionamiento al azar}\PY{l+s+s1}{\PYZsq{}}\PY{p}{)}
          
          \PY{c+c1}{\PYZsh{} Posicionamiento espectral}
          \PY{n}{nx}\PY{o}{.}\PY{n}{draw\PYZus{}spectral}\PY{p}{(}\PY{n}{dolph}\PY{p}{,} \PY{n}{ax} \PY{o}{=} \PY{n}{axes}\PY{p}{[}\PY{l+m+mi}{4}\PY{p}{]}\PY{p}{,}
                           \PY{n}{node\PYZus{}size} \PY{o}{=} \PY{n}{ns}\PY{p}{,} \PY{n}{node\PYZus{}color}\PY{o}{=}\PY{n}{colores}\PY{p}{)}
          \PY{n}{axes}\PY{p}{[}\PY{l+m+mi}{4}\PY{p}{]}\PY{o}{.}\PY{n}{set\PYZus{}title}\PY{p}{(}\PY{l+s+s1}{\PYZsq{}}\PY{l+s+s1}{Posicionamiento espectral}\PY{l+s+s1}{\PYZsq{}}\PY{p}{)}
          
          \PY{c+c1}{\PYZsh{} Posicionamiento por resortes}
          \PY{n}{nx}\PY{o}{.}\PY{n}{draw\PYZus{}spring}\PY{p}{(}\PY{n}{dolph}\PY{p}{,} \PY{n}{ax} \PY{o}{=} \PY{n}{axes}\PY{p}{[}\PY{l+m+mi}{5}\PY{p}{]}\PY{p}{,}
                           \PY{n}{node\PYZus{}size} \PY{o}{=} \PY{n}{ns}\PY{p}{,} \PY{n}{with\PYZus{}labels}\PY{o}{=}\PY{n+nb+bp}{False}\PY{p}{,} \PY{n}{node\PYZus{}color}\PY{o}{=}\PY{n}{colores}\PY{p}{)}
          \PY{n}{axes}\PY{p}{[}\PY{l+m+mi}{5}\PY{p}{]}\PY{o}{.}\PY{n}{set\PYZus{}title}\PY{p}{(}\PY{l+s+s1}{\PYZsq{}}\PY{l+s+s1}{Posicionamiento por resortes}\PY{l+s+s1}{\PYZsq{}}\PY{p}{)}
          
          \PY{c+c1}{\PYZsh{} Posicionamiento multipartito al azar. Posiciono al azar y}
          \PY{c+c1}{\PYZsh{} luego desplazo lateralmente según género}
          
          \PY{n}{multi\PYZus{}random\PYZus{}pos} \PY{o}{=} \PY{n}{position\PYZus{}multipartito\PYZus{}random}\PY{p}{(}\PY{n}{dolph}\PY{p}{,}
                                                          \PY{p}{[}\PY{l+s+s1}{\PYZsq{}}\PY{l+s+s1}{f}\PY{l+s+s1}{\PYZsq{}}\PY{p}{,} \PY{l+s+s1}{\PYZsq{}}\PY{l+s+s1}{m}\PY{l+s+s1}{\PYZsq{}}\PY{p}{,} \PY{l+s+s1}{\PYZsq{}}\PY{l+s+s1}{NA}\PY{l+s+s1}{\PYZsq{}}\PY{p}{]}\PY{p}{,} \PY{l+s+s1}{\PYZsq{}}\PY{l+s+s1}{gender}\PY{l+s+s1}{\PYZsq{}}\PY{p}{)}
          \PY{n}{nx}\PY{o}{.}\PY{n}{draw}\PY{p}{(}\PY{n}{dolph}\PY{p}{,} \PY{n}{ax} \PY{o}{=} \PY{n}{axes}\PY{p}{[}\PY{l+m+mi}{6}\PY{p}{]}\PY{p}{,} \PY{n}{node\PYZus{}size} \PY{o}{=} \PY{n}{ns}\PY{p}{,} \PY{n}{node\PYZus{}color}\PY{o}{=}\PY{n}{colores}\PY{p}{,}
                  \PY{n}{pos}\PY{o}{=}\PY{n}{multi\PYZus{}random\PYZus{}pos}\PY{p}{)}
          \PY{n}{axes}\PY{p}{[}\PY{l+m+mi}{6}\PY{p}{]}\PY{o}{.}\PY{n}{set\PYZus{}title}\PY{p}{(}\PY{l+s+s1}{\PYZsq{}}\PY{l+s+s1}{Posicionamiento multipartito al azar}\PY{l+s+s1}{\PYZsq{}}\PY{p}{)}
          
          
          \PY{c+c1}{\PYZsh{} Posicionamiento multipartito espectral. Posiciono por espectro y}
          \PY{c+c1}{\PYZsh{} luego desplazo lateralmente según género}
          \PY{n}{multi\PYZus{}spectral\PYZus{}pos} \PY{o}{=} \PY{n}{position\PYZus{}multipartito\PYZus{}spectral}\PY{p}{(}\PY{n}{dolph}\PY{p}{,}
                                                              \PY{p}{[}\PY{l+s+s1}{\PYZsq{}}\PY{l+s+s1}{f}\PY{l+s+s1}{\PYZsq{}}\PY{p}{,} \PY{l+s+s1}{\PYZsq{}}\PY{l+s+s1}{m}\PY{l+s+s1}{\PYZsq{}}\PY{p}{,} \PY{l+s+s1}{\PYZsq{}}\PY{l+s+s1}{NA}\PY{l+s+s1}{\PYZsq{}}\PY{p}{]}\PY{p}{,} \PY{l+s+s1}{\PYZsq{}}\PY{l+s+s1}{gender}\PY{l+s+s1}{\PYZsq{}}\PY{p}{,} \PY{n}{dhorizontal}\PY{o}{=}\PY{l+m+mf}{0.5}\PY{p}{)}
          \PY{n}{nx}\PY{o}{.}\PY{n}{draw}\PY{p}{(}\PY{n}{dolph}\PY{p}{,} \PY{n}{ax} \PY{o}{=} \PY{n}{axes}\PY{p}{[}\PY{l+m+mi}{7}\PY{p}{]}\PY{p}{,} \PY{n}{node\PYZus{}size} \PY{o}{=} \PY{n}{ns}\PY{p}{,} \PY{n}{node\PYZus{}color}\PY{o}{=}\PY{n}{colores}\PY{p}{,}
                  \PY{n}{pos}\PY{o}{=}\PY{n}{multi\PYZus{}spectral\PYZus{}pos}\PY{p}{)}
          \PY{n}{axes}\PY{p}{[}\PY{l+m+mi}{7}\PY{p}{]}\PY{o}{.}\PY{n}{set\PYZus{}title}\PY{p}{(}\PY{l+s+s1}{\PYZsq{}}\PY{l+s+s1}{Posicionamiento multipartito espectral}\PY{l+s+s1}{\PYZsq{}}\PY{p}{)}
          
          \PY{c+c1}{\PYZsh{} Posicionamiento multipartito por resortes. Posiciono por resortes y}
          \PY{c+c1}{\PYZsh{} luego desplazo lateralmente según género}
          \PY{n}{multi\PYZus{}spring\PYZus{}pos} \PY{o}{=} \PY{n}{position\PYZus{}multipartito\PYZus{}spring}\PY{p}{(}\PY{n}{dolph}\PY{p}{,} \PY{p}{[}\PY{l+s+s1}{\PYZsq{}}\PY{l+s+s1}{f}\PY{l+s+s1}{\PYZsq{}}\PY{p}{,} \PY{l+s+s1}{\PYZsq{}}\PY{l+s+s1}{m}\PY{l+s+s1}{\PYZsq{}}\PY{p}{,} \PY{l+s+s1}{\PYZsq{}}\PY{l+s+s1}{NA}\PY{l+s+s1}{\PYZsq{}}\PY{p}{]}\PY{p}{,}
                                                          \PY{l+s+s1}{\PYZsq{}}\PY{l+s+s1}{gender}\PY{l+s+s1}{\PYZsq{}}\PY{p}{,} \PY{n}{dhorizontal}\PY{o}{=}\PY{l+m+mi}{2}\PY{p}{)}
          \PY{n}{nx}\PY{o}{.}\PY{n}{draw}\PY{p}{(}\PY{n}{dolph}\PY{p}{,} \PY{n}{ax} \PY{o}{=} \PY{n}{axes}\PY{p}{[}\PY{l+m+mi}{8}\PY{p}{]}\PY{p}{,} \PY{n}{node\PYZus{}size} \PY{o}{=} \PY{n}{ns}\PY{p}{,} \PY{n}{node\PYZus{}color}\PY{o}{=}\PY{n}{colores}\PY{p}{,}
                  \PY{n}{pos}\PY{o}{=}\PY{n}{multi\PYZus{}spring\PYZus{}pos}\PY{p}{)}
          \PY{n}{axes}\PY{p}{[}\PY{l+m+mi}{8}\PY{p}{]}\PY{o}{.}\PY{n}{set\PYZus{}title}\PY{p}{(}\PY{l+s+s1}{\PYZsq{}}\PY{l+s+s1}{Posicionamiento multipartito por resortes}\PY{l+s+s1}{\PYZsq{}}\PY{p}{)}
          
          \PY{n}{fig}\PY{o}{.}\PY{n}{tight\PYZus{}layout}\PY{p}{(}\PY{p}{)}
          \PY{n}{plt}\PY{o}{.}\PY{n}{show}\PY{p}{(}\PY{p}{)}
\end{Verbatim}


    \begin{center}
    \adjustimage{max size={0.9\linewidth}{0.9\paperheight}}{output_59_0.png}
    \end{center}
    { \hspace*{\fill} \\}
    
    En todas las visualizaciones, el color celeste indica delfín hembra y el
colón rojo indica delfín macho. El color verde indica ausencia de
información.

\begin{itemize}
\tightlist
\item
  La visualización 1 ("layout multipartito") no es para nada adecuada en
  nuestro caso debido a que nuestra red no es multipartita, y no es
  posible ver claramente los links entre nodos de un mismo grupo.
\item
  Las visualizaciones circulares tampoco resultan particularmente
  informativas, a pesar de que en los círculos concéntricos se
  distinguen claramente los diferentes grupos. Pasa algo parecido a lo
  que ocurre con el "layout multipartito".
\item
  Entre las visualizaciones al azar, espectral y por resortes, la
  visualización por resortes parece ser la más clara para observar qué
  subgrupos de nodos se conectan más con cuáles. La visualización
  espectral se ve interesante pero no sabemos interpretarla.
\item
  En la visualización por resortes, se observa que existen dos grupos de
  delfines que interactuan mayormente entre sí, uno de los cuales es
  transversal a la división por género.
\item
  En las últimas 3 visualizaciones, realizamos un desplazamiento lateral
  de los nodos según su información de género (ver archivo
  \texttt{graficar\_multipartito.py}). En el caso de la visualización
  por resortes, esto permite observar mejor lo dicho anteriormente:
  puede notarse que un cluster tiene aproximadamente la misma cantidad
  de delfines hembras que machos, mientras que el otro está compuesto
  principalmente por machos.
\end{itemize}

    \subsection{Punto b}\label{punto-b}

    Primero vamos a quedarnos con un subgrafo del original, en donde
elminamos los nodos con genero no definido.

    \begin{Verbatim}[commandchars=\\\{\}]
{\color{incolor}In [{\color{incolor}131}]:} \PY{n}{delfines\PYZus{}con\PYZus{}info} \PY{o}{=} \PY{p}{[}\PY{n}{d} \PY{k}{for} \PY{n}{d} \PY{o+ow}{in} \PY{n}{dolph}\PY{o}{.}\PY{n}{nodes}\PY{p}{(}\PY{p}{)} \PY{k}{if} \PY{n}{d} \PY{o+ow}{not} \PY{o+ow}{in} \PY{n}{particiones}\PY{p}{[}\PY{l+m+mi}{1}\PY{p}{]}\PY{p}{]}
          \PY{n}{dolph2} \PY{o}{=} \PY{n}{nx}\PY{o}{.}\PY{n}{subgraph}\PY{p}{(}\PY{n}{dolph}\PY{p}{,} \PY{n}{delfines\PYZus{}con\PYZus{}info}\PY{p}{)}\PY{o}{.}\PY{n}{copy}\PY{p}{(}\PY{p}{)}
\end{Verbatim}


    Lo que hacemos en este punto es generar mediante Monte Carlo una
aproximación a la distribución de probabilidad del número de enlaces
entre delfines de distinto género bajo la hipótesis nula de que la
topología y la información de género son independientes. Para ello
generamos 100.000 variantes de la red observada, redistribuyendo
aleatoriamente los géneros de los nodos según una distribución uniforme.

    \begin{Verbatim}[commandchars=\\\{\}]
{\color{incolor}In [{\color{incolor}142}]:} \PY{n}{n\PYZus{}simulaciones} \PY{o}{=} \PY{n+nb}{int}\PY{p}{(}\PY{l+m+mf}{1e5}\PY{p}{)}
          \PY{n}{enlaces\PYZus{}entre\PYZus{}grupos} \PY{o}{=} \PY{n}{np}\PY{o}{.}\PY{n}{zeros}\PY{p}{(}\PY{p}{(}\PY{n}{n\PYZus{}simulaciones}\PY{p}{)}\PY{p}{)}
          \PY{n}{grafo\PYZus{}h0} \PY{o}{=} \PY{n}{dolph2}\PY{o}{.}\PY{n}{copy}\PY{p}{(}\PY{p}{)}
          \PY{c+c1}{\PYZsh{} Vamos a ir modificando este grafo \PYZdq{}in place\PYZdq{} (no lo clonamos n veces)}
          \PY{k}{for} \PY{n}{i} \PY{o+ow}{in} \PY{n+nb}{range}\PY{p}{(}\PY{n}{n\PYZus{}simulaciones}\PY{p}{)}\PY{p}{:}
              \PY{c+c1}{\PYZsh{} Mezclamos la lista de nombres de delfines.}
              \PY{c+c1}{\PYZsh{} A los primeros 24 delfines les reasignamos género hembra}
              \PY{c+c1}{\PYZsh{} El resto van a ser macho.}
              \PY{n}{hembras} \PY{o}{=} \PY{n}{sample}\PY{p}{(}\PY{n+nb}{list}\PY{p}{(}\PY{n}{grafo\PYZus{}h0}\PY{o}{.}\PY{n}{nodes}\PY{p}{(}\PY{p}{)}\PY{p}{)}\PY{p}{,} \PY{l+m+mi}{24}\PY{p}{)}
              \PY{k}{for} \PY{n}{nombre} \PY{o+ow}{in} \PY{n}{grafo\PYZus{}h0}\PY{o}{.}\PY{n}{nodes}\PY{p}{(}\PY{p}{)}\PY{p}{:}
                  \PY{k}{if} \PY{n}{nombre} \PY{o+ow}{in} \PY{n}{hembras}\PY{p}{:}
                      \PY{n}{grafo\PYZus{}h0}\PY{o}{.}\PY{n}{nodes}\PY{p}{(}\PY{p}{)}\PY{p}{[}\PY{n}{nombre}\PY{p}{]}\PY{p}{[}\PY{l+s+s1}{\PYZsq{}}\PY{l+s+s1}{gender}\PY{l+s+s1}{\PYZsq{}}\PY{p}{]} \PY{o}{=} \PY{l+s+s1}{\PYZsq{}}\PY{l+s+s1}{f}\PY{l+s+s1}{\PYZsq{}}
                  \PY{k}{else}\PY{p}{:}
                      \PY{n}{grafo\PYZus{}h0}\PY{o}{.}\PY{n}{nodes}\PY{p}{(}\PY{p}{)}\PY{p}{[}\PY{n}{nombre}\PY{p}{]}\PY{p}{[}\PY{l+s+s1}{\PYZsq{}}\PY{l+s+s1}{gender}\PY{l+s+s1}{\PYZsq{}}\PY{p}{]} \PY{o}{=} \PY{l+s+s1}{\PYZsq{}}\PY{l+s+s1}{m}\PY{l+s+s1}{\PYZsq{}}
              \PY{n}{enlaces\PYZus{}entre\PYZus{}grupos}\PY{p}{[}\PY{n}{i}\PY{p}{]} \PY{o}{=} \PY{n}{contar\PYZus{}enlaces\PYZus{}entre\PYZus{}grupos}\PY{p}{(}\PY{n}{grafo\PYZus{}h0}\PY{p}{,} \PY{l+s+s1}{\PYZsq{}}\PY{l+s+s1}{gender}\PY{l+s+s1}{\PYZsq{}}\PY{p}{)}
\end{Verbatim}


    Visualizamos la distribución de enlaces para las redes simuladas,
marcando el numero total de enlaces de la red original.

    \begin{Verbatim}[commandchars=\\\{\}]
{\color{incolor}In [{\color{incolor}143}]:} \PY{n}{valor\PYZus{}observado} \PY{o}{=} \PY{n}{contar\PYZus{}enlaces\PYZus{}entre\PYZus{}grupos}\PY{p}{(}\PY{n}{dolph2}\PY{p}{,} \PY{l+s+s1}{\PYZsq{}}\PY{l+s+s1}{gender}\PY{l+s+s1}{\PYZsq{}}\PY{p}{)}
          \PY{n}{fig}\PY{p}{,} \PY{n}{ax} \PY{o}{=} \PY{n}{plt}\PY{o}{.}\PY{n}{subplots}\PY{p}{(}\PY{n}{figsize}\PY{o}{=}\PY{p}{(}\PY{l+m+mi}{12}\PY{p}{,} \PY{l+m+mi}{6}\PY{p}{)}\PY{p}{)}
          \PY{n}{ax}\PY{o}{.}\PY{n}{axvline}\PY{p}{(}\PY{n}{valor\PYZus{}observado}\PY{p}{,} \PY{n}{color}\PY{o}{=}\PY{l+s+s1}{\PYZsq{}}\PY{l+s+s1}{deeppink}\PY{l+s+s1}{\PYZsq{}}\PY{p}{,}
                     \PY{n}{label}\PY{o}{=}\PY{l+s+s1}{\PYZsq{}}\PY{l+s+s1}{Valor observado = \PYZob{}\PYZcb{}}\PY{l+s+s1}{\PYZsq{}}\PY{o}{.}\PY{n}{format}\PY{p}{(}\PY{n}{valor\PYZus{}observado}\PY{p}{)}\PY{p}{)}
          \PY{n}{ax}\PY{o}{.}\PY{n}{legend}\PY{p}{(}\PY{n}{fontsize}\PY{o}{=}\PY{l+m+mi}{18}\PY{p}{)}
          \PY{n}{histograma}\PY{p}{(}\PY{n}{enlaces\PYZus{}entre\PYZus{}grupos}\PY{p}{,} \PY{n}{bins}\PY{o}{=}\PY{l+m+mi}{54}\PY{p}{,} \PY{n}{density}\PY{o}{=}\PY{n+nb+bp}{True}\PY{p}{,} \PY{n}{ax}\PY{o}{=}\PY{n}{ax}\PY{p}{,}
                     \PY{n}{titulo}\PY{o}{=}\PY{l+s+sa}{r}\PY{l+s+s1}{\PYZsq{}}\PY{l+s+s1}{Distribución de enlaces entre delfines de géneros distintos bajo \PYZdl{}H\PYZus{}0\PYZdl{}}\PY{l+s+s1}{\PYZsq{}}\PY{p}{,}
                     \PY{n}{xlabel}\PY{o}{=}\PY{l+s+s1}{\PYZsq{}}\PY{l+s+s1}{\PYZsh{} de enlaces}\PY{l+s+s1}{\PYZsq{}}\PY{p}{)}
          \PY{n}{plt}\PY{o}{.}\PY{n}{show}\PY{p}{(}\PY{p}{)}
\end{Verbatim}


    \begin{center}
    \adjustimage{max size={0.9\linewidth}{0.9\paperheight}}{output_67_0.png}
    \end{center}
    { \hspace*{\fill} \\}
    
    Vamos a calcular el p-value. Para eso, integramos el histograma a partir
del valor real hasta menos infinito. Como la distribucion es acampanada,
duplicamos el valor obtenido para contemplar los valores que se
encuentran del otro lado de la distribucion con probabilidad menor que
la de del valor real del numero de enlaces.

    \begin{Verbatim}[commandchars=\\\{\}]
{\color{incolor}In [{\color{incolor}144}]:} \PY{n}{p\PYZus{}value}\PY{p}{(}\PY{n}{enlaces\PYZus{}entre\PYZus{}grupos}\PY{p}{,} \PY{l+m+mi}{52}\PY{p}{,} \PY{l+s+s1}{\PYZsq{}}\PY{l+s+s1}{izq}\PY{l+s+s1}{\PYZsq{}}\PY{p}{)}
\end{Verbatim}


\begin{Verbatim}[commandchars=\\\{\}]
{\color{outcolor}Out[{\color{outcolor}144}]:} 0.00107
\end{Verbatim}
            
    Esto quiere decir que, si \(H_0\) fuera correcta, la probabilidad de
haber observado lo que observamos es del 0,1\% aproximadamente. Por lo
tanto podemos concluir que la topología y los géneros sí están
relacionadas.

    Finalmente, si bien esto no añade realmente más información, construimos
también la distribución de modularidades bajo \(H_0\) y la visualizamos.
Como calcular modularidades toma más tiempo, para esto realizamos una
simulación con menos historias.

    \begin{Verbatim}[commandchars=\\\{\}]
{\color{incolor}In [{\color{incolor}145}]:} \PY{n}{n\PYZus{}simulaciones} \PY{o}{=} \PY{l+m+mi}{1000}
          \PY{n}{modularidades} \PY{o}{=} \PY{n}{np}\PY{o}{.}\PY{n}{zeros}\PY{p}{(}\PY{p}{(}\PY{n}{n\PYZus{}simulaciones}\PY{p}{)}\PY{p}{)}
          \PY{n}{grafo\PYZus{}h0} \PY{o}{=} \PY{n}{dolph2}\PY{o}{.}\PY{n}{copy}\PY{p}{(}\PY{p}{)}
          \PY{c+c1}{\PYZsh{} Vamos a ir modificando este grafo \PYZdq{}in place\PYZdq{} (no lo clonamos n veces)}
          \PY{k}{for} \PY{n}{i} \PY{o+ow}{in} \PY{n+nb}{range}\PY{p}{(}\PY{n}{n\PYZus{}simulaciones}\PY{p}{)}\PY{p}{:}
              \PY{c+c1}{\PYZsh{} Mezclamos la lista de nombres de delfines.}
              \PY{c+c1}{\PYZsh{} A los primeros 24 delfines les reasignamos género hembra}
              \PY{c+c1}{\PYZsh{} El resto van a ser macho.}
              \PY{n}{hembras} \PY{o}{=} \PY{n}{sample}\PY{p}{(}\PY{n+nb}{list}\PY{p}{(}\PY{n}{grafo\PYZus{}h0}\PY{o}{.}\PY{n}{nodes}\PY{p}{(}\PY{p}{)}\PY{p}{)}\PY{p}{,} \PY{l+m+mi}{24}\PY{p}{)}
              \PY{k}{for} \PY{n}{nombre} \PY{o+ow}{in} \PY{n}{grafo\PYZus{}h0}\PY{o}{.}\PY{n}{nodes}\PY{p}{(}\PY{p}{)}\PY{p}{:}
                  \PY{k}{if} \PY{n}{nombre} \PY{o+ow}{in} \PY{n}{hembras}\PY{p}{:}
                      \PY{n}{grafo\PYZus{}h0}\PY{o}{.}\PY{n}{nodes}\PY{p}{(}\PY{p}{)}\PY{p}{[}\PY{n}{nombre}\PY{p}{]}\PY{p}{[}\PY{l+s+s1}{\PYZsq{}}\PY{l+s+s1}{gender}\PY{l+s+s1}{\PYZsq{}}\PY{p}{]} \PY{o}{=} \PY{l+s+s1}{\PYZsq{}}\PY{l+s+s1}{f}\PY{l+s+s1}{\PYZsq{}}
                  \PY{k}{else}\PY{p}{:}
                      \PY{n}{grafo\PYZus{}h0}\PY{o}{.}\PY{n}{nodes}\PY{p}{(}\PY{p}{)}\PY{p}{[}\PY{n}{nombre}\PY{p}{]}\PY{p}{[}\PY{l+s+s1}{\PYZsq{}}\PY{l+s+s1}{gender}\PY{l+s+s1}{\PYZsq{}}\PY{p}{]} \PY{o}{=} \PY{l+s+s1}{\PYZsq{}}\PY{l+s+s1}{m}\PY{l+s+s1}{\PYZsq{}}
              \PY{n}{modularidades}\PY{p}{[}\PY{n}{i}\PY{p}{]} \PY{o}{=} \PY{n}{modularidad}\PY{p}{(}\PY{n}{grafo\PYZus{}h0}\PY{p}{,} \PY{l+s+s1}{\PYZsq{}}\PY{l+s+s1}{gender}\PY{l+s+s1}{\PYZsq{}}\PY{p}{)}
\end{Verbatim}


    \begin{Verbatim}[commandchars=\\\{\}]
{\color{incolor}In [{\color{incolor}147}]:} \PY{n}{modularidad\PYZus{}observada} \PY{o}{=} \PY{n}{modularidad}\PY{p}{(}\PY{n}{dolph2}\PY{p}{,} \PY{l+s+s1}{\PYZsq{}}\PY{l+s+s1}{gender}\PY{l+s+s1}{\PYZsq{}}\PY{p}{)}
          \PY{n}{fig}\PY{p}{,} \PY{n}{ax} \PY{o}{=} \PY{n}{plt}\PY{o}{.}\PY{n}{subplots}\PY{p}{(}\PY{n}{figsize}\PY{o}{=}\PY{p}{(}\PY{l+m+mi}{12}\PY{p}{,} \PY{l+m+mi}{6}\PY{p}{)}\PY{p}{)}
          \PY{n}{ax}\PY{o}{.}\PY{n}{axvline}\PY{p}{(}\PY{n}{modularidad\PYZus{}observada}\PY{p}{,} \PY{n}{color}\PY{o}{=}\PY{l+s+s1}{\PYZsq{}}\PY{l+s+s1}{deeppink}\PY{l+s+s1}{\PYZsq{}}\PY{p}{,}
                     \PY{n}{label}\PY{o}{=}\PY{l+s+s1}{\PYZsq{}}\PY{l+s+s1}{Valor observado = \PYZob{}\PYZcb{}}\PY{l+s+s1}{\PYZsq{}}\PY{o}{.}\PY{n}{format}\PY{p}{(}\PY{n}{modularidad\PYZus{}observada}\PY{p}{)}\PY{p}{)}
          \PY{n}{ax}\PY{o}{.}\PY{n}{legend}\PY{p}{(}\PY{n}{fontsize}\PY{o}{=}\PY{l+m+mi}{18}\PY{p}{)}
          \PY{n}{histograma}\PY{p}{(}\PY{n}{modularidades}\PY{p}{,} \PY{n}{bins}\PY{o}{=}\PY{l+m+mi}{15}\PY{p}{,} \PY{n}{density}\PY{o}{=}\PY{n+nb+bp}{True}\PY{p}{,} \PY{n}{ax}\PY{o}{=}\PY{n}{ax}\PY{p}{,}
                     \PY{n}{titulo}\PY{o}{=}\PY{l+s+sa}{r}\PY{l+s+s1}{\PYZsq{}}\PY{l+s+s1}{Distribución de modularidad bajo \PYZdl{}H\PYZus{}0\PYZdl{}}\PY{l+s+s1}{\PYZsq{}}\PY{p}{,}
                     \PY{n}{xlabel}\PY{o}{=}\PY{l+s+s1}{\PYZsq{}}\PY{l+s+s1}{Modularidad}\PY{l+s+s1}{\PYZsq{}}\PY{p}{)}
          \PY{n}{plt}\PY{o}{.}\PY{n}{show}\PY{p}{(}\PY{p}{)}
\end{Verbatim}


    \begin{center}
    \adjustimage{max size={0.9\linewidth}{0.9\paperheight}}{output_73_0.png}
    \end{center}
    { \hspace*{\fill} \\}
    
    \subsubsection{Punto c)}\label{punto-c}

    Primero defino funciones que me sacan porcentajes de nodos de la red
siguiendo un criterio especifico, con el objetivo de desarmar la red de
la forma mas eficiente. El criterio es el siguiente: la mejor forma de
desarmar una componente gigante es empezando por remover los puentes.
Por otro lado, los puentes son aquellos nodos que unen secciones del
grafo que no estan conectadas entre si (por lo menos por algun camino de
largo del orden del camino del puente). Es decir, los nodos puentes
tienen coeficiente de clustering chico comparado con otros nodos del
grafo, ya que sus vecinos no estan conectados entre si. Por lo tanto, la
funcion desarme\_de\_red() saca un porcentaje de nodos x\% que tienen el
menor coeficiente de clustering.

    \begin{Verbatim}[commandchars=\\\{\}]
{\color{incolor}In [{\color{incolor} }]:} \PY{k}{def} \PY{n+nf}{desarme}\PY{p}{(}\PY{n}{g}\PY{p}{,} \PY{n}{porcentaje\PYZus{}nodos\PYZus{}sacados}\PY{p}{)}\PY{p}{:}
            \PY{n}{n}\PY{o}{=} \PY{n+nb}{int}\PY{p}{(}\PY{n+nb}{len}\PY{p}{(}\PY{n}{g}\PY{o}{.}\PY{n}{nodes}\PY{o}{.}\PY{n}{keys}\PY{p}{(}\PY{p}{)}\PY{p}{)} \PY{o}{*} \PY{n}{porcentaje\PYZus{}nodos\PYZus{}sacados}\PY{p}{)}
            \PY{n}{G} \PY{o}{=} \PY{n}{g}\PY{o}{.}\PY{n}{copy}\PY{p}{(}\PY{p}{)}
            \PY{n}{nodes} \PY{o}{=} \PY{p}{[}\PY{p}{]}
            \PY{n}{clustering} \PY{o}{=} \PY{p}{[}\PY{p}{]}
            \PY{k}{for} \PY{n}{a} \PY{o+ow}{in} \PY{n}{G}\PY{o}{.}\PY{n}{nodes}\PY{o}{.}\PY{n}{keys}\PY{p}{(}\PY{p}{)}\PY{p}{:}
                \PY{k}{if} \PY{n}{G}\PY{o}{.}\PY{n}{degree}\PY{p}{(}\PY{n}{a}\PY{p}{)}\PY{o}{\PYZgt{}}\PY{l+m+mi}{1}\PY{p}{:}
                    \PY{n}{nodes}\PY{o}{.}\PY{n}{append}\PY{p}{(}\PY{n}{a}\PY{p}{)}
                    \PY{n}{clustering}\PY{o}{.}\PY{n}{append}\PY{p}{(}\PY{n}{nx}\PY{o}{.}\PY{n}{clustering}\PY{p}{(}\PY{n}{G}\PY{p}{)}\PY{p}{[}\PY{n}{a}\PY{p}{]}\PY{p}{)}
            \PY{k}{for} \PY{n}{i} \PY{o+ow}{in} \PY{n+nb}{range}\PY{p}{(}\PY{n}{n}\PY{p}{)}\PY{p}{:}
                \PY{n}{j} \PY{o}{=} \PY{n}{clustering}\PY{o}{.}\PY{n}{index}\PY{p}{(}\PY{n+nb}{min}\PY{p}{(}\PY{n}{clustering}\PY{p}{)}\PY{p}{)}
                \PY{n}{G}\PY{o}{.}\PY{n}{remove\PYZus{}node}\PY{p}{(}\PY{n}{nodes}\PY{p}{[}\PY{n}{j}\PY{p}{]}\PY{p}{)}
                \PY{n}{clustering}\PY{o}{.}\PY{n}{pop}\PY{p}{(}\PY{n}{j}\PY{p}{)}
                \PY{n}{nodes}\PY{o}{.}\PY{n}{pop}\PY{p}{(}\PY{n}{j}\PY{p}{)}
            \PY{k}{return} \PY{n}{G}
        
        
        \PY{k}{def} \PY{n+nf}{cociente}\PY{p}{(}\PY{n}{g}\PY{p}{)}\PY{p}{:}
            \PY{l+s+sd}{\PYZdq{}\PYZdq{}\PYZdq{}Agarra una red G y va sacando una cierta proporcion de nodos hasta que}
        \PY{l+s+sd}{    el cociente entre la cant de nodos de la componente gigante y la segunda }
        \PY{l+s+sd}{    comoponente sea cercano a 1.\PYZdq{}\PYZdq{}\PYZdq{}}
            \PY{n}{rango\PYZus{}porcentajes} \PY{o}{=} \PY{n}{np}\PY{o}{.}\PY{n}{arange}\PY{p}{(}\PY{l+m+mi}{0}\PY{p}{,} \PY{l+m+mf}{0.85}\PY{p}{,} \PY{l+m+mi}{1}\PY{o}{/}\PY{n}{g}\PY{o}{.}\PY{n}{order}\PY{p}{(}\PY{p}{)}\PY{p}{)}
            \PY{n}{d} \PY{o}{=} \PY{p}{[}\PY{p}{]}
            \PY{k}{for} \PY{n}{i} \PY{o+ow}{in} \PY{n}{rango\PYZus{}porcentajes}\PY{p}{:}
                \PY{n}{g\PYZus{}desarmado} \PY{o}{=} \PY{n}{desarme}\PY{p}{(}\PY{n}{g}\PY{p}{,} \PY{n}{i}\PY{p}{)}
                \PY{n}{lengths} \PY{o}{=} \PY{p}{[}\PY{n+nb}{len}\PY{p}{(}\PY{n}{j}\PY{p}{)} \PY{k}{for} \PY{n}{j} \PY{o+ow}{in} \PY{n+nb}{sorted}\PY{p}{(}\PY{n}{nx}\PY{o}{.}\PY{n}{connected\PYZus{}components}\PY{p}{(}\PY{n}{g\PYZus{}desarmado}\PY{p}{)}\PY{p}{,} \PY{n}{key}\PY{o}{=}\PY{n+nb}{len}\PY{p}{,} \PY{n}{reverse}\PY{o}{=}\PY{n+nb+bp}{True}\PY{p}{)}\PY{p}{]}
                \PY{k}{if} \PY{n+nb}{len}\PY{p}{(}\PY{n}{lengths}\PY{p}{)}\PY{o}{\PYZgt{}}\PY{o}{=}\PY{l+m+mi}{2}\PY{p}{:}
                    \PY{n}{d}\PY{o}{.}\PY{n}{append}\PY{p}{(}\PY{n}{lengths}\PY{p}{[}\PY{l+m+mi}{0}\PY{p}{]}\PY{o}{/}\PY{n}{lengths}\PY{p}{[}\PY{l+m+mi}{1}\PY{p}{]}\PY{p}{)}
            \PY{n}{d2} \PY{o}{=} \PY{n}{np}\PY{o}{.}\PY{n}{asarray}\PY{p}{(}\PY{n}{d}\PY{p}{)}
            \PY{n}{d2} \PY{o}{=} \PY{n+nb}{abs}\PY{p}{(}\PY{n}{d2}\PY{o}{\PYZhy{}}\PY{l+m+mi}{1}\PY{p}{)}
            \PY{n}{d2} \PY{o}{=} \PY{n+nb}{list}\PY{p}{(}\PY{n}{d2}\PY{p}{)}
            \PY{n}{index} \PY{o}{=} \PY{n}{d2}\PY{o}{.}\PY{n}{index}\PY{p}{(}\PY{n+nb}{min}\PY{p}{(}\PY{n}{d2}\PY{p}{)}\PY{p}{)}
            \PY{n}{porcentaje} \PY{o}{=} \PY{n}{rango\PYZus{}porcentajes}\PY{p}{[}\PY{n}{index}\PY{p}{]}
            \PY{k}{return} \PY{n}{porcentaje}\PY{p}{,} \PY{n+nb}{min}\PY{p}{(}\PY{n}{d}\PY{p}{)}
        
        \PY{n}{porcentaje} \PY{o}{=} \PY{n}{np}\PY{o}{.}\PY{n}{linspace}\PY{p}{(}\PY{l+m+mi}{0}\PY{p}{,} \PY{l+m+mf}{0.8}\PY{p}{,} \PY{l+m+mi}{9}\PY{p}{)}
            
        \PY{n}{f}\PY{p}{,} \PY{p}{(}\PY{p}{[}\PY{n}{ax1}\PY{p}{,} \PY{n}{ax2}\PY{p}{,} \PY{n}{ax3}\PY{p}{]}\PY{p}{,} \PY{p}{[}\PY{n}{ax4}\PY{p}{,} \PY{n}{ax5}\PY{p}{,} \PY{n}{ax6}\PY{p}{]}\PY{p}{,} \PY{p}{[}\PY{n}{ax7}\PY{p}{,} \PY{n}{ax8}\PY{p}{,} \PY{n}{ax9}\PY{p}{]}\PY{p}{)} \PY{o}{=} \PY{n}{plt}\PY{o}{.}\PY{n}{subplots}\PY{p}{(}\PY{l+m+mi}{3}\PY{p}{,} \PY{l+m+mi}{3}\PY{p}{)}
        \PY{c+c1}{\PYZsh{}f.tight\PYZus{}layout()}
        \PY{n}{f}\PY{o}{.}\PY{n}{suptitle}\PY{p}{(}\PY{l+s+s1}{\PYZsq{}}\PY{l+s+s1}{Desarme de la red en funcion de }\PY{l+s+s1}{\PYZpc{}}\PY{l+s+s1}{ nodos sacados.}\PY{l+s+s1}{\PYZsq{}}\PY{p}{)}
        \PY{n}{plt}\PY{o}{.}\PY{n}{sca}\PY{p}{(}\PY{n}{ax1}\PY{p}{)}
        \PY{n}{ax1}\PY{o}{.}\PY{n}{set\PYZus{}title}\PY{p}{(}\PY{l+s+s1}{\PYZsq{}}\PY{l+s+s1}{0}\PY{l+s+s1}{\PYZpc{}}\PY{l+s+s1}{\PYZsq{}}\PY{p}{)}
        \PY{n}{nx}\PY{o}{.}\PY{n}{draw}\PY{p}{(}\PY{n}{desarme}\PY{p}{(}\PY{n}{dolph}\PY{p}{,} \PY{n}{porcentaje}\PY{p}{[}\PY{l+m+mi}{0}\PY{p}{]}\PY{p}{)}\PY{p}{,} \PY{n}{node\PYZus{}size}\PY{o}{=}\PY{l+m+mi}{15}\PY{p}{)}
        \PY{n}{plt}\PY{o}{.}\PY{n}{sca}\PY{p}{(}\PY{n}{ax2}\PY{p}{)}
        \PY{n}{ax2}\PY{o}{.}\PY{n}{set\PYZus{}title}\PY{p}{(}\PY{l+s+s1}{\PYZsq{}}\PY{l+s+s1}{10}\PY{l+s+s1}{\PYZpc{}}\PY{l+s+s1}{\PYZsq{}}\PY{p}{)}
        \PY{n}{nx}\PY{o}{.}\PY{n}{draw}\PY{p}{(}\PY{n}{desarme}\PY{p}{(}\PY{n}{dolph}\PY{p}{,} \PY{n}{porcentaje}\PY{p}{[}\PY{l+m+mi}{1}\PY{p}{]}\PY{p}{)}\PY{p}{,} \PY{n}{node\PYZus{}size}\PY{o}{=}\PY{l+m+mi}{15}\PY{p}{)}
        \PY{n}{plt}\PY{o}{.}\PY{n}{sca}\PY{p}{(}\PY{n}{ax3}\PY{p}{)}
        \PY{n}{ax3}\PY{o}{.}\PY{n}{set\PYZus{}title}\PY{p}{(}\PY{l+s+s1}{\PYZsq{}}\PY{l+s+s1}{20}\PY{l+s+s1}{\PYZpc{}}\PY{l+s+s1}{\PYZsq{}}\PY{p}{)}
        \PY{n}{nx}\PY{o}{.}\PY{n}{draw}\PY{p}{(}\PY{n}{desarme}\PY{p}{(}\PY{n}{dolph}\PY{p}{,} \PY{n}{porcentaje}\PY{p}{[}\PY{l+m+mi}{2}\PY{p}{]}\PY{p}{)}\PY{p}{,} \PY{n}{node\PYZus{}size}\PY{o}{=}\PY{l+m+mi}{15}\PY{p}{)}
        \PY{n}{plt}\PY{o}{.}\PY{n}{sca}\PY{p}{(}\PY{n}{ax4}\PY{p}{)}
        \PY{n}{ax4}\PY{o}{.}\PY{n}{set\PYZus{}title}\PY{p}{(}\PY{l+s+s1}{\PYZsq{}}\PY{l+s+s1}{30}\PY{l+s+s1}{\PYZpc{}}\PY{l+s+s1}{\PYZsq{}}\PY{p}{)}
        \PY{n}{nx}\PY{o}{.}\PY{n}{draw}\PY{p}{(}\PY{n}{desarme}\PY{p}{(}\PY{n}{dolph}\PY{p}{,} \PY{n}{porcentaje}\PY{p}{[}\PY{l+m+mi}{3}\PY{p}{]}\PY{p}{)}\PY{p}{,} \PY{n}{node\PYZus{}size}\PY{o}{=}\PY{l+m+mi}{15}\PY{p}{)}
        \PY{n}{plt}\PY{o}{.}\PY{n}{sca}\PY{p}{(}\PY{n}{ax5}\PY{p}{)}
        \PY{n}{ax5}\PY{o}{.}\PY{n}{set\PYZus{}title}\PY{p}{(}\PY{l+s+s1}{\PYZsq{}}\PY{l+s+s1}{40}\PY{l+s+s1}{\PYZpc{}}\PY{l+s+s1}{\PYZsq{}}\PY{p}{)}
        \PY{n}{nx}\PY{o}{.}\PY{n}{draw}\PY{p}{(}\PY{n}{desarme}\PY{p}{(}\PY{n}{dolph}\PY{p}{,} \PY{n}{porcentaje}\PY{p}{[}\PY{l+m+mi}{4}\PY{p}{]}\PY{p}{)}\PY{p}{,} \PY{n}{node\PYZus{}size}\PY{o}{=}\PY{l+m+mi}{15}\PY{p}{)}
        \PY{n}{plt}\PY{o}{.}\PY{n}{sca}\PY{p}{(}\PY{n}{ax6}\PY{p}{)}
        \PY{n}{ax6}\PY{o}{.}\PY{n}{set\PYZus{}title}\PY{p}{(}\PY{l+s+s1}{\PYZsq{}}\PY{l+s+s1}{50}\PY{l+s+s1}{\PYZpc{}}\PY{l+s+s1}{\PYZsq{}}\PY{p}{)}
        \PY{n}{nx}\PY{o}{.}\PY{n}{draw}\PY{p}{(}\PY{n}{desarme}\PY{p}{(}\PY{n}{dolph}\PY{p}{,} \PY{n}{porcentaje}\PY{p}{[}\PY{l+m+mi}{5}\PY{p}{]}\PY{p}{)}\PY{p}{,} \PY{n}{node\PYZus{}size}\PY{o}{=}\PY{l+m+mi}{15}\PY{p}{)}
        \PY{n}{plt}\PY{o}{.}\PY{n}{sca}\PY{p}{(}\PY{n}{ax7}\PY{p}{)}
        \PY{n}{ax7}\PY{o}{.}\PY{n}{set\PYZus{}title}\PY{p}{(}\PY{l+s+s1}{\PYZsq{}}\PY{l+s+s1}{60}\PY{l+s+s1}{\PYZpc{}}\PY{l+s+s1}{\PYZsq{}}\PY{p}{)}
        \PY{n}{nx}\PY{o}{.}\PY{n}{draw}\PY{p}{(}\PY{n}{desarme}\PY{p}{(}\PY{n}{dolph}\PY{p}{,} \PY{n}{porcentaje}\PY{p}{[}\PY{l+m+mi}{6}\PY{p}{]}\PY{p}{)}\PY{p}{,} \PY{n}{node\PYZus{}size}\PY{o}{=}\PY{l+m+mi}{15}\PY{p}{)}
        \PY{n}{plt}\PY{o}{.}\PY{n}{sca}\PY{p}{(}\PY{n}{ax8}\PY{p}{)}
        \PY{n}{ax8}\PY{o}{.}\PY{n}{set\PYZus{}title}\PY{p}{(}\PY{l+s+s1}{\PYZsq{}}\PY{l+s+s1}{70}\PY{l+s+s1}{\PYZpc{}}\PY{l+s+s1}{\PYZsq{}}\PY{p}{)}
        \PY{n}{nx}\PY{o}{.}\PY{n}{draw}\PY{p}{(}\PY{n}{desarme}\PY{p}{(}\PY{n}{dolph}\PY{p}{,} \PY{n}{porcentaje}\PY{p}{[}\PY{l+m+mi}{7}\PY{p}{]}\PY{p}{)}\PY{p}{,} \PY{n}{node\PYZus{}size}\PY{o}{=}\PY{l+m+mi}{15}\PY{p}{)}
        \PY{n}{plt}\PY{o}{.}\PY{n}{sca}\PY{p}{(}\PY{n}{ax9}\PY{p}{)}
        \PY{n}{ax9}\PY{o}{.}\PY{n}{set\PYZus{}title}\PY{p}{(}\PY{l+s+s1}{\PYZsq{}}\PY{l+s+s1}{80}\PY{l+s+s1}{\PYZpc{}}\PY{l+s+s1}{\PYZsq{}}\PY{p}{)}
        \PY{n}{nx}\PY{o}{.}\PY{n}{draw}\PY{p}{(}\PY{n}{desarme}\PY{p}{(}\PY{n}{dolph}\PY{p}{,} \PY{n}{porcentaje}\PY{p}{[}\PY{l+m+mi}{8}\PY{p}{]}\PY{p}{)}\PY{p}{,} \PY{n}{node\PYZus{}size}\PY{o}{=}\PY{l+m+mi}{15}\PY{p}{)}
\end{Verbatim}


    \begin{figure}
\centering
\includegraphics{attachment:desarme_red_funcion_porcentaje.png}
\caption{desarme\_red\_funcion\_porcentaje.png}
\end{figure}

    En la imagen que devuelve el codigo se puede observar como se desarma la
red, en funcion del porcentaje de nodos removidos. Se puede observar que
existe un porcentaje critico a partir del cual la componente gigante
queda desarmada casi en su totalidad.

Si se quiere dividir la componente original en 2 componentes de tamaño
similar, se debe utilizar la función desarme\_de\_red() hasta que la
cantidad de nodos de una componente sea parecida a la cantidad de nodos
en la otra, con algún criterio para definir \(parecida\). Observando el
gráfico, se puede observar que dicho porcentaje se encuentre entre el
60\% y el 70\%.

    \section{Ejercicio 3}\label{ejercicio-3}

    \begin{Verbatim}[commandchars=\\\{\}]
{\color{incolor}In [{\color{incolor}156}]:} \PY{k+kn}{import} \PY{n+nn}{networkx} \PY{k+kn}{as} \PY{n+nn}{nx}
          \PY{k+kn}{import} \PY{n+nn}{matplotlib.pyplot} \PY{k+kn}{as} \PY{n+nn}{plt}
          \PY{k+kn}{import} \PY{n+nn}{numpy} \PY{k+kn}{as} \PY{n+nn}{np}
          \PY{k+kn}{from} \PY{n+nn}{networkx.readwrite.gml} \PY{k+kn}{import} \PY{n}{read\PYZus{}gml}
          \PY{k+kn}{from} \PY{n+nn}{\PYZus{}\PYZus{}future\PYZus{}\PYZus{}} \PY{k+kn}{import} \PY{n}{division}
          \PY{k+kn}{import} \PY{n+nn}{rpy2.robjects} \PY{k+kn}{as} \PY{n+nn}{ro} \PY{c+c1}{\PYZsh{} Al hacer esto se inicializa un subproceso de R}
          \PY{k+kn}{from} \PY{n+nn}{rpy2.robjects.packages} \PY{k+kn}{import} \PY{n}{importr}
          \PY{k+kn}{from} \PY{n+nn}{histograma} \PY{k+kn}{import} \PY{n}{histograma}
          \PY{c+c1}{\PYZsh{} Usando importr, importamos paquetes de R que van a funcionar algo }
          \PY{c+c1}{\PYZsh{} así como módulos de Python}
          
          \PY{n}{internet} \PY{o}{=} \PY{n}{read\PYZus{}gml}\PY{p}{(}\PY{l+s+s1}{\PYZsq{}}\PY{l+s+s1}{tc01\PYZus{}data/as\PYZhy{}22july06.gml}\PY{l+s+s1}{\PYZsq{}}\PY{p}{)}
          \PY{n}{nodes} \PY{o}{=} \PY{p}{[}\PY{p}{]}
          \PY{n}{degrees} \PY{o}{=} \PY{p}{[}\PY{p}{]}
          \PY{k}{for} \PY{n}{a}\PY{p}{,} \PY{n}{b} \PY{o+ow}{in} \PY{n}{internet}\PY{o}{.}\PY{n}{degree}\PY{p}{(}\PY{p}{)}\PY{p}{:}
              \PY{n}{nodes}\PY{o}{.}\PY{n}{append}\PY{p}{(}\PY{n}{a}\PY{p}{)}
              \PY{n}{degrees}\PY{o}{.}\PY{n}{append}\PY{p}{(}\PY{n}{b}\PY{p}{)}
\end{Verbatim}


    \begin{Verbatim}[commandchars=\\\{\}]
{\color{incolor}In [{\color{incolor}157}]:} \PY{c+c1}{\PYZsh{}\PYZpc{}\PYZpc{} PUNTO A: Comparación de visualizaciones}
          
          \PY{c+c1}{\PYZsh{} Para comparar los bineados logarítmicos y no logarítmicos, lo justo es}
          \PY{c+c1}{\PYZsh{} excluir a los nodos de grado 0 en ambos }
          
          \PY{n}{fig}\PY{p}{,} \PY{n}{axes} \PY{o}{=} \PY{n}{plt}\PY{o}{.}\PY{n}{subplots}\PY{p}{(}\PY{l+m+mi}{4}\PY{p}{,} \PY{l+m+mi}{2}\PY{p}{,} \PY{n}{figsize}\PY{o}{=}\PY{p}{(}\PY{l+m+mi}{12}\PY{p}{,}\PY{l+m+mi}{10}\PY{p}{)}\PY{p}{)}
          \PY{n}{axes} \PY{o}{=} \PY{n}{axes}\PY{o}{.}\PY{n}{flatten}\PY{p}{(}\PY{p}{)}
          \PY{n}{logbinss} \PY{o}{=} \PY{p}{[}\PY{l+m+mi}{0}\PY{p}{,} \PY{l+m+mi}{0}\PY{p}{,} \PY{l+m+mi}{0}\PY{p}{,} \PY{l+m+mi}{0}\PY{p}{,} \PY{l+m+mi}{1}\PY{p}{,} \PY{l+m+mi}{1}\PY{p}{,} \PY{l+m+mi}{1}\PY{p}{,} \PY{l+m+mi}{1}\PY{p}{]}
          \PY{n}{logxs}    \PY{o}{=} \PY{p}{[}\PY{l+m+mi}{0}\PY{p}{,} \PY{l+m+mi}{0}\PY{p}{,} \PY{l+m+mi}{1}\PY{p}{,} \PY{l+m+mi}{1}\PY{p}{,} \PY{l+m+mi}{0}\PY{p}{,} \PY{l+m+mi}{0}\PY{p}{,} \PY{l+m+mi}{1}\PY{p}{,} \PY{l+m+mi}{1}\PY{p}{]}
          \PY{n}{logys}    \PY{o}{=} \PY{p}{[}\PY{l+m+mi}{0}\PY{p}{,} \PY{l+m+mi}{1}\PY{p}{,} \PY{l+m+mi}{0}\PY{p}{,} \PY{l+m+mi}{1}\PY{p}{,} \PY{l+m+mi}{0}\PY{p}{,} \PY{l+m+mi}{1}\PY{p}{,} \PY{l+m+mi}{0}\PY{p}{,} \PY{l+m+mi}{1}\PY{p}{]}
          
          \PY{n}{t} \PY{o}{=} \PY{p}{[}\PY{l+s+s1}{\PYZsq{}}\PY{l+s+s1}{Bines lineales}\PY{l+s+s1}{\PYZsq{}}\PY{p}{,} \PY{l+s+s1}{\PYZsq{}}\PY{l+s+s1}{Bines logarítmicos}\PY{l+s+s1}{\PYZsq{}}\PY{p}{]}
          \PY{n}{titulos}  \PY{o}{=} \PY{p}{[}\PY{n}{t}\PY{p}{[}\PY{n}{i}\PY{p}{]} \PY{k}{for} \PY{n}{i} \PY{o+ow}{in} \PY{n}{logbinss}\PY{p}{]}
          \PY{n}{xlabels} \PY{o}{=} \PY{p}{[}\PY{p}{(}\PY{l+s+s1}{\PYZsq{}}\PY{l+s+s1}{Grado (adim.)}\PY{l+s+s1}{\PYZsq{}} \PY{k}{if} \PY{n}{i} \PY{o+ow}{in} \PY{p}{[}\PY{l+m+mi}{6}\PY{p}{,}\PY{l+m+mi}{7}\PY{p}{]} \PY{k}{else} \PY{n+nb+bp}{None}\PY{p}{)} \PY{k}{for} \PY{n}{i} \PY{o+ow}{in} \PY{n+nb}{range}\PY{p}{(}\PY{l+m+mi}{8}\PY{p}{)}\PY{p}{]}
          \PY{n}{ylabels} \PY{o}{=} \PY{p}{[}\PY{p}{(}\PY{n+nb+bp}{True} \PY{k}{if} \PY{n}{i} \PY{o}{\PYZpc{}} \PY{l+m+mi}{2} \PY{o}{==} \PY{l+m+mi}{0} \PY{k}{else} \PY{n+nb+bp}{False}\PY{p}{)} \PY{k}{for} \PY{n}{i} \PY{o+ow}{in} \PY{n+nb}{range}\PY{p}{(}\PY{l+m+mi}{8}\PY{p}{)}\PY{p}{]}
          
          \PY{k}{for} \PY{n}{i} \PY{o+ow}{in} \PY{n+nb}{range}\PY{p}{(}\PY{l+m+mi}{8}\PY{p}{)}\PY{p}{:}
              \PY{n}{histograma}\PY{p}{(}\PY{n}{degrees}\PY{p}{,}
                         \PY{n}{logbins}\PY{o}{=}\PY{n}{logbinss}\PY{p}{[}\PY{n}{i}\PY{p}{]}\PY{p}{,} \PY{n}{logx}\PY{o}{=}\PY{n}{logxs}\PY{p}{[}\PY{n}{i}\PY{p}{]}\PY{p}{,} \PY{n}{logy}\PY{o}{=}\PY{n}{logys}\PY{p}{[}\PY{n}{i}\PY{p}{]}\PY{p}{,} \PY{n}{ax}\PY{o}{=}\PY{n}{axes}\PY{p}{[}\PY{n}{i}\PY{p}{]}\PY{p}{,}
                         \PY{n}{titulo}\PY{o}{=}\PY{n}{titulos}\PY{p}{[}\PY{n}{i}\PY{p}{]}\PY{p}{,} \PY{n}{xlabel}\PY{o}{=}\PY{n}{xlabels}\PY{p}{[}\PY{n}{i}\PY{p}{]}\PY{p}{,} \PY{n}{ylabel}\PY{o}{=}\PY{n}{ylabels}\PY{p}{[}\PY{n}{i}\PY{p}{]}\PY{p}{,}
                         \PY{n}{ecolor}\PY{o}{=}\PY{l+s+s1}{\PYZsq{}}\PY{l+s+s1}{k}\PY{l+s+s1}{\PYZsq{}}\PY{p}{,} \PY{n}{errbars}\PY{o}{=}\PY{n+nb+bp}{False}\PY{p}{,} 
                         \PY{n}{labelsize}\PY{o}{=}\PY{l+m+mi}{10}\PY{p}{,} \PY{n}{ticksize}\PY{o}{=}\PY{l+m+mi}{10}\PY{p}{,}
                         \PY{n}{bins}\PY{o}{=}\PY{p}{(}\PY{l+m+mi}{1}\PY{p}{,} \PY{n+nb}{max}\PY{p}{(}\PY{n}{degrees}\PY{p}{)} \PY{o}{+} \PY{l+m+mi}{2}\PY{p}{,} \PY{l+m+mi}{100}\PY{p}{)}\PY{p}{)}
\end{Verbatim}


    \begin{center}
    \adjustimage{max size={0.9\linewidth}{0.9\paperheight}}{output_81_0.png}
    \end{center}
    { \hspace*{\fill} \\}
    
    Por alguna razón este código no está haciendo lo que queremos en el
notebook, por lo tanto copiamos abajo la imagen con los histogramas
pre-generados (con ese mismo código).

    \begin{figure}
\centering
\includegraphics{attachment:Ej3.png}
\caption{Ej3.png}
\end{figure}

    La visualización más adecuada es con bines logarítmicos y escala
log-log.

    \begin{Verbatim}[commandchars=\\\{\}]
{\color{incolor}In [{\color{incolor}158}]:} \PY{l+s+sd}{\PYZsq{}\PYZsq{}\PYZsq{}Algo interesante de ver es que el 98\PYZpc{} de los degrees estan entre los }
          \PY{l+s+sd}{degrees 0 y 20:\PYZsq{}\PYZsq{}\PYZsq{}}
          \PY{n}{frac} \PY{o}{=} \PY{n}{np}\PY{o}{.}\PY{n}{sum}\PY{p}{(}\PY{p}{[}\PY{n}{d} \PY{o}{\PYZlt{}}\PY{o}{=} \PY{l+m+mi}{20} \PY{k}{for} \PY{n}{d} \PY{o+ow}{in} \PY{n}{degrees}\PY{p}{]}\PY{p}{)} \PY{o}{/} \PY{n+nb}{len}\PY{p}{(}\PY{n}{degrees}\PY{p}{)}
          \PY{k}{print}\PY{p}{(}\PY{n}{frac}\PY{p}{)}
\end{Verbatim}


    \begin{Verbatim}[commandchars=\\\{\}]
0.9805339023646736

    \end{Verbatim}

    \subsubsection{AJUSTE DE LEY DE POTENCIAS VIA MÉTODO
CLAUSET-SHALIZI-NEWMAN}\label{ajuste-de-ley-de-potencias-via-muxe9todo-clauset-shalizi-newman}

    Vamos a llamar a R desde Python usando la librería rpy2

    \begin{Verbatim}[commandchars=\\\{\}]
{\color{incolor}In [{\color{incolor}4}]:} \PY{c+c1}{\PYZsh{}\PYZsh{} EJECUTAR ESTO si no tienen instalado el paquete igraph (para instalarlo)}
        \PY{c+c1}{\PYZsh{}\PYZsh{} import rpy2\PYZsq{}s package module}
        \PY{c+c1}{\PYZsh{}\PYZsh{} select a mirror for R packages}
        \PY{c+c1}{\PYZsh{}utils = importr(\PYZsq{}utils\PYZsq{})}
        \PY{c+c1}{\PYZsh{}utils.chooseCRANmirror(ind=2) \PYZsh{} elijo de dónde descargar el paquete}
        \PY{c+c1}{\PYZsh{}\PYZsh{} Instalo}
        \PY{c+c1}{\PYZsh{}from rpy2.robjects.vectors import StrVector}
        \PY{c+c1}{\PYZsh{}utils.install\PYZus{}packages(StrVector([\PYZsq{}igraph\PYZsq{}]))}
\end{Verbatim}


    \begin{Verbatim}[commandchars=\\\{\}]
{\color{incolor}In [{\color{incolor}5}]:} \PY{c+c1}{\PYZsh{} Realizo el ajuste de la powerlaw}
        \PY{n}{igraph} \PY{o}{=} \PY{n}{importr}\PY{p}{(}\PY{l+s+s1}{\PYZsq{}}\PY{l+s+s1}{igraph}\PY{l+s+s1}{\PYZsq{}}\PY{p}{)}
        \PY{c+c1}{\PYZsh{} Creamos un vector de R pasándole los degrees}
        \PY{n}{degrees\PYZus{}r} \PY{o}{=} \PY{n}{ro}\PY{o}{.}\PY{n}{FloatVector}\PY{p}{(}\PY{n}{degrees}\PY{p}{)}
        \PY{c+c1}{\PYZsh{} Documentación de fit\PYZus{}power\PYZus{}law:}
        \PY{c+c1}{\PYZsh{} https://rdrr.io/cran/igraph/man/fit\PYZus{}power\PYZus{}law.html}
        \PY{n}{resultado} \PY{o}{=} \PY{n}{igraph}\PY{o}{.}\PY{n}{fit\PYZus{}power\PYZus{}law}\PY{p}{(}\PY{n}{degrees\PYZus{}r}\PY{p}{,} \PY{n}{implementation}\PY{o}{=}\PY{l+s+s1}{\PYZsq{}}\PY{l+s+s1}{plfit}\PY{l+s+s1}{\PYZsq{}}\PY{p}{)}
        \PY{k}{print}\PY{p}{(}\PY{n}{resultado}\PY{o}{.}\PY{n}{r\PYZus{}repr}\PY{p}{(}\PY{p}{)}\PY{p}{)}
\end{Verbatim}


    \begin{Verbatim}[commandchars=\\\{\}]
list(continuous = FALSE, alpha = 2.08715666887787, xmin = 5, 
    logLik = -7690.8859032442, KS.stat = 0.0102570630609345, 
    KS.p = 0.969070170897117)

    \end{Verbatim}

    \begin{Verbatim}[commandchars=\\\{\}]
{\color{incolor}In [{\color{incolor}7}]:} \PY{c+c1}{\PYZsh{} Graficamos histograma + ajuste}
        \PY{n}{kmin} \PY{o}{=} \PY{n}{resultado}\PY{o}{.}\PY{n}{rx2}\PY{p}{(}\PY{l+s+s1}{\PYZsq{}}\PY{l+s+s1}{xmin}\PY{l+s+s1}{\PYZsq{}}\PY{p}{)}\PY{p}{[}\PY{l+m+mi}{0}\PY{p}{]}
        \PY{n}{gamma} \PY{o}{=} \PY{n}{resultado}\PY{o}{.}\PY{n}{rx2}\PY{p}{(}\PY{l+s+s1}{\PYZsq{}}\PY{l+s+s1}{alpha}\PY{l+s+s1}{\PYZsq{}}\PY{p}{)}\PY{p}{[}\PY{l+m+mi}{0}\PY{p}{]}
        \PY{n}{ksp} \PY{o}{=} \PY{n}{resultado}\PY{o}{.}\PY{n}{rx2}\PY{p}{(}\PY{l+s+s1}{\PYZsq{}}\PY{l+s+s1}{KS.p}\PY{l+s+s1}{\PYZsq{}}\PY{p}{)}\PY{p}{[}\PY{l+m+mi}{0}\PY{p}{]}
        
        \PY{k+kn}{from} \PY{n+nn}{scipy.special} \PY{k+kn}{import} \PY{n}{zeta}
        \PY{k}{def} \PY{n+nf}{powerlaw}\PY{p}{(}\PY{n}{x}\PY{p}{,} \PY{n}{gamma}\PY{p}{,} \PY{n}{kmin}\PY{p}{)}\PY{p}{:}
            \PY{c+c1}{\PYZsh{} Como nuestro ajuste fue sobre una distribución de probabilidad discreta,}
            \PY{c+c1}{\PYZsh{} la cte de normalización es 1 sobre la función zeta de Riemann generalizada}
            \PY{k}{return} \PY{n}{x}\PY{o}{*}\PY{o}{*}\PY{p}{(}\PY{o}{\PYZhy{}}\PY{n}{gamma}\PY{p}{)} \PY{o}{/} \PY{n}{zeta}\PY{p}{(}\PY{n}{gamma}\PY{p}{,} \PY{n}{kmin}\PY{p}{)}
        
        \PY{c+c1}{\PYZsh{} ACLARACION IMPORTANTE}
        \PY{c+c1}{\PYZsh{} Para que se grafique bien la ley de potencias, es necesario llamar}
        \PY{c+c1}{\PYZsh{} a la función powerlaw poniendo kmin=1. Esto se debe a que el histograma}
        \PY{c+c1}{\PYZsh{} de grados está normalizado arrancando desde k=1, y no desde el kmin que}
        \PY{c+c1}{\PYZsh{} elige la función fit\PYZus{}power\PYZus{}law.}
        
        \PY{n}{fig}\PY{p}{,} \PY{n}{ax} \PY{o}{=} \PY{n}{plt}\PY{o}{.}\PY{n}{subplots}\PY{p}{(}\PY{n}{figsize}\PY{o}{=}\PY{p}{(}\PY{l+m+mi}{8}\PY{p}{,}\PY{l+m+mi}{6}\PY{p}{)}\PY{p}{)}
        \PY{n}{titulo} \PY{o}{=} \PY{l+s+s1}{\PYZsq{}}\PY{l+s+s1}{Histograma de grados con ajuste por ley de potencias}\PY{l+s+s1}{\PYZsq{}}
        \PY{n}{xs} \PY{o}{=} \PY{n}{np}\PY{o}{.}\PY{n}{linspace}\PY{p}{(}\PY{l+m+mi}{1}\PY{p}{,} \PY{n+nb}{max}\PY{p}{(}\PY{n}{degrees}\PY{p}{)} \PY{o}{+} \PY{l+m+mi}{2}\PY{p}{,} \PY{l+m+mi}{1000}\PY{p}{)}
        \PY{n}{ax}\PY{o}{.}\PY{n}{plot}\PY{p}{(}\PY{n}{xs}\PY{p}{,} \PY{n}{powerlaw}\PY{p}{(}\PY{n}{xs}\PY{p}{,} \PY{n}{gamma}\PY{p}{,} \PY{l+m+mi}{1}\PY{p}{)}\PY{p}{,} \PY{l+s+s1}{\PYZsq{}}\PY{l+s+s1}{\PYZhy{}\PYZhy{}}\PY{l+s+s1}{\PYZsq{}}\PY{p}{,} \PY{n}{color}\PY{o}{=}\PY{l+s+s1}{\PYZsq{}}\PY{l+s+s1}{deeppink}\PY{l+s+s1}{\PYZsq{}}\PY{p}{,}
                \PY{n}{label}\PY{o}{=}\PY{l+s+sa}{r}\PY{l+s+s1}{\PYZsq{}}\PY{l+s+s1}{\PYZdl{}p(k) }\PY{l+s+s1}{\PYZbs{}}\PY{l+s+s1}{propto k\PYZca{}\PYZob{}\PYZhy{}}\PY{l+s+s1}{\PYZbs{}}\PY{l+s+s1}{gamma\PYZcb{}\PYZdl{}}\PY{l+s+s1}{\PYZsq{}}\PY{p}{)}
        \PY{c+c1}{\PYZsh{}xs = np.arange(1, max(degrees) + 2)}
        \PY{c+c1}{\PYZsh{}ax.plot(xs, powerlaw(xs, gamma, kmin), \PYZsq{}o\PYZsq{}, color=\PYZsq{}deeppink\PYZsq{},}
        \PY{c+c1}{\PYZsh{}        label=r\PYZsq{}\PYZdl{}\PYZbs{}gamma = \PYZdl{}\PYZsq{} + \PYZsq{}\PYZob{}:.4g\PYZcb{}\PYZsq{}.format(gamma))}
        \PY{n}{ax}\PY{o}{.}\PY{n}{plot}\PY{p}{(}\PY{p}{[}\PY{p}{]}\PY{p}{,} \PY{p}{[}\PY{p}{]}\PY{p}{,} \PY{l+s+s1}{\PYZsq{}}\PY{l+s+s1}{ }\PY{l+s+s1}{\PYZsq{}}\PY{p}{,} \PY{n}{label}\PY{o}{=}\PY{l+s+sa}{r}\PY{l+s+s1}{\PYZsq{}}\PY{l+s+s1}{\PYZdl{}}\PY{l+s+s1}{\PYZbs{}}\PY{l+s+s1}{gamma = \PYZdl{}}\PY{l+s+s1}{\PYZsq{}} \PY{o}{+} \PY{l+s+s1}{\PYZsq{}}\PY{l+s+s1}{\PYZob{}:.4g\PYZcb{}}\PY{l+s+s1}{\PYZsq{}}\PY{o}{.}\PY{n}{format}\PY{p}{(}\PY{n}{gamma}\PY{p}{)}\PY{p}{)}
        \PY{n}{ax}\PY{o}{.}\PY{n}{plot}\PY{p}{(}\PY{p}{[}\PY{p}{]}\PY{p}{,} \PY{p}{[}\PY{p}{]}\PY{p}{,} \PY{l+s+s1}{\PYZsq{}}\PY{l+s+s1}{ }\PY{l+s+s1}{\PYZsq{}}\PY{p}{,} \PY{n}{label}\PY{o}{=}\PY{l+s+sa}{r}\PY{l+s+s1}{\PYZsq{}}\PY{l+s+s1}{\PYZdl{}K\PYZus{}\PYZob{}min\PYZcb{} = \PYZdl{}}\PY{l+s+s1}{\PYZsq{}} \PY{o}{+} \PY{l+s+s1}{\PYZsq{}}\PY{l+s+s1}{\PYZob{}:.0f\PYZcb{}}\PY{l+s+s1}{\PYZsq{}}\PY{o}{.}\PY{n}{format}\PY{p}{(}\PY{n}{kmin}\PY{p}{)}\PY{p}{)}
        \PY{n}{ax}\PY{o}{.}\PY{n}{plot}\PY{p}{(}\PY{p}{[}\PY{p}{]}\PY{p}{,} \PY{p}{[}\PY{p}{]}\PY{p}{,} \PY{l+s+s1}{\PYZsq{}}\PY{l+s+s1}{ }\PY{l+s+s1}{\PYZsq{}}\PY{p}{,} \PY{n}{label}\PY{o}{=}\PY{l+s+s1}{\PYZsq{}}\PY{l+s+s1}{p\PYZhy{}value (KS) = \PYZob{}:.2g\PYZcb{}}\PY{l+s+s1}{\PYZsq{}}\PY{o}{.}\PY{n}{format}\PY{p}{(}\PY{n}{ksp}\PY{p}{)}\PY{p}{)}
        \PY{n}{ax}\PY{o}{.}\PY{n}{legend}\PY{p}{(}\PY{p}{)}
        
        \PY{n}{histograma}\PY{p}{(}\PY{n}{degrees}\PY{p}{,} \PY{n}{logbins}\PY{o}{=}\PY{n+nb+bp}{True}\PY{p}{,} \PY{n}{ax}\PY{o}{=}\PY{n}{ax}\PY{p}{,} \PY{n}{titulo}\PY{o}{=}\PY{n}{titulo}\PY{p}{,}
                   \PY{n}{logx}\PY{o}{=}\PY{n+nb+bp}{True}\PY{p}{,} \PY{n}{logy}\PY{o}{=}\PY{n+nb+bp}{True}\PY{p}{,}
                   \PY{n}{xlabel}\PY{o}{=}\PY{l+s+s1}{\PYZsq{}}\PY{l+s+s1}{k (adim.)}\PY{l+s+s1}{\PYZsq{}}\PY{p}{,} \PY{n}{ylabel}\PY{o}{=}\PY{n+nb+bp}{True}\PY{p}{,} \PY{n}{ecolor}\PY{o}{=}\PY{l+s+s1}{\PYZsq{}}\PY{l+s+s1}{k}\PY{l+s+s1}{\PYZsq{}}\PY{p}{,} \PY{n}{errbars}\PY{o}{=}\PY{n+nb+bp}{False}\PY{p}{,} 
                   \PY{n}{labelsize}\PY{o}{=}\PY{l+m+mi}{18}\PY{p}{,} \PY{n}{ticksize}\PY{o}{=}\PY{l+m+mi}{16}\PY{p}{,} \PY{n}{bins}\PY{o}{=}\PY{p}{(}\PY{l+m+mi}{1}\PY{p}{,} \PY{n+nb}{max}\PY{p}{(}\PY{n}{degrees}\PY{p}{)} \PY{o}{+} \PY{l+m+mi}{2}\PY{p}{,} \PY{l+m+mi}{50}\PY{p}{)}\PY{p}{)}
\end{Verbatim}


    \begin{center}
    \adjustimage{max size={0.9\linewidth}{0.9\paperheight}}{output_90_0.png}
    \end{center}
    { \hspace*{\fill} \\}
    
\begin{Verbatim}[commandchars=\\\{\}]
{\color{outcolor}Out[{\color{outcolor}7}]:} (<Figure size 576x432 with 1 Axes>,
         <matplotlib.axes.\_subplots.AxesSubplot at 0x7f31c2252da0>)
\end{Verbatim}
            
    El ajuste es excelente dado que el p-value del test Kolmogorov-Smirnov
es 0.97.

    \section{Ejercicio 4}\label{ejercicio-4}

    \subsubsection{Punto a}\label{punto-a}

\subsubsection{i)}\label{i}

    Primero importamos las librerías que necesitamos, e importamos las redes
sobre las cuales vamos a trabajar.

    \begin{Verbatim}[commandchars=\\\{\}]
{\color{incolor}In [{\color{incolor}8}]:} \PY{k+kn}{from} \PY{n+nn}{\PYZus{}\PYZus{}future\PYZus{}\PYZus{}} \PY{k+kn}{import} \PY{n}{division}
        \PY{k+kn}{import} \PY{n+nn}{networkx} \PY{k+kn}{as} \PY{n+nn}{nx}
        \PY{k+kn}{import} \PY{n+nn}{matplotlib.pyplot} \PY{k+kn}{as} \PY{n+nn}{plt}
        \PY{k+kn}{import} \PY{n+nn}{numpy} \PY{k+kn}{as} \PY{n+nn}{np}
        \PY{k+kn}{import} \PY{n+nn}{pandas} \PY{k+kn}{as} \PY{n+nn}{pd}
        \PY{k+kn}{from} \PY{n+nn}{networkx.readwrite.gml} \PY{k+kn}{import} \PY{n}{read\PYZus{}gml}
        
        \PY{k+kn}{from} \PY{n+nn}{scipy.odr} \PY{k+kn}{import} \PY{n}{Model}\PY{p}{,} \PY{n}{RealData}\PY{p}{,} \PY{n}{ODR}
        \PY{k+kn}{from} \PY{n+nn}{scipy} \PY{k+kn}{import} \PY{n}{stats}
        
        \PY{k+kn}{from} \PY{n+nn}{histograma} \PY{k+kn}{import} \PY{n}{histograma}
\end{Verbatim}


    \begin{Verbatim}[commandchars=\\\{\}]
{\color{incolor}In [{\color{incolor}9}]:} \PY{n}{net\PYZus{}science} \PY{o}{=} \PY{n}{read\PYZus{}gml}\PY{p}{(}\PY{l+s+s1}{\PYZsq{}}\PY{l+s+s1}{tc01\PYZus{}data/netscience.gml}\PY{l+s+s1}{\PYZsq{}}\PY{p}{)}
        \PY{n}{july} \PY{o}{=} \PY{n}{read\PYZus{}gml}\PY{p}{(}\PY{l+s+s1}{\PYZsq{}}\PY{l+s+s1}{tc01\PYZus{}data/as\PYZhy{}22july06.gml}\PY{l+s+s1}{\PYZsq{}}\PY{p}{)}
\end{Verbatim}


    Para calcular el promedio del grado medio de los vecinos de nodos de
grado k (\(k_{nn}\)), utilizamos la funcion \(annd\)(). Dada una red,
esta función genera 2 listas: una lista con el grado de cada nodo, y una
lista con el grado medio de los vecinos (\(annd\)) de dicho nodo,
calculada con nx.average\_neighbor\_degree(). Notar que estas listas
están ordenadas de la misma forma, es decir, la componente i-ésima de
ambas listas corresponden al mismo nodo. Luego, la función itera sobre
todos los nodos con la condición de que promedie los \(annd\) de todos
los nodos con igual grado y lo guarde en una lista. Finalmente la
función devuelve la lista de grados, y la lista de los promedios de
\(annd\) por grado.

La funcion nan\_delete() remueve los nan y los inf de la lista de los
promedios de \(annd\)s que aparecen por promediar listas vacías.

    \begin{Verbatim}[commandchars=\\\{\}]
{\color{incolor}In [{\color{incolor}10}]:} \PY{k}{def} \PY{n+nf}{nan\PYZus{}delete}\PY{p}{(}\PY{n}{k\PYZus{}nn}\PY{p}{)}\PY{p}{:}
             \PY{l+s+sd}{\PYZdq{}\PYZdq{}\PYZdq{}Remueve los inf y los nan de la lista manteniendo su k especifico.}
         \PY{l+s+sd}{    .}
         \PY{l+s+sd}{    .}
         \PY{l+s+sd}{    \PYZdq{}\PYZdq{}\PYZdq{}}
             \PY{n}{k} \PY{o}{=} \PY{n}{np}\PY{o}{.}\PY{n}{arange}\PY{p}{(}\PY{l+m+mi}{0}\PY{p}{,} \PY{n+nb}{len}\PY{p}{(}\PY{n}{k\PYZus{}nn}\PY{p}{)}\PY{p}{,} \PY{l+m+mi}{1}\PY{p}{)}
             \PY{n}{k\PYZus{}nn\PYZus{}temp} \PY{o}{=} \PY{p}{[}\PY{p}{]}
             \PY{n}{k\PYZus{}temp} \PY{o}{=} \PY{p}{[}\PY{p}{]}    
             \PY{k}{for} \PY{n}{i} \PY{o+ow}{in} \PY{n+nb}{range}\PY{p}{(}\PY{n+nb}{len}\PY{p}{(}\PY{n}{k\PYZus{}nn}\PY{p}{)}\PY{p}{)}\PY{p}{:}
                 \PY{k}{if} \PY{o+ow}{not} \PY{n}{np}\PY{o}{.}\PY{n}{isinf}\PY{p}{(}\PY{n}{k\PYZus{}nn}\PY{p}{[}\PY{n}{i}\PY{p}{]}\PY{p}{)} \PY{o+ow}{and} \PY{o+ow}{not} \PY{n}{np}\PY{o}{.}\PY{n}{isnan}\PY{p}{(}\PY{n}{k\PYZus{}nn}\PY{p}{[}\PY{n}{i}\PY{p}{]}\PY{p}{)} \PY{o+ow}{and} \PY{o+ow}{not} \PY{n}{np}\PY{o}{.}\PY{n}{isinf}\PY{p}{(}\PY{n}{k}\PY{p}{[}\PY{n}{i}\PY{p}{]}\PY{p}{)}\PY{p}{:}
                     \PY{n}{k\PYZus{}nn\PYZus{}temp}\PY{o}{.}\PY{n}{append}\PY{p}{(}\PY{n}{k\PYZus{}nn}\PY{p}{[}\PY{n}{i}\PY{p}{]}\PY{p}{)}
                     \PY{n}{k\PYZus{}temp}\PY{o}{.}\PY{n}{append}\PY{p}{(}\PY{n}{k}\PY{p}{[}\PY{n}{i}\PY{p}{]}\PY{p}{)}
             \PY{k}{return} \PY{n}{k\PYZus{}temp}\PY{p}{,} \PY{n}{k\PYZus{}nn\PYZus{}temp}
         
         \PY{k}{def} \PY{n+nf}{annd}\PY{p}{(}\PY{n}{red}\PY{p}{)}\PY{p}{:}
             \PY{l+s+sd}{\PYZdq{}\PYZdq{}\PYZdq{} annd (average neighbouhr degree distribution) devuelve el annd en orden}
         \PY{l+s+sd}{    de los grados del nodo.}
         \PY{l+s+sd}{    }
         \PY{l+s+sd}{    Returns: k, k\PYZus{}nn}
         \PY{l+s+sd}{    }
         \PY{l+s+sd}{    k: array con los grados de la red (eje X)}
         \PY{l+s+sd}{    k\PYZus{}nn: array con los annd promediados por grado (eje Y)}
         \PY{l+s+sd}{    .}
         \PY{l+s+sd}{    .}
         \PY{l+s+sd}{    \PYZdq{}\PYZdq{}\PYZdq{}}
             \PY{n}{nombres} \PY{o}{=} \PY{n+nb}{list}\PY{p}{(}\PY{n}{red}\PY{o}{.}\PY{n}{nodes}\PY{p}{)}
             \PY{n}{avnedeg} \PY{o}{=} \PY{n}{nx}\PY{o}{.}\PY{n}{average\PYZus{}neighbor\PYZus{}degree}\PY{p}{(}\PY{n}{red}\PY{p}{)}
             \PY{n}{grados} \PY{o}{=} \PY{n}{nx}\PY{o}{.}\PY{n}{degree}\PY{p}{(}\PY{n}{red}\PY{p}{)}
             \PY{n}{a} \PY{o}{=} \PY{p}{[}\PY{p}{]}
             \PY{k}{for} \PY{n}{i} \PY{o+ow}{in} \PY{n+nb}{range}\PY{p}{(}\PY{n+nb}{max}\PY{p}{(}\PY{n+nb}{dict}\PY{p}{(}\PY{n}{nx}\PY{o}{.}\PY{n}{degree}\PY{p}{(}\PY{n}{red}\PY{p}{)}\PY{p}{)}\PY{o}{.}\PY{n}{values}\PY{p}{(}\PY{p}{)}\PY{p}{)}\PY{o}{+}\PY{l+m+mi}{1}\PY{p}{)}\PY{p}{:}
                 \PY{n}{b} \PY{o}{=} \PY{p}{[}\PY{p}{]}
                 \PY{k}{for} \PY{n}{j} \PY{o+ow}{in} \PY{n+nb}{range}\PY{p}{(}\PY{n+nb}{len}\PY{p}{(}\PY{n}{nombres}\PY{p}{)}\PY{p}{)}\PY{p}{:}
                     \PY{k}{if} \PY{n}{i} \PY{o}{==} \PY{n}{grados}\PY{p}{[}\PY{n}{nombres}\PY{p}{[}\PY{n}{j}\PY{p}{]}\PY{p}{]}\PY{p}{:}
                         \PY{n}{b}\PY{o}{.}\PY{n}{append}\PY{p}{(}\PY{n}{avnedeg}\PY{p}{[}\PY{n}{nombres}\PY{p}{[}\PY{n}{j}\PY{p}{]}\PY{p}{]}\PY{p}{)}
                 \PY{n}{a}\PY{o}{.}\PY{n}{append}\PY{p}{(}\PY{n}{np}\PY{o}{.}\PY{n}{mean}\PY{p}{(}\PY{n}{b}\PY{p}{)}\PY{p}{)}
             \PY{n}{k}\PY{p}{,} \PY{n}{k\PYZus{}nn} \PY{o}{=} \PY{n}{nan\PYZus{}delete}\PY{p}{(}\PY{n}{a}\PY{p}{)}
             \PY{k}{return} \PY{n}{k}\PY{p}{,} \PY{n}{k\PYZus{}nn}
         
         \PY{n}{degree\PYZus{}2}\PY{p}{,} \PY{n}{annd\PYZus{}2} \PY{o}{=} \PY{n}{annd}\PY{p}{(}\PY{n}{net\PYZus{}science}\PY{p}{)}
         \PY{n}{degree\PYZus{}1}\PY{p}{,} \PY{n}{annd\PYZus{}1} \PY{o}{=} \PY{n}{annd}\PY{p}{(}\PY{n}{july}\PY{p}{)}
         
         \PY{c+c1}{\PYZsh{}OBS: si correr esta celda devuelve un error de Mean of empty slice, o similar, no pasa nada.}
\end{Verbatim}


    \begin{Verbatim}[commandchars=\\\{\}]
/home/gabo/anaconda3/lib/python3.6/site-packages/numpy/core/fromnumeric.py:2957: RuntimeWarning: Mean of empty slice.
  out=out, **kwargs)
/home/gabo/anaconda3/lib/python3.6/site-packages/numpy/core/\_methods.py:80: RuntimeWarning: invalid value encountered in double\_scalars
  ret = ret.dtype.type(ret / rcount)

    \end{Verbatim}

    \subsubsection{ii)}\label{ii}

    Para observar la tendencia de los \(k_{nn}\) graficamos, para cada red,
4 formas distintas de observar los gráficos: la escala lineal, la escala
log-log, la cumulativa en escala log-log, y un histograma de los
\(k_{nn}\). Ambos gráficos fueron generados con la funcón
gráficos\_multiples(), pasando degree\_i como el eje x y annd\_i como el
eje y.

    \begin{Verbatim}[commandchars=\\\{\}]
{\color{incolor}In [{\color{incolor}11}]:} \PY{k}{def} \PY{n+nf}{graficos\PYZus{}multiples}\PY{p}{(}\PY{n}{x}\PY{p}{,} \PY{n}{y}\PY{p}{,} \PY{n}{network\PYZus{}name}\PY{p}{)}\PY{p}{:}
             \PY{n}{f}\PY{p}{,} \PY{p}{(}\PY{p}{[}\PY{n}{ax1}\PY{p}{,} \PY{n}{ax2}\PY{p}{]}\PY{p}{,} \PY{p}{[}\PY{n}{ax3}\PY{p}{,} \PY{n}{ax4}\PY{p}{]}\PY{p}{)} \PY{o}{=} \PY{n}{plt}\PY{o}{.}\PY{n}{subplots}\PY{p}{(}\PY{l+m+mi}{2}\PY{p}{,} \PY{l+m+mi}{2}\PY{p}{)}
             \PY{n}{f}\PY{o}{.}\PY{n}{tight\PYZus{}layout}\PY{p}{(}\PY{p}{)}
             \PY{n}{f}\PY{o}{.}\PY{n}{suptitle}\PY{p}{(}\PY{n}{network\PYZus{}name}\PY{p}{,} \PY{n}{x}\PY{o}{=}\PY{l+m+mf}{0.53}\PY{p}{,} \PY{n}{y} \PY{o}{=} \PY{l+m+mf}{0.99} \PY{p}{,}\PY{n}{fontsize}\PY{o}{=}\PY{l+m+mi}{15}\PY{p}{)}
\end{Verbatim}


    \begin{Verbatim}[commandchars=\\\{\}]
{\color{incolor}In [{\color{incolor}12}]:}     \PY{n}{plt}\PY{o}{.}\PY{n}{sca}\PY{p}{(}\PY{n}{ax1}\PY{p}{)}
             \PY{n}{ax1}\PY{o}{.}\PY{n}{set\PYZus{}title}\PY{p}{(}\PY{l+s+s1}{\PYZsq{}}\PY{l+s+s1}{(a)  Lineal}\PY{l+s+s1}{\PYZsq{}}\PY{p}{)}
             \PY{n}{ax1}\PY{o}{.}\PY{n}{plot}\PY{p}{(}\PY{n}{x}\PY{p}{,} \PY{n}{y}\PY{p}{,} \PY{l+s+s1}{\PYZsq{}}\PY{l+s+s1}{.}\PY{l+s+s1}{\PYZsq{}}\PY{p}{)}
             \PY{n}{ax1}\PY{o}{.}\PY{n}{set\PYZus{}ylabel}\PY{p}{(}\PY{l+s+sa}{r}\PY{l+s+s1}{\PYZsq{}}\PY{l+s+s1}{\PYZdl{}k\PYZus{}\PYZob{}nn\PYZcb{}\PYZdl{}}\PY{l+s+s1}{\PYZsq{}}\PY{p}{)}
             \PY{n}{ax1}\PY{o}{.}\PY{n}{set\PYZus{}xlabel}\PY{p}{(}\PY{l+s+s1}{\PYZsq{}}\PY{l+s+s1}{k}\PY{l+s+s1}{\PYZsq{}}\PY{p}{)}
             
             \PY{n}{plt}\PY{o}{.}\PY{n}{sca}\PY{p}{(}\PY{n}{ax2}\PY{p}{)}
             \PY{n}{ax2}\PY{o}{.}\PY{n}{set\PYZus{}title}\PY{p}{(}\PY{l+s+s1}{\PYZsq{}}\PY{l+s+s1}{(b)  Log Log}\PY{l+s+s1}{\PYZsq{}}\PY{p}{)}
             \PY{n}{ax2}\PY{o}{.}\PY{n}{loglog}\PY{p}{(}\PY{n}{x}\PY{p}{,} \PY{n}{y}\PY{p}{,} \PY{l+s+s1}{\PYZsq{}}\PY{l+s+s1}{.}\PY{l+s+s1}{\PYZsq{}}\PY{p}{)}
             \PY{n}{ax2}\PY{o}{.}\PY{n}{set\PYZus{}ylabel}\PY{p}{(}\PY{l+s+sa}{r}\PY{l+s+s1}{\PYZsq{}}\PY{l+s+s1}{\PYZdl{}k\PYZus{}\PYZob{}nn\PYZcb{}\PYZdl{}}\PY{l+s+s1}{\PYZsq{}}\PY{p}{)}
             \PY{n}{ax2}\PY{o}{.}\PY{n}{set\PYZus{}xlabel}\PY{p}{(}\PY{l+s+s1}{\PYZsq{}}\PY{l+s+s1}{k}\PY{l+s+s1}{\PYZsq{}}\PY{p}{)}
             \PY{c+c1}{\PYZsh{}ax2.yscale(\PYZsq{}log\PYZsq{})}
             \PY{c+c1}{\PYZsh{}ax2.xscale(\PYZsq{}log\PYZsq{})}
             
             \PY{n}{plt}\PY{o}{.}\PY{n}{sca}\PY{p}{(}\PY{n}{ax3}\PY{p}{)}
             \PY{n}{coefs\PYZus{}reversed} \PY{o}{=} \PY{n}{np}\PY{o}{.}\PY{n}{flip}\PY{p}{(}\PY{n}{y}\PY{p}{,} \PY{l+m+mi}{0}\PY{p}{)}
             \PY{n}{cumulative} \PY{o}{=} \PY{n}{np}\PY{o}{.}\PY{n}{cumsum}\PY{p}{(}\PY{n}{coefs\PYZus{}reversed}\PY{p}{)}
             \PY{n}{cumulative} \PY{o}{=} \PY{n}{np}\PY{o}{.}\PY{n}{flip}\PY{p}{(}\PY{n}{cumulative}\PY{p}{,} \PY{l+m+mi}{0}\PY{p}{)}
             \PY{n}{ax3}\PY{o}{.}\PY{n}{set\PYZus{}title}\PY{p}{(}\PY{l+s+s1}{\PYZsq{}}\PY{l+s+s1}{(c)  Cumulative}\PY{l+s+s1}{\PYZsq{}}\PY{p}{)}
             \PY{n}{ax3}\PY{o}{.}\PY{n}{loglog}\PY{p}{(}\PY{n}{x}\PY{p}{,} \PY{n}{cumulative}\PY{p}{,} \PY{l+s+s1}{\PYZsq{}}\PY{l+s+s1}{.}\PY{l+s+s1}{\PYZsq{}}\PY{p}{)}
             \PY{n}{ax3}\PY{o}{.}\PY{n}{set\PYZus{}xlabel}\PY{p}{(}\PY{l+s+sa}{r}\PY{l+s+s1}{\PYZsq{}}\PY{l+s+s1}{\PYZdl{}k\PYZus{}\PYZob{}nn\PYZcb{}\PYZdl{}}\PY{l+s+s1}{\PYZsq{}}\PY{p}{)}
             
             \PY{n}{plt}\PY{o}{.}\PY{n}{sca}\PY{p}{(}\PY{n}{ax4}\PY{p}{)}
             \PY{n}{ax4}\PY{o}{.}\PY{n}{set\PYZus{}title}\PY{p}{(}\PY{l+s+s1}{\PYZsq{}}\PY{l+s+s1}{(d)  Log binned histogram}\PY{l+s+s1}{\PYZsq{}}\PY{p}{)}
             \PY{n}{histograma}\PY{p}{(}\PY{n}{y}\PY{p}{,} \PY{n}{ax}\PY{o}{=}\PY{n}{ax4}\PY{p}{,} \PY{n}{xlabel}\PY{o}{=}\PY{l+s+sa}{r}\PY{l+s+s1}{\PYZsq{}}\PY{l+s+s1}{\PYZdl{}k\PYZus{}\PYZob{}nn\PYZcb{}\PYZdl{}}\PY{l+s+s1}{\PYZsq{}}\PY{p}{,} \PY{n}{labelsize}\PY{o}{=}\PY{l+m+mi}{10}\PY{p}{,} \PY{n}{ticksize}\PY{o}{=}\PY{l+m+mi}{10}\PY{p}{)}
         
         \PY{n}{graficos\PYZus{}multiples}\PY{p}{(}\PY{n}{degree\PYZus{}1}\PY{p}{,} \PY{n}{annd\PYZus{}1}\PY{p}{,} \PY{l+s+s1}{\PYZsq{}}\PY{l+s+s1}{July}\PY{l+s+s1}{\PYZsq{}}\PY{p}{)}
         \PY{n}{graficos\PYZus{}multiples}\PY{p}{(}\PY{n}{degree\PYZus{}2}\PY{p}{,} \PY{n}{annd\PYZus{}2}\PY{p}{,} \PY{l+s+s1}{\PYZsq{}}\PY{l+s+s1}{Net\PYZus{}science}\PY{l+s+s1}{\PYZsq{}}\PY{p}{)}
\end{Verbatim}


    \begin{Verbatim}[commandchars=\\\{\}]

        ---------------------------------------------------------------------------

        NameError                                 Traceback (most recent call last)

        <ipython-input-12-305ea6ce618f> in <module>()
    ----> 1 plt.sca(ax1)
          2 ax1.set\_title('(a)  Lineal')
          3 ax1.plot(x, y, '.')
          4 ax1.set\_ylabel(r'\$k\_\{nn\}\$')
          5 ax1.set\_xlabel('k')


        NameError: name 'ax1' is not defined

    \end{Verbatim}

    \begin{figure}
\centering
\includegraphics{attachment:multiples\%20graphics.png}
\caption{multiples\%20graphics.png}
\end{figure}

    En la red de July se puede observar que decrece el \(annd\) promedio a
medida que aumenta el grado del nodo. Por otro lado, en la red
Net\_science el \(annd\) promedio aumenta con el grado de los nodos. El
análisis de este comportamiento se vera en el punto iv).

    \subsubsection{iii)}\label{iii}

    Se ajustó una recta al logaritmo de \(k_{nn}\) en función del logaritmo
de k.

    \begin{Verbatim}[commandchars=\\\{\}]
{\color{incolor}In [{\color{incolor} }]:} \PY{k}{def} \PY{n+nf}{linear}\PY{p}{(}\PY{n}{M}\PY{p}{,} \PY{n}{x}\PY{p}{)}\PY{p}{:}
            \PY{n}{m}\PY{p}{,} \PY{n}{b} \PY{o}{=} \PY{n}{M}
            \PY{k}{return} \PY{n}{x}\PY{o}{*}\PY{n}{m} \PY{o}{+} \PY{n}{b}
        
        \PY{k}{def} \PY{n+nf}{ajuste\PYZus{}lineal}\PY{p}{(}\PY{n}{degree}\PY{p}{,} \PY{n}{annd}\PY{p}{)}\PY{p}{:}
            \PY{k}{if} \PY{n}{degree}\PY{p}{[}\PY{l+m+mi}{0}\PY{p}{]} \PY{o}{==} \PY{l+m+mf}{0.}\PY{p}{:}
                \PY{n}{log\PYZus{}k\PYZus{}nn} \PY{o}{=} \PY{p}{[}\PY{n}{np}\PY{o}{.}\PY{n}{log}\PY{p}{(}\PY{n}{i}\PY{p}{)} \PY{k}{for} \PY{n}{i} \PY{o+ow}{in} \PY{n}{annd}\PY{p}{[}\PY{l+m+mi}{1}\PY{p}{:}\PY{p}{]}\PY{p}{]}
                \PY{n}{log\PYZus{}k} \PY{o}{=} \PY{p}{[}\PY{n}{np}\PY{o}{.}\PY{n}{log}\PY{p}{(}\PY{n}{i}\PY{p}{)} \PY{k}{for} \PY{n}{i} \PY{o+ow}{in} \PY{n}{degree}\PY{p}{[}\PY{l+m+mi}{1}\PY{p}{:}\PY{p}{]}\PY{p}{]}
            \PY{k}{else}\PY{p}{:}
                \PY{n}{log\PYZus{}k\PYZus{}nn} \PY{o}{=} \PY{p}{[}\PY{n}{np}\PY{o}{.}\PY{n}{log}\PY{p}{(}\PY{n}{i}\PY{p}{)} \PY{k}{for} \PY{n}{i} \PY{o+ow}{in} \PY{n}{annd}\PY{p}{]}
                \PY{n}{log\PYZus{}k} \PY{o}{=} \PY{p}{[}\PY{n}{np}\PY{o}{.}\PY{n}{log}\PY{p}{(}\PY{n}{i}\PY{p}{)} \PY{k}{for} \PY{n}{i} \PY{o+ow}{in} \PY{n}{degree}\PY{p}{]}        
            \PY{n}{linear\PYZus{}model} \PY{o}{=} \PY{n}{Model}\PY{p}{(}\PY{n}{linear}\PY{p}{)}
            \PY{n}{data} \PY{o}{=} \PY{n}{RealData}\PY{p}{(}\PY{n}{log\PYZus{}k}\PY{p}{,} \PY{n}{log\PYZus{}k\PYZus{}nn}\PY{p}{)}
            \PY{n}{odr} \PY{o}{=} \PY{n}{ODR}\PY{p}{(}\PY{n}{data}\PY{p}{,} \PY{n}{linear\PYZus{}model}\PY{p}{,} \PY{n}{beta0}\PY{o}{=}\PY{p}{[}\PY{l+m+mf}{0.}\PY{p}{,} \PY{l+m+mf}{1.}\PY{p}{]}\PY{p}{)}
            \PY{n}{out} \PY{o}{=} \PY{n}{odr}\PY{o}{.}\PY{n}{run}\PY{p}{(}\PY{p}{)}
            \PY{n}{log\PYZus{}modelo} \PY{o}{=} \PY{p}{[}\PY{n}{j}\PY{o}{*}\PY{n}{out}\PY{o}{.}\PY{n}{beta}\PY{p}{[}\PY{l+m+mi}{0}\PY{p}{]}\PY{o}{+}\PY{n}{out}\PY{o}{.}\PY{n}{beta}\PY{p}{[}\PY{l+m+mi}{1}\PY{p}{]} \PY{k}{for} \PY{n}{j} \PY{o+ow}{in} \PY{n}{log\PYZus{}k}\PY{p}{]}
            \PY{k}{return} \PY{n}{log\PYZus{}k}\PY{p}{,} \PY{n}{log\PYZus{}modelo}\PY{p}{,} \PY{n}{out}\PY{o}{.}\PY{n}{beta}\PY{p}{[}\PY{l+m+mi}{0}\PY{p}{]}
        
        \PY{n}{f}\PY{p}{,} \PY{p}{(}\PY{n}{ax1}\PY{p}{,} \PY{n}{ax2}\PY{p}{)} \PY{o}{=} \PY{n}{plt}\PY{o}{.}\PY{n}{subplots}\PY{p}{(}\PY{l+m+mi}{1}\PY{p}{,} \PY{l+m+mi}{2}\PY{p}{)}
        \PY{n}{f}\PY{o}{.}\PY{n}{tight\PYZus{}layout}\PY{p}{(}\PY{p}{)}
        \PY{n}{f}\PY{o}{.}\PY{n}{suptitle}\PY{p}{(}\PY{l+s+s1}{\PYZsq{}}\PY{l+s+s1}{Ajustes lineales}\PY{l+s+s1}{\PYZsq{}}\PY{p}{,} \PY{n}{x}\PY{o}{=}\PY{l+m+mf}{0.53}\PY{p}{,} \PY{n}{y} \PY{o}{=} \PY{l+m+mf}{0.99} \PY{p}{,}\PY{n}{fontsize}\PY{o}{=}\PY{l+m+mi}{15}\PY{p}{)}
        
        \PY{n}{plt}\PY{o}{.}\PY{n}{sca}\PY{p}{(}\PY{n}{ax1}\PY{p}{)}
        \PY{n}{ax1}\PY{o}{.}\PY{n}{set\PYZus{}title}\PY{p}{(}\PY{l+s+s1}{\PYZsq{}}\PY{l+s+s1}{July}\PY{l+s+s1}{\PYZsq{}}\PY{p}{)}
        \PY{n}{ax1}\PY{o}{.}\PY{n}{plot}\PY{p}{(}\PY{n}{np}\PY{o}{.}\PY{n}{log}\PY{p}{(}\PY{n}{degree\PYZus{}1}\PY{p}{)}\PY{p}{,} \PY{n}{np}\PY{o}{.}\PY{n}{log}\PY{p}{(}\PY{n}{annd\PYZus{}1}\PY{p}{)}\PY{p}{,} \PY{l+s+s1}{\PYZsq{}}\PY{l+s+s1}{.}\PY{l+s+s1}{\PYZsq{}}\PY{p}{)}
        \PY{n}{ax1}\PY{o}{.}\PY{n}{plot}\PY{p}{(}\PY{n}{ajuste\PYZus{}lineal}\PY{p}{(}\PY{n}{degree\PYZus{}1}\PY{p}{,} \PY{n}{annd\PYZus{}1}\PY{p}{)}\PY{p}{[}\PY{l+m+mi}{0}\PY{p}{]}\PY{p}{,} \PY{n}{ajuste\PYZus{}lineal}\PY{p}{(}\PY{n}{degree\PYZus{}1}\PY{p}{,} \PY{n}{annd\PYZus{}1}\PY{p}{)}\PY{p}{[}\PY{l+m+mi}{1}\PY{p}{]}\PY{p}{)}
        \PY{n}{ax1}\PY{o}{.}\PY{n}{set\PYZus{}ylabel}\PY{p}{(}\PY{l+s+sa}{r}\PY{l+s+s1}{\PYZsq{}}\PY{l+s+s1}{\PYZdl{}k\PYZus{}\PYZob{}nn\PYZcb{}\PYZdl{}}\PY{l+s+s1}{\PYZsq{}}\PY{p}{)}
        \PY{n}{ax1}\PY{o}{.}\PY{n}{set\PYZus{}xlabel}\PY{p}{(}\PY{l+s+s1}{\PYZsq{}}\PY{l+s+s1}{k}\PY{l+s+s1}{\PYZsq{}}\PY{p}{)}
        \PY{n}{ax1}\PY{o}{.}\PY{n}{text}\PY{p}{(}\PY{l+m+mi}{5}\PY{p}{,} \PY{l+m+mf}{6.5}\PY{p}{,} \PY{l+s+sa}{r}\PY{l+s+s1}{\PYZsq{}}\PY{l+s+s1}{\PYZdl{}}\PY{l+s+s1}{\PYZbs{}}\PY{l+s+s1}{mu\PYZdl{} = }\PY{l+s+si}{\PYZpc{}s}\PY{l+s+s1}{\PYZsq{}} \PY{o}{\PYZpc{}} \PY{n+nb}{round}\PY{p}{(}\PY{n}{ajuste\PYZus{}lineal}\PY{p}{(}\PY{n}{degree\PYZus{}1}\PY{p}{,} \PY{n}{annd\PYZus{}1}\PY{p}{)}\PY{p}{[}\PY{l+m+mi}{2}\PY{p}{]}\PY{p}{,} \PY{l+m+mi}{2}\PY{p}{)}\PY{p}{)}
        
        \PY{n}{plt}\PY{o}{.}\PY{n}{sca}\PY{p}{(}\PY{n}{ax2}\PY{p}{)}
        \PY{n}{ax2}\PY{o}{.}\PY{n}{set\PYZus{}title}\PY{p}{(}\PY{l+s+s1}{\PYZsq{}}\PY{l+s+s1}{Net Science}\PY{l+s+s1}{\PYZsq{}}\PY{p}{)}
        \PY{n}{ax2}\PY{o}{.}\PY{n}{plot}\PY{p}{(}\PY{n}{np}\PY{o}{.}\PY{n}{log}\PY{p}{(}\PY{n}{degree\PYZus{}2}\PY{p}{)}\PY{p}{,} \PY{n}{np}\PY{o}{.}\PY{n}{log}\PY{p}{(}\PY{n}{annd\PYZus{}2}\PY{p}{)}\PY{p}{,} \PY{l+s+s1}{\PYZsq{}}\PY{l+s+s1}{.}\PY{l+s+s1}{\PYZsq{}}\PY{p}{)}
        \PY{n}{ax2}\PY{o}{.}\PY{n}{plot}\PY{p}{(}\PY{n}{ajuste\PYZus{}lineal}\PY{p}{(}\PY{n}{degree\PYZus{}2}\PY{p}{,} \PY{n}{annd\PYZus{}2}\PY{p}{)}\PY{p}{[}\PY{l+m+mi}{0}\PY{p}{]}\PY{p}{,} \PY{n}{ajuste\PYZus{}lineal}\PY{p}{(}\PY{n}{degree\PYZus{}2}\PY{p}{,} \PY{n}{annd\PYZus{}2}\PY{p}{)}\PY{p}{[}\PY{l+m+mi}{1}\PY{p}{]}\PY{p}{)}
        \PY{n}{ax2}\PY{o}{.}\PY{n}{set\PYZus{}ylabel}\PY{p}{(}\PY{l+s+sa}{r}\PY{l+s+s1}{\PYZsq{}}\PY{l+s+s1}{\PYZdl{}k\PYZus{}\PYZob{}nn\PYZcb{}\PYZdl{}}\PY{l+s+s1}{\PYZsq{}}\PY{p}{)}
        \PY{n}{ax2}\PY{o}{.}\PY{n}{set\PYZus{}xlabel}\PY{p}{(}\PY{l+s+s1}{\PYZsq{}}\PY{l+s+s1}{k}\PY{l+s+s1}{\PYZsq{}}\PY{p}{)}
        \PY{n}{ax2}\PY{o}{.}\PY{n}{text}\PY{p}{(}\PY{l+m+mi}{1}\PY{p}{,} \PY{l+m+mf}{2.5}\PY{p}{,} \PY{l+s+sa}{r}\PY{l+s+s1}{\PYZsq{}}\PY{l+s+s1}{\PYZdl{}}\PY{l+s+s1}{\PYZbs{}}\PY{l+s+s1}{mu\PYZdl{} = }\PY{l+s+si}{\PYZpc{}s}\PY{l+s+s1}{\PYZsq{}} \PY{o}{\PYZpc{}} \PY{n+nb}{round}\PY{p}{(}\PY{n}{ajuste\PYZus{}lineal}\PY{p}{(}\PY{n}{degree\PYZus{}2}\PY{p}{,} \PY{n}{annd\PYZus{}2}\PY{p}{)}\PY{p}{[}\PY{l+m+mi}{2}\PY{p}{]}\PY{p}{,} \PY{l+m+mi}{2}\PY{p}{)}\PY{p}{)}
\end{Verbatim}


    \begin{figure}
\centering
\includegraphics{attachment:Ajustes_lineales.png}
\caption{Ajustes\_lineales.png}
\end{figure}

    Se puede observar que el exponente de correlación resultó negativo para
la red July, mientras que para la red Net science resultó positivo.

    \subsubsection{iv)}\label{iv}

    Finalmente se calculo el coeficiente de asortatividad de Newmann para
ambas redes utilizando la funcion de networkx
nx.degree\_assortativity\_coefficient(), como se puede observar en la
siguiente tabla.

    \begin{Verbatim}[commandchars=\\\{\}]
{\color{incolor}In [{\color{incolor}13}]:} \PY{n}{data} \PY{o}{=} \PY{n}{pd}\PY{o}{.}\PY{n}{DataFrame}\PY{p}{(}\PY{p}{\PYZob{}}\PY{l+s+s2}{\PYZdq{}}\PY{l+s+s2}{Red}\PY{l+s+s2}{\PYZdq{}}\PY{p}{:} \PY{p}{[}\PY{l+s+s1}{\PYZsq{}}\PY{l+s+s1}{July}\PY{l+s+s1}{\PYZsq{}}\PY{p}{,}\PY{l+s+s1}{\PYZsq{}}\PY{l+s+s1}{Net Science}\PY{l+s+s1}{\PYZsq{}}\PY{p}{]}\PY{p}{,}
                              \PY{l+s+s2}{\PYZdq{}}\PY{l+s+s2}{Newmann}\PY{l+s+s2}{\PYZdq{}}\PY{p}{:}\PY{p}{[}\PY{o}{\PYZhy{}}\PY{l+m+mf}{0.198}\PY{p}{,}\PY{l+m+mf}{0.46}\PY{p}{]}\PY{p}{,}
                              \PY{l+s+s2}{\PYZdq{}}\PY{l+s+s2}{Barabasi}\PY{l+s+s2}{\PYZdq{}}\PY{p}{:}\PY{p}{[}\PY{o}{\PYZhy{}}\PY{l+m+mf}{0.56} \PY{p}{,}\PY{l+m+mf}{0.34}\PY{p}{]}\PY{p}{,}
                             \PY{p}{\PYZcb{}}\PY{p}{)}\PY{c+c1}{\PYZsh{}empty dataframe}
         \PY{n}{data}
\end{Verbatim}


\begin{Verbatim}[commandchars=\\\{\}]
{\color{outcolor}Out[{\color{outcolor}13}]:}            Red  Newmann  Barabasi
         0         July   -0.198     -0.56
         1  Net Science    0.460      0.34
\end{Verbatim}
            
    Para las dos redes, ambos estimadores no solo resultaron del mismo
orden, sino que también tienen el mismo signo. El hecho de que el
estimador de Newmann dé negativo indica que la red es disortativa,
mientras que si resulta positivo la red es asortativa.

En este caso, la propiedad de la red que se esta estudiando es el
promedio del \(annd\) por grado (\(k_{nn}\)). Por lo tanto, que la red
sea asortativa en este aspecto indica que los nodos con alto grado se
van a relacionar con nodos de alto grado, mientras que los nodos con
grado bajo se van a relacionar con nodos de grado bajo. Es decir, en
general los vecinos de los nodos de alto grado van a tener a su vez un
grado alto, por lo que su \(annd\) va a ser elevado; y los nodos de
grado bajo van a tener vecinos con grado bajo, por lo que su \(annd\) va
a ser pequeño en comparación a los anteriores.

Este estimador es a su vez consistente con el estimador de Barabási, que
indica una pendiente positiva en el \(k_{nn}\) por grado, para redes
asortativas. Es decir, a medida que aumento el grado, también voy a
aumentar el \(k_{nn}\).

Por el contrario, para redes disortativas va a ocurrir lo contrario. Los
nodos con alto grado se van a relacionar en general con nodos de grado
bajo, y viceversa. Por lo tanto, el \(annd\) de los nodos de grado alto
va a ser pequeño, ya que sus vecinos van a tener grado bajo, mientras
que el \(annd\) de los nodos de grado bajo va a ser grande en general,
ya que van a estar relacionados con nodos de grado alto resultando en un
elevado \(annd\).

De nuevo, este estimador es consistente con el estimador de barabási.
Para estas redes indica una pendiente negativa en el \(k_{nn}\) por
grado. Es decir, a medida que aumento el grado del nodo, en promedio el
\(annd\) va a decrecer.

    \subsubsection{Punto b}\label{punto-b}

    En este punto corremos los mismos codigos que en el Punto a, pero para
las redes de proteínas g\_y2h y g\_apms.

    \begin{figure}
\centering
\includegraphics{attachment:multiples\%20graphics\%202.png}
\caption{multiples\%20graphics\%202.png}
\end{figure}

    Luego, si se ajustan los graficos log-log por rectas, se observa que la
red g\_y2h tiene una pendiente negativa, mientras que la red g\_apms
tiene una pendiente positiva.

    \begin{figure}
\centering
\includegraphics{attachment:Ajustes_lineales_2.png}
\caption{Ajustes\_lineales\_2.png}
\end{figure}

    Por otro lado, si se calcula el coeficiente de asortatividad de Newmann
para ambas redes, resulta

    \begin{Verbatim}[commandchars=\\\{\}]
{\color{incolor}In [{\color{incolor}14}]:} \PY{n}{data} \PY{o}{=} \PY{n}{pd}\PY{o}{.}\PY{n}{DataFrame}\PY{p}{(}\PY{p}{\PYZob{}}\PY{l+s+s2}{\PYZdq{}}\PY{l+s+s2}{Red}\PY{l+s+s2}{\PYZdq{}}\PY{p}{:} \PY{p}{[}\PY{l+s+s1}{\PYZsq{}}\PY{l+s+s1}{g\PYZus{}y2h}\PY{l+s+s1}{\PYZsq{}}\PY{p}{,}\PY{l+s+s1}{\PYZsq{}}\PY{l+s+s1}{g\PYZus{}apms}\PY{l+s+s1}{\PYZsq{}}\PY{p}{]}\PY{p}{,}
                              \PY{l+s+s2}{\PYZdq{}}\PY{l+s+s2}{Newmann}\PY{l+s+s2}{\PYZdq{}}\PY{p}{:}\PY{p}{[}\PY{o}{\PYZhy{}}\PY{l+m+mf}{0.06}\PY{p}{,}\PY{l+m+mf}{0.61}\PY{p}{]}\PY{p}{,}
                              \PY{l+s+s2}{\PYZdq{}}\PY{l+s+s2}{Barabasi}\PY{l+s+s2}{\PYZdq{}}\PY{p}{:}\PY{p}{[}\PY{o}{\PYZhy{}}\PY{l+m+mf}{0.19} \PY{p}{,}\PY{l+m+mf}{0.61}\PY{p}{]}\PY{p}{,}
                             \PY{p}{\PYZcb{}}\PY{p}{)}\PY{c+c1}{\PYZsh{}empty dataframe}
         \PY{n}{data}
\end{Verbatim}


\begin{Verbatim}[commandchars=\\\{\}]
{\color{outcolor}Out[{\color{outcolor}14}]:}       Red  Newmann  Barabasi
         0   g\_y2h    -0.06     -0.19
         1  g\_apms     0.61      0.61
\end{Verbatim}
            
    Nuevamente, los coeficientes resultan del mismo orden y del mismo signo
para los dos métodos. Luego, se puede concluir que la red g\_apms es una
red asortativa, donde en promedio los nodos con alto \(annd\) se
relacionan con nodos de alto \(annd\), mientras que los nodos de bajo
\(annd\) se relacionan con nodos de bajo \(annd\). Por otro lado, la red
g\_y2h es una red disortativa en la cual los nodos de bajo grado tienden
a relacionarse con nodos de alto \(annd\) en promedio, y los nodos de
grado alto tienden a relacionarse con nodos de bajo \(annd\) en
promedio.


    % Add a bibliography block to the postdoc
    
    
    
    \end{document}
